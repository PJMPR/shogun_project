% ===========================================================
%  Sylabus: wychowanie fizyczne (WF)
% ===========================================================
\documentclass[12pt, a4paper]{article}

\usepackage[T1]{fontenc}
\usepackage[utf8]{inputenc}
\usepackage[polish]{babel}
\usepackage{lmodern}
\usepackage{microtype}
\usepackage[a4paper, top=2.5cm, bottom=2.5cm, left=2.5cm, right=2.5cm]{geometry}
\usepackage{xcolor}
\usepackage{graphicx}
\usepackage{booktabs}
\usepackage{tabularx}
\usepackage{longtable}
\usepackage{multirow}
\usepackage{array}
\usepackage{colortbl}
\usepackage{enumitem}
\usepackage{fancyhdr}
\usepackage{titlesec}
\usepackage{mdframed}
\usepackage[colorlinks=true, linkcolor=red!70!black, urlcolor=red!70!black]{hyperref}
\usepackage{eso-pic}
\usepackage{tikz}

\definecolor{pjatkRed}{RGB}{180,0,0}
\definecolor{pjatkGray}{RGB}{80,80,80}
\definecolor{pjatkLightGray}{RGB}{245,245,245}
\definecolor{tableHeader}{RGB}{220,220,220}

\pagestyle{fancy}\fancyhf{}
\renewcommand{\headrulewidth}{0.4pt}
\renewcommand{\footrulewidth}{0.4pt}
\fancyhead[L]{\small\textcolor{pjatkGray}{PJATK -- Filia w Gdańsku \textbar\ Informatyka}}
\fancyhead[R]{\small\textcolor{pjatkGray}{Sylabus: WF}}
\fancyfoot[C]{\small\thepage}

\titleformat{\section}{\large\bfseries\color{pjatkRed}}{\thesection.}{0.5em}{}
  [\color{pjatkRed}\rule{\linewidth}{0.8pt}]
\setlist{noitemsep, topsep=3pt, parsep=2pt}

\newmdenv[linecolor=pjatkRed, linewidth=1.2pt, backgroundcolor=pjatkLightGray,
  innerleftmargin=10pt, innerrightmargin=10pt, innertopmargin=8pt,
  innerbottommargin=8pt, roundcorner=4pt]{infobox}

\begin{document}

\AddToShipoutPictureBG{%
  \begin{tikzpicture}[remember picture, overlay]
    \node[opacity=0.5] at (current page.center) {%
      \includegraphics[width=14cm]{C:/Users/adamu/WebstormProjects/pj-studies/latex/PJATK_pl_sygnet_transparent-eps-converted-to}%
    };
  \end{tikzpicture}%
}

\begin{center}
  \includegraphics[height=2cm]{C:/Users/adamu/WebstormProjects/pj-studies/latex/PJATK_pl_poziom_1}\\[0.8cm]
  {\LARGE\bfseries\color{pjatkRed} SYLABUS PRZEDMIOTU}\\[0.8cm]
\end{center}

\begin{infobox}
\begin{tabularx}{\textwidth}{@{}lX@{}}
  \textbf{Nazwa przedmiotu:}  & {\bfseries wychowanie fizyczne} \\[3pt]
  \textbf{Kod przedmiotu:}    & WF \\[3pt]
  \textbf{Kierunek / Profil:} & Informatyka / praktyczny \\[3pt]
  \textbf{Tryb studiów:}      & stacjonarny \\[3pt]
  \textbf{Rok / Semestr:}     &  /  \\[3pt]
  \textbf{Charakter:}         & obowiązkowy \\[3pt]
  \textbf{Odpowiedzialny:}    &  \\[3pt]
  \textbf{Wersja z dnia:}     & 19.02.2026 \\
\end{tabularx}
\end{infobox}

\vspace{1cm}

\section{Godziny zajęć i punkty ECTS}

\begin{center}
\begin{tabular}{|>{\centering\arraybackslash}p{2.0cm}
                |>{\centering\arraybackslash}p{2.0cm}
                |>{\centering\arraybackslash}p{2.0cm}
                |>{\centering\arraybackslash}p{2.4cm}
                |>{\centering\arraybackslash}p{2.4cm}
                |>{\centering\arraybackslash}p{2.0cm}
                |>{\centering\arraybackslash}p{1.4cm}|}
\hline
\rowcolor{tableHeader}
\textbf{Wykłady} & \textbf{Ćwiczenia} & \textbf{Laboratorium} &
\textbf{Z prowadzącym} & \textbf{Praca własna} & \textbf{Łącznie} & \textbf{ECTS} \\
\hline
--- & 30 h & --- & 30 h & --- & 30 h & \textbf{0} \\
\hline
\end{tabular}
\end{center}

\section{Cel dydaktyczny}

kształtowanie zdolności rozpoznawania i oceny własnej sprawności fizycznej i rozwoju fizycznego, zachęcanie do uczestnictwa w sportowych i rekreacyjnych formach aktywności fizycznej, poznanie i umiejętne stosowanie zasad bezpieczeństwa podczas aktywności fizycznej, zrozumienie związku aktywności fizycznej ze zdrowiem oraz kształtowanie umiejętności praktykowania zachowań prozdrowotnych. Zajęcia na basenie, saunie ,siłowni, sali gimnastycznej

\section{Przedmioty wprowadzające}

\begin{tabularx}{\textwidth}{lX}
  \hline
  \textbf{Przedmiot} & \textbf{Wymagane zagadnienia} \\
  \hline
  ----------------------------------------------------------- & ------------------------------------------------------------------ \\
  \hline
\end{tabularx}

\section{Treści programowe}

\begin{enumerate}
  \item Wszechstronny rozwój organizmu, korygowanie wad postawy, przez odpowiedni dobór środków i metod stymulujących i korygujących rozwój i funkcjonowanie układu ruchowego, sercowo-naczyniowego, oddechowego i nerwowego.
  \item Rozwój sprawności kondycyjnej i koordynacyjnej oraz dostarczenie studentom wiadomości i umiejętności umożliwiających samokontrolę, samoocenę i samodzielne podejmowanie działań w tym zakresie.
  \item Wykształcenie umiejętności ruchowych przydatnych w aktywności zdrowotnej, utylitarnej, rekreacyjnej i sportowej.
  \item Wyposażenie studentów w niezbędną wiedzę i umiejętności umożliwiające bezpieczną organizację zajęć ruchowych w różnych warunkach środowiskowych, indywidualnie, w grupie rówieśniczej oraz w rodzinie.
  \item Ukształtowanie postawy świadomego i permanentnego uczestnictwa studentów w różnych formach aktywności sportowo-rekreacyjnych w czasie nauki oraz po jej ukończeniu dla zachowania zdrowia fizycznego i psychicznego.
  \item Kształtowanie postaw osobowościowych: poczucia własnej wartości, szacunku dla innych osób, zwłaszcza słabszych i mniej sprawnych.
  \item Kształtowanie współdziałania w zespole, grupie, akceptacji siebie i innych, kultury kibicowania, stosowania zasady „fair play” w sporcie i w życiu.
  \item Szczegółowy program przedmiotu w rozbiciu na tygodnie/bloki tematyczne jest dostępny u prowadzącego zajęcia i koordynatora WF(uzależniony jest od semestru i form realizacji WF)
\end{enumerate}

\section{Efekty kształcenia}

\subsection*{Wiedza}
\begin{itemize}
  \item Brak danych.
\end{itemize}

\subsection*{Umiejętności}
\begin{itemize}
  \item student jest gotów do nauki o kulturzefizycznej
  \item krytycznej oceny posiadanej wiedzy i odbieranych treści uznawania znaczenia wiedzy w rozwiązywaniu problemów poznawczych i praktycznych oraz zasięganiaopinii ekspertów w przypadku trudności z samodzielnym rozwiązaniem problemu
  \item tworzenia i rozwijania wzorów właściwegopostępowania w środowisku pracy i życiapodejmowania inicjatyw, krytycznej oceny siebie oraz zespołów i organizacji, wktórych uczestniczy.
\end{itemize}

\subsection*{Kompetencje społeczne}
\begin{itemize}
  \item Znajomość tematyki: rozwój fizyczny, rozwój sprawności fizycznej, zdrowy styl życia, higiena, hartowanie organizmu.
  \item Znajomość i przestrzeganie zasad bezpieczeństwa na obiektach sportowych.
  \item Znaomość regulaminów, przepisów i zasad poznanych dyscyplin sportowych.
  \item Stosowanie zasady „czystej gry” i sportowego kibicowania.
\end{itemize}

\section{Kryteria oceny}

\begin{itemize}
  \item Ćwiczenia – praca w grupach,
  \item Praca w parach, praca indywidualna
  \item Ćwiczenia :
  \item Zaliczenie bez oceny
  \item Wykład --------------------------------
  \item Zaliczenie bez oceny
  \item Kryteria oceny
  \item Ćwiczenia :
  \item -Aktywna obecność na zajęciach, maksymalnie 2 nieusprawiedliwione nieobecności w czasie semestru(zwolnienie lekarskie przedstawione zgodnie z regulaminem).Obecność poświadczona jest w dzienniku zajęć.
  \item -Aktywna obecność na zajęciach w wymiarze 45 minut
  \item -Zajęcia spoza oferty PJATK poświadczone zaświadczeniem (zgodnie z załącznikiem 1 w regulaminie zaliczenia przedmiotu)
  \item -W przypadku długotrwałej choroby lub innych przeciwskazań zdrowotnych możliwe jest zwolnienie lekarskie od specjalisty.
  \item Kryteria zaliczenia:
  \item -aktywne uczestnictwo w zajęciach
  \item -maksymalnie 2 nieobecności(w przypadku zajęć realizowanych przez PJATK)
  \item -obecność 100\%(w przypadku zajęć spoza oferty PJATK
  \item -obecność 100\% w przypadku ITN
  \item .
\end{itemize}

\section{Metody dydaktyczne}

Wykład, laboratoria, praca własna studenta.

\section{Literatura}

\textbf{Podstawowa:}
\begin{itemize}
  \item Brak danych.
\end{itemize}

\textbf{Uzupełniająca:}
\begin{itemize}
  \item Brak danych.
\end{itemize}

\end{document}
