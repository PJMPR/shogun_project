% ===========================================================
%  Sylabus: Podstawy Programowania (PRG1)
% ===========================================================
\documentclass[12pt, a4paper]{article}

\usepackage[T1]{fontenc}
\usepackage[utf8]{inputenc}
\usepackage[polish]{babel}
\usepackage{lmodern}
\usepackage{microtype}
\usepackage[a4paper, top=2.5cm, bottom=2.5cm, left=2.5cm, right=2.5cm]{geometry}
\usepackage{xcolor}
\usepackage{graphicx}
\usepackage{booktabs}
\usepackage{tabularx}
\usepackage{longtable}
\usepackage{multirow}
\usepackage{array}
\usepackage{colortbl}
\usepackage{enumitem}
\usepackage{fancyhdr}
\usepackage{titlesec}
\usepackage{mdframed}
\usepackage[colorlinks=true, linkcolor=red!70!black, urlcolor=red!70!black]{hyperref}
\usepackage{eso-pic}
\usepackage{tikz}

\definecolor{pjatkRed}{RGB}{180,0,0}
\definecolor{pjatkGray}{RGB}{80,80,80}
\definecolor{pjatkLightGray}{RGB}{245,245,245}
\definecolor{tableHeader}{RGB}{220,220,220}

\pagestyle{fancy}\fancyhf{}
\renewcommand{\headrulewidth}{0.4pt}
\renewcommand{\footrulewidth}{0.4pt}
\fancyhead[L]{\small\textcolor{pjatkGray}{PJATK -- Filia w Gdańsku \textbar\ Informatyka}}
\fancyhead[R]{\small\textcolor{pjatkGray}{Sylabus: PRG1}}
\fancyfoot[C]{\small\thepage}

\titleformat{\section}{\large\bfseries\color{pjatkRed}}{\thesection.}{0.5em}{}
  [\color{pjatkRed}\rule{\linewidth}{0.8pt}]
\setlist{noitemsep, topsep=3pt, parsep=2pt}

\newmdenv[linecolor=pjatkRed, linewidth=1.2pt, backgroundcolor=pjatkLightGray,
  innerleftmargin=10pt, innerrightmargin=10pt, innertopmargin=8pt,
  innerbottommargin=8pt, roundcorner=4pt]{infobox}

\begin{document}

\AddToShipoutPictureBG{%
  \begin{tikzpicture}[remember picture, overlay]
    \node[opacity=0.5] at (current page.center) {%
      \includegraphics[width=14cm]{C:/Users/adamu/WebstormProjects/pj-studies/latex/PJATK_pl_sygnet_transparent-eps-converted-to}%
    };
  \end{tikzpicture}%
}

\begin{center}
  \includegraphics[height=2cm]{C:/Users/adamu/WebstormProjects/pj-studies/latex/PJATK_pl_poziom_1}\\[0.8cm]
  {\LARGE\bfseries\color{pjatkRed} SYLABUS PRZEDMIOTU}\\[0.8cm]
\end{center}

\begin{infobox}
\begin{tabularx}{\textwidth}{@{}lX@{}}
  \textbf{Nazwa przedmiotu:}  & {\bfseries Podstawy Programowania} \\[3pt]
  \textbf{Kod przedmiotu:}    & PRG1 \\[3pt]
  \textbf{Kierunek / Profil:} & Informatyka / praktyczny \\[3pt]
  \textbf{Tryb studiów:}      & stacjonarny \\[3pt]
  \textbf{Rok / Semestr:}     & 1 / 1 \\[3pt]
  \textbf{Charakter:}         & obowiązkowy \\[3pt]
  \textbf{Odpowiedzialny:}    & mgr inż. Adam Urbanowicz \\[3pt]
  \textbf{Wersja z dnia:}     & 19.02.2026 \\
\end{tabularx}
\end{infobox}

\vspace{1cm}

\section{Godziny zajęć i punkty ECTS}

\begin{center}
\begin{tabular}{|>{\centering\arraybackslash}p{2.0cm}
                |>{\centering\arraybackslash}p{2.0cm}
                |>{\centering\arraybackslash}p{2.0cm}
                |>{\centering\arraybackslash}p{2.4cm}
                |>{\centering\arraybackslash}p{2.4cm}
                |>{\centering\arraybackslash}p{2.0cm}
                |>{\centering\arraybackslash}p{1.4cm}|}
\hline
\rowcolor{tableHeader}
\textbf{Wykłady} & \textbf{Ćwiczenia} & \textbf{Laboratorium} &
\textbf{Z prowadzącym} & \textbf{Praca własna} & \textbf{Łącznie} & \textbf{ECTS} \\
\hline
30 h & --- & 60 h & 56 h & 94 h & 150 h & \textbf{6} \\
\hline
\end{tabular}
\end{center}

\section{Forma zajęć}

\begin{tabular}{ll}
  \hline
  \textbf{Forma zajęć} & \textbf{Sposób zaliczenia} \\
  \hline
  Laboratorium & Zaliczenie z oceną \\
  \hline
\end{tabular}

\section{Cel dydaktyczny}

Celem zajęć jest opanowanie przez studentów podstaw klasycznych technik programowania strukturalnego na bazie języka C\#. Wprowadzone zostają również elementy programowania obiektowego. Zajęcia mają na celu rozwijanie umiejętności abstrakcyjnego myślenia oraz rozwiązywania prostych problemów programistycznych z wykorzystaniem platformy .NET i aplikacji konsolowych.

\section{Treści programowe}

\begin{enumerate}
  \item Pojęcie algorytmu. Sposoby przedstawiania algorytmów. Zmienne, typy, operator przypisania, instrukcje Console.WriteLine i Console.ReadLine, pierwszy program w C\#.
  \item Wprowadzenie do środowiska .NET i Visual Studio. Tworzenie projektu, edycja kodu źródłowego, kompilacja, uruchamianie i debugowanie aplikacji konsolowych. Implementacja prostych obliczeń matematycznych.
  \item Instrukcja warunkowa if, if else, zagnieżdżona instrukcja if. Instrukcja switch.
  \item Implementacja programów wymagających użycia instrukcji warunkowych i wielowariantowych.
  \item Pętle for, while, do while. Różnice i zastosowania. Zagnieżdżanie pętli. Instrukcje break i continue.
  \item Implementacja programów wymagających użycia pętli.
  \item Typy całkowite i zmiennoprzecinkowe. Typ znakowy i logiczny. Konwersje typów i rzutowania. Kwalifikator const. Operatory arytmetyczne, relacyjne i logiczne. Priorytety operatorów.
  \item Implementacja programów z wykorzystaniem zmiennych różnych typów oraz operatorów.
  \item Tablice jednowymiarowe i wielowymiarowe. Inicjalizacja tablic. Podstawowe algorytmy sortowania.
  \item Implementacja programów wykorzystujących tablice jednowymiarowe i wielowymiarowe.
  \item Metody. Przekazywanie parametrów. Zwracanie wartości. Metody void. Przeciążanie metod. Metody biblioteczne.
  \item Definiowanie i stosowanie metod. Implementacja aplikacji wykorzystujących metody.
  \item Rekurencja. Wady i zalety stosowania rekurencji. Klasa string i jej metody.
  \item Implementacja programów wykorzystujących rekurencję oraz operacje na łańcuchach znakowych.
  \item Typy referencyjne i wartościowe. Wprowadzenie do kolekcji: List, ArrayList. Zarządzanie pamięcią w .NET. Garbage Collector.
  \item Implementacja programów wykorzystujących kolekcje i dynamiczne struktury danych.
  \item Wstęp do testowania jednostkowego
\end{enumerate}

\section{Efekty kształcenia}

\subsection*{Wiedza}
\begin{itemize}
  \item Wymienia zasady programowania strukturalnego i obiektowego w języku C\#, w tym techniki iteracji, rekurencji, organizacji pamięci zarządzanej, obsługi typów wartościowych i referencyjnych. Opisuje podstawowe algorytmy wyszukiwania i sortowania danych.
\end{itemize}

\subsection*{Umiejętności}
\begin{itemize}
  \item Pozyskuje informacje dotyczące programowania z dokumentacji technicznej, literatury i Internetu. Konstruuje rozwiązania prostych zadań programistycznych w języku C\# wymagających użycia instrukcji warunkowych, pętli, rekurencji, metod z odpowiednim sposobem przekazywania parametrów, tablic i kolekcji. Testuje poprawność wykonywanego oprogramowania oraz wykorzystuje zintegrowane środowisko programistyczne do konfigurowania, edytowania i testowania aplikacji konsolowych.
\end{itemize}

\section{Kryteria oceny}

\begin{itemize}
  \item Ćwiczenia / Laboratorium/Lektorat:
  \item rozwiązywanie zadań programistycznych
  \item Kryteria oceny
  \item Ćwiczenia/Laboratorium/Projekt/Lektorat
  \item Zadania laboratoryjne (czas oddawania: następne zajęcia) - 20\%
  \item Kolokwia (dwa): 60\%
  \item Projekt programistyczny: 20\%
  \item Brak
\end{itemize}

\section{Metody dydaktyczne}

Wykład, laboratoria, praca własna studenta.

\section{Literatura}

\textbf{Podstawowa:}
\begin{itemize}
  \item Andrew Troelsen, Phil Japikse, Pro C\# 10 with .NET 6, Apress, 2022
  \item Ian Griffiths, Programming C\# 10, O'Reilly Media, 2022
\end{itemize}

\textbf{Uzupełniająca:}
\begin{itemize}
  \item Jon Skeet, C\# in Depth, Fourth Edition, Manning Publications, 2019
  \item Robert C. Martin, Czysty kod. Podręcznik dobrego programisty, Helion, 2010
\end{itemize}

\end{document}
