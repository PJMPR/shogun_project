% ===========================================================
%  Sylabus: Użytkowanie Komputerów i Podstawy Systemów Operacyjnych (UKOS)
% ===========================================================
\documentclass[12pt, a4paper]{article}

\usepackage[T1]{fontenc}
\usepackage[utf8]{inputenc}
\usepackage[polish]{babel}
\usepackage{lmodern}
\usepackage{microtype}
\usepackage[a4paper, top=2.5cm, bottom=2.5cm, left=2.5cm, right=2.5cm]{geometry}
\usepackage{xcolor}
\usepackage{graphicx}
\usepackage{booktabs}
\usepackage{tabularx}
\usepackage{longtable}
\usepackage{multirow}
\usepackage{array}
\usepackage{colortbl}
\usepackage{enumitem}
\usepackage{fancyhdr}
\usepackage{titlesec}
\usepackage{mdframed}
\usepackage[colorlinks=true, linkcolor=red!70!black, urlcolor=red!70!black]{hyperref}
\usepackage{eso-pic}
\usepackage{tikz}

\definecolor{pjatkRed}{RGB}{180,0,0}
\definecolor{pjatkGray}{RGB}{80,80,80}
\definecolor{pjatkLightGray}{RGB}{245,245,245}
\definecolor{tableHeader}{RGB}{220,220,220}

\pagestyle{fancy}\fancyhf{}
\renewcommand{\headrulewidth}{0.4pt}
\renewcommand{\footrulewidth}{0.4pt}
\fancyhead[L]{\small\textcolor{pjatkGray}{PJATK -- Filia w Gdańsku \textbar\ Informatyka}}
\fancyhead[R]{\small\textcolor{pjatkGray}{Sylabus: UKOS}}
\fancyfoot[C]{\small\thepage}

\titleformat{\section}{\large\bfseries\color{pjatkRed}}{\thesection.}{0.5em}{}
  [\color{pjatkRed}\rule{\linewidth}{0.8pt}]
\setlist{noitemsep, topsep=3pt, parsep=2pt}

\newmdenv[linecolor=pjatkRed, linewidth=1.2pt, backgroundcolor=pjatkLightGray,
  innerleftmargin=10pt, innerrightmargin=10pt, innertopmargin=8pt,
  innerbottommargin=8pt, roundcorner=4pt]{infobox}

\begin{document}

\AddToShipoutPictureBG{%
  \begin{tikzpicture}[remember picture, overlay]
    \node[opacity=0.5] at (current page.center) {%
      \includegraphics[width=14cm]{C:/Users/adamu/WebstormProjects/pj-studies/latex/PJATK_pl_sygnet_transparent-eps-converted-to}%
    };
  \end{tikzpicture}%
}

\begin{center}
  \includegraphics[height=2cm]{C:/Users/adamu/WebstormProjects/pj-studies/latex/PJATK_pl_poziom_1}\\[0.8cm]
  {\LARGE\bfseries\color{pjatkRed} SYLABUS PRZEDMIOTU}\\[0.8cm]
\end{center}

\begin{infobox}
\begin{tabularx}{\textwidth}{@{}lX@{}}
  \textbf{Nazwa przedmiotu:}  & {\bfseries Użytkowanie Komputerów i Podstawy Systemów Operacyjnych} \\[3pt]
  \textbf{Kod przedmiotu:}    & UKOS \\[3pt]
  \textbf{Kierunek / Profil:} & Informatyka / praktyczny \\[3pt]
  \textbf{Tryb studiów:}      & niestacjonarny \\[3pt]
  \textbf{Rok / Semestr:}     & 1 / 1 \\[3pt]
  \textbf{Charakter:}         & obowiązkowy \\[3pt]
  \textbf{Odpowiedzialny:}    & Mgr inż. Michał Hyla \\[3pt]
  \textbf{Wersja z dnia:}     & 19.02.2026 \\
\end{tabularx}
\end{infobox}

\vspace{1cm}

\section{Godziny zajęć i punkty ECTS}

\begin{center}
\begin{tabular}{|>{\centering\arraybackslash}p{2.0cm}
                |>{\centering\arraybackslash}p{2.0cm}
                |>{\centering\arraybackslash}p{2.0cm}
                |>{\centering\arraybackslash}p{2.4cm}
                |>{\centering\arraybackslash}p{2.4cm}
                |>{\centering\arraybackslash}p{2.0cm}
                |>{\centering\arraybackslash}p{1.4cm}|}
\hline
\rowcolor{tableHeader}
\textbf{Wykłady} & \textbf{Ćwiczenia} & \textbf{Laboratorium} &
\textbf{Z prowadzącym} & \textbf{Praca własna} & \textbf{Łącznie} & \textbf{ECTS} \\
\hline
30 h & --- & 30 h & 60 h & 65 h & 125 h & \textbf{5} \\
\hline
\end{tabular}
\end{center}

\section{Forma zajęć}

\begin{tabular}{ll}
  \hline
  \textbf{Forma zajęć} & \textbf{Sposób zaliczenia} \\
  \hline
  Laboratorium & Zaliczenie z oceną \\
  \hline
\end{tabular}

\section{Cel dydaktyczny}

Cele dydaktyczne przedmiotu obejmują zapoznanie studentów z podstawami pracy w systemie Linux, w tym z obsługą wiersza poleceń oraz automatyzacją zadań za pomocą skryptów bash. Kluczowe jest również zrozumienie działania systemów kontroli wersji (np. git), co pomaga w efektywnej pracy zespołowej w projektach programistycznych. Cele poboczne obejmują bardziej praktyczne umiejętności, takie jak tworzenie funkcji i skryptów automatyzujących codzienne zadania programistyczne, praca z plikami, pakowanie danych, oraz korzystanie z narzędzi takich jak Midnight Commander czy FTP

\section{Przedmioty wprowadzające}

\begin{tabularx}{\textwidth}{lX}
  \hline
  \textbf{Przedmiot} & \textbf{Wymagane zagadnienia} \\
  \hline
  Nie dotyczy & Nie dotyczy \\
  \hline
\end{tabularx}

\section{Treści programowe}

\begin{enumerate}
  \item Wprowadzenie do systemu Linux oraz interfejsu wiersza poleceń (CLI)Studenci poznają podstawowe pojęcia dotyczące systemów operacyjnych, obsługi systemu Linux oraz podstawowych komend CLI.
  \item Praca z systemami plików i zarządzanie nimiStudenci zdobywają umiejętności w zakresie pracy z systemami plików, tworzenia, usuwania i zarządzania plikami oraz katalogami w systemie Linux.
  \item Skrypty BashTworzenie i uruchamianie skryptów Bash, automatyzacja zadań systemowych, praca z pętlami, funkcjami i warunkami.
  \item Zarządzanie procesami w systemie operacyjnymOmówienie zarządzania procesami w systemie Linux, monitorowanie ich działania, manipulowanie zadaniami i zarządzanie współbieżnością.
  \item Kontrola wersji z użyciem GITPraktyczne wprowadzenie do systemów kontroli wersji, takich jak GIT, oraz zarządzanie wersjami oprogramowania.
  \item Architektura systemów wieloprocesorowychZrozumienie architektury systemów komputerowych, w tym wieloprocesorowych, oraz zarządzanie współbieżnością.
  \item Podstawy sieci komputerowych i bezpieczeństwaStudenci poznają technologie sieciowe, protokoły komunikacyjne oraz podstawowe zagadnienia związane z bezpieczeństwem.
  \item Instalacja i konfiguracja systemów operacyjnychProces instalacji i konfiguracji systemów operacyjnych, a także podstawowa administracja systemami.
  \item Problemy synchronizacji zadańRozwiązywanie klasycznych problemów synchronizacji zadań, dobór algorytmów szeregowania i analiza ich efektywności.
  \item Symulacja komputerowa i analiza systemów operacyjnychStudenci wykonują symulacje komputerowe w celu analizy działania systemów operacyjnych oraz ich parametrów.
\end{enumerate}

\section{Efekty kształcenia}

\subsection*{Wiedza}
\begin{itemize}
  \item Student zna i rozumie zasady działania architektury systemów komputerowych oraz systemu Linux.
  \item Student zna podstawowe pojęcia dotyczące systemów operacyjnych, ich budowy i działania
  \item Student rozumie zasady działania sieci komputerowych, protokołów komunikacyjnych i zagadnień bezpieczeństwa.
\end{itemize}

\subsection*{Umiejętności}
\begin{itemize}
  \item Student jest w stenie bezpiecznie korzystać z komputera
  \item Student potrafi zainstalować, skonfigurować i administrować system operacyjny
  \item Student potrafi zainstalować kompletny system (OS, baza danych, aplikacje) i go uruchomić
\end{itemize}

\section{Kryteria oceny}

\begin{itemize}
  \item Laboratorium/:
  \item Praca indywidualna ze skryptem
  \item Praca grupowa
  \item Warsztat
  \item Kryteria oceny
  \item Laboratorium: 2 kolokwia, prace domowe
\end{itemize}

\section{Metody dydaktyczne}

Wykład, laboratoria, praca własna studenta.

\section{Literatura}

\textbf{Podstawowa:}
\begin{itemize}
  \item Sosna Ł. Linux. Komendy i polecenia. VI ed. Gliwice: Helion; 2023.
  \item Negus C. Linux. Biblia. 10th ed. Gliwice: Helion; 2021.
\end{itemize}

\textbf{Uzupełniająca:}
\begin{itemize}
  \item Tanenbaum AS, Bos H. Systemy operacyjne. 4th ed. Gliwice: Helion; 2023.
\end{itemize}

\end{document}
