% ===========================================================
%  Sylabus: Zarządzanie własnym przedsięwzięciem (POZ)
% ===========================================================
\documentclass[12pt, a4paper]{article}

\usepackage[T1]{fontenc}
\usepackage[utf8]{inputenc}
\usepackage[polish]{babel}
\usepackage{lmodern}
\usepackage{microtype}
\usepackage[a4paper, top=2.5cm, bottom=2.5cm, left=2.5cm, right=2.5cm]{geometry}
\usepackage{xcolor}
\usepackage{graphicx}
\usepackage{booktabs}
\usepackage{tabularx}
\usepackage{longtable}
\usepackage{multirow}
\usepackage{array}
\usepackage{colortbl}
\usepackage{enumitem}
\usepackage{fancyhdr}
\usepackage{titlesec}
\usepackage{mdframed}
\usepackage[colorlinks=true, linkcolor=red!70!black, urlcolor=red!70!black]{hyperref}
\usepackage{eso-pic}
\usepackage{tikz}

\definecolor{pjatkRed}{RGB}{180,0,0}
\definecolor{pjatkGray}{RGB}{80,80,80}
\definecolor{pjatkLightGray}{RGB}{245,245,245}
\definecolor{tableHeader}{RGB}{220,220,220}

\pagestyle{fancy}\fancyhf{}
\renewcommand{\headrulewidth}{0.4pt}
\renewcommand{\footrulewidth}{0.4pt}
\fancyhead[L]{\small\textcolor{pjatkGray}{PJATK -- Filia w Gdańsku \textbar\ Informatyka}}
\fancyhead[R]{\small\textcolor{pjatkGray}{Sylabus: POZ}}
\fancyfoot[C]{\small\thepage}

\titleformat{\section}{\large\bfseries\color{pjatkRed}}{\thesection.}{0.5em}{}
  [\color{pjatkRed}\rule{\linewidth}{0.8pt}]
\setlist{noitemsep, topsep=3pt, parsep=2pt}

\newmdenv[linecolor=pjatkRed, linewidth=1.2pt, backgroundcolor=pjatkLightGray,
  innerleftmargin=10pt, innerrightmargin=10pt, innertopmargin=8pt,
  innerbottommargin=8pt, roundcorner=4pt]{infobox}

\begin{document}

\AddToShipoutPictureBG{%
  \begin{tikzpicture}[remember picture, overlay]
    \node[opacity=0.5] at (current page.center) {%
      \includegraphics[width=14cm]{C:/Users/adamu/WebstormProjects/pj-studies/latex/PJATK_pl_sygnet_transparent-eps-converted-to}%
    };
  \end{tikzpicture}%
}

\begin{center}
  \includegraphics[height=2cm]{C:/Users/adamu/WebstormProjects/pj-studies/latex/PJATK_pl_poziom_1}\\[0.8cm]
  {\LARGE\bfseries\color{pjatkRed} SYLABUS PRZEDMIOTU}\\[0.8cm]
\end{center}

\begin{infobox}
\begin{tabularx}{\textwidth}{@{}lX@{}}
  \textbf{Nazwa przedmiotu:}  & {\bfseries Zarządzanie własnym przedsięwzięciem} \\[3pt]
  \textbf{Kod przedmiotu:}    & POZ \\[3pt]
  \textbf{Kierunek / Profil:} & Informatyka / praktyczny \\[3pt]
  \textbf{Tryb studiów:}      & niestacjonarny \\[3pt]
  \textbf{Rok / Semestr:}     & 4 / 7 \\[3pt]
  \textbf{Charakter:}         & obieralny \\[3pt]
  \textbf{Odpowiedzialny:}    & dr Albert Śledzianowski \\[3pt]
  \textbf{Wersja z dnia:}     & 19.02.2026 \\
\end{tabularx}
\end{infobox}

\vspace{1cm}

\section{Godziny zajęć i punkty ECTS}

\begin{center}
\begin{tabular}{|>{\centering\arraybackslash}p{2.0cm}
                |>{\centering\arraybackslash}p{2.0cm}
                |>{\centering\arraybackslash}p{2.0cm}
                |>{\centering\arraybackslash}p{2.4cm}
                |>{\centering\arraybackslash}p{2.4cm}
                |>{\centering\arraybackslash}p{2.0cm}
                |>{\centering\arraybackslash}p{1.4cm}|}
\hline
\rowcolor{tableHeader}
\textbf{Wykłady} & \textbf{Ćwiczenia} & \textbf{Laboratorium} &
\textbf{Z prowadzącym} & \textbf{Praca własna} & \textbf{Łącznie} & \textbf{ECTS} \\
\hline
8 h & 8 h & --- & 16 h & 34 h & 50 h & \textbf{2} \\
\hline
\end{tabular}
\end{center}

\section{Forma zajęć}

\begin{tabular}{ll}
  \hline
  \textbf{Forma zajęć} & \textbf{Sposób zaliczenia} \\
  \hline
  Ćwiczenia & Zaliczenie z oceną \\
  Wykład & Nieoceniany \\
  \hline
\end{tabular}

\section{Cel dydaktyczny}

Celem przedmiotu jest omówienie podstawowych kwestii ekonomicznych: gospodarka, system ekonomiczny i społeczny; pieniądz i rynek; uwarunkowania technologiczne, etyczne i prawne; gospodarka oparta na informacji i na wiedzy; marketing; podstawy rachunkowości; formy prowadzenia działalności gospodarczej; zakładanie i zarządzanie firmą

\section{Przedmioty wprowadzające}

\begin{tabularx}{\textwidth}{lX}
  \hline
  \textbf{Przedmiot} & \textbf{Wymagane zagadnienia} \\
  \hline
  brak & brak \\
  \hline
\end{tabularx}

\section{Treści programowe}

\begin{enumerate}
  \item Pojęcia podstawowe: rynek, popyt i podaż.Nowoczesny system ekonomiczny ispołeczny. Ekonomia, technologia i etyka.
  \item Rynek, Konkurencja. Wartość i powstawanieceny. Pieniądz. Pieniądz w świecie IT.
  \item Państwo a gospodarka. Podatki. Inflacja.Bezrobocie. Rola pieniądza w gospodarce.Rola rządu. Prawa i obowiązki przedsiębiorcyi obywatela.Zarządzanie czasem. Wykresy Gantta
  \item Nowoczesna makrogospodarka . Gospodarka globalna. Strategie rozwoju światowego.Kryzysy ekonomiczne i kryzysy finansowe.Rozwój gospodarczy i rozwój społeczny.Modele rozwoju ekonomicznego.
  \item Ożywianie i schładzanie gospodarki.Gospodarka USA vs gospodarka UE. Strategializbońska. Japonia. Chiny i Indie. Korporacje ifirmy międzynarodowe. Przejęcia wrogie iprzyjazne. Przepływ kapitału i inwestycje FDI.Kryzysy finansowe.Budżetowanie przedsięwzięć i rachunek cash flow.
  \item Giełda jako przykład organizacji rynkuinformacji.Planowanie działań i formułowanie strategii.
  \item Gospodarka oparta na wiedzy. Firmainnowacyjna. Rynek e-usług. Handelelektroniczny.Logistyka (MRP)
  \item Podstawy marketingu. Misja i wizja firmy.Budowa wizerunku. Budowa produktu.Marketing. Analiza business case.
  \item Marketing: reklama a zarządzanie rynkiem.Kanały informacyjne. Grupa targetowa.Komunikacja wewnętrzna i komunikacja zewnętrzna. Kampanie marketingowe.Nowoczesny marketing. Marketing w świecieglobalnej wymiany komunikacji.Budowanie kampanii marketingowej
  \item Marketing. Logotypy i system identyfikacjiwizualnej. Kampanie społeczne. Firmaodpowiedzialna społecznie. Reagowanie nasytuacje kryzysowe.
  \item Budowa i zarządzanie własnegoprzedsiębiorstwa. Formy prowadzeniadziałalności gospodarczej.Symulator ekonomiczny.
  \item Podstawy rachunkowości finansowej. Bilansspółki. Wycena firmy. Rachunkowośćzarządcza. Rachunek ekonomiczny firmy.Rachunkowość i cash flow.
  \item Umocowanie prawne przedsiębiorcy.Elementy prawa dla menedżerów. Podstawyprawa dla informatyków. Kodeks cywilny.Kodeks spółek handlowych.
  \item Ustawa o prawie autorskim. Analiza przypadków prawnych.
\end{enumerate}

\section{Efekty kształcenia}

\subsection*{Wiedza}
\begin{itemize}
  \item Student zna i rozumie w jaki sposób zaplanować przedsięwzięcie informatyczne. Zna sposób wstępnej oceny ekonomicznej, aspektów społecznych oraz analizy wykonalności przedsięwzięcia.
  \item Student zna i rozumie podstawowe problemów etycznych, społecznych i zawodowych informatyki oraz odpowiedzialności związanej z działalnością
  \item w obszarze informatyki; ma podstawową wiedzę w zakresie ochrony własności intelektualnej oraz prawa patentowego i autorskiego.
  \item Student zna i rozumie zasady prowadzenia działalności
  \item gospodarczej, szczególnie przedsięwzięć. Zna ogólne zasady tworzenia przedsiębiorczości, szczególnie w zakresie zastosowań rozwiązań informatycznych
\end{itemize}

\subsection*{Umiejętności}
\begin{itemize}
  \item Student potrafi przeanalizować projekt informatyczny pod względem wpływu na otoczenie oraz jego opłacalności.
\end{itemize}

\subsection*{Kompetencje społeczne}
\begin{itemize}
  \item Student jest gotów do podejmowania starania, aby przekazać informacje i opinie w sposób powszechnie zrozumiały
  \item Student jest gotów do działania w sposób przedsiębiorczy.
\end{itemize}

\section{Kryteria oceny}

\begin{itemize}
  \item Prezentacja multimedialna
  \item Analiza przypadków
  \item Kryteria oceny
  \item Ćwiczenia/Laboratorium/Projekt/Lektorat
  \item Punkty gromadzone za pracę na ćwiczeniach i wykład (kolokwium)
  \item Ustalenie oceny na podstawie liczby punktów.
  \item Punkty gromadzone za pracę na ćwiczeniach 70\%
  \item Punkty z kolokwium z treści wykładu 30\%
  \item Przedmiot zalicza 50\%+1 punkt
  \item brak.
\end{itemize}

\section{Metody dydaktyczne}

Wykład, laboratoria, praca własna studenta.

\section{Literatura}

\textbf{Podstawowa:}
\begin{itemize}
  \item "Mikroekonomia" - David Begg, Rudiger Dornbusch, Stanley Fischer, PWE (2014)
\end{itemize}

\textbf{Uzupełniająca:}
\begin{itemize}
  \item 1. „Banki i rynki finansowe. Od zaufania publicznego do kasyna” - Zbyszek Grocholski
  \item 2. "Wolny wybór" – Milton Friedman
  \item 3. „Mapa i terytorium” - Alan Greenspan
  \item 4. “Kapitał XXI wieku” – Thomas Piketty
  \item 5. „Makro i mikroekonimia” – Wydawnictwo PWN
\end{itemize}

\end{document}
