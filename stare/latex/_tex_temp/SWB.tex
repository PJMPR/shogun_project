% ===========================================================
%  Sylabus: Systemy wbudowane (SWB)
% ===========================================================
\documentclass[12pt, a4paper]{article}

\usepackage[T1]{fontenc}
\usepackage[utf8]{inputenc}
\usepackage[polish]{babel}
\usepackage{lmodern}
\usepackage{microtype}
\usepackage[a4paper, top=2.5cm, bottom=2.5cm, left=2.5cm, right=2.5cm]{geometry}
\usepackage{xcolor}
\usepackage{graphicx}
\usepackage{booktabs}
\usepackage{tabularx}
\usepackage{longtable}
\usepackage{multirow}
\usepackage{array}
\usepackage{colortbl}
\usepackage{enumitem}
\usepackage{fancyhdr}
\usepackage{titlesec}
\usepackage{mdframed}
\usepackage[colorlinks=true, linkcolor=red!70!black, urlcolor=red!70!black]{hyperref}
\usepackage{eso-pic}
\usepackage{tikz}

\definecolor{pjatkRed}{RGB}{180,0,0}
\definecolor{pjatkGray}{RGB}{80,80,80}
\definecolor{pjatkLightGray}{RGB}{245,245,245}
\definecolor{tableHeader}{RGB}{220,220,220}

\pagestyle{fancy}\fancyhf{}
\renewcommand{\headrulewidth}{0.4pt}
\renewcommand{\footrulewidth}{0.4pt}
\fancyhead[L]{\small\textcolor{pjatkGray}{PJATK -- Filia w Gdańsku \textbar\ Informatyka}}
\fancyhead[R]{\small\textcolor{pjatkGray}{Sylabus: SWB}}
\fancyfoot[C]{\small\thepage}

\titleformat{\section}{\large\bfseries\color{pjatkRed}}{\thesection.}{0.5em}{}
  [\color{pjatkRed}\rule{\linewidth}{0.8pt}]
\setlist{noitemsep, topsep=3pt, parsep=2pt}

\newmdenv[linecolor=pjatkRed, linewidth=1.2pt, backgroundcolor=pjatkLightGray,
  innerleftmargin=10pt, innerrightmargin=10pt, innertopmargin=8pt,
  innerbottommargin=8pt, roundcorner=4pt]{infobox}

\begin{document}

\AddToShipoutPictureBG{%
  \begin{tikzpicture}[remember picture, overlay]
    \node[opacity=0.5] at (current page.center) {%
      \includegraphics[width=14cm]{C:/Users/adamu/WebstormProjects/pj-studies/latex/PJATK_pl_sygnet_transparent-eps-converted-to}%
    };
  \end{tikzpicture}%
}

\begin{center}
  \includegraphics[height=2cm]{C:/Users/adamu/WebstormProjects/pj-studies/latex/PJATK_pl_poziom_1}\\[0.8cm]
  {\LARGE\bfseries\color{pjatkRed} SYLABUS PRZEDMIOTU}\\[0.8cm]
\end{center}

\begin{infobox}
\begin{tabularx}{\textwidth}{@{}lX@{}}
  \textbf{Nazwa przedmiotu:}  & {\bfseries Systemy wbudowane} \\[3pt]
  \textbf{Kod przedmiotu:}    & SWB \\[3pt]
  \textbf{Kierunek / Profil:} & Informatyka / praktyczny \\[3pt]
  \textbf{Tryb studiów:}      & niestacjonarny \\[3pt]
  \textbf{Rok / Semestr:}     & 3 / 5 \\[3pt]
  \textbf{Charakter:}         & obowiązkowy \\[3pt]
  \textbf{Odpowiedzialny:}    & dr Paweł Syty \\[3pt]
  \textbf{Wersja z dnia:}     & 19.02.2026 \\
\end{tabularx}
\end{infobox}

\vspace{1cm}

\section{Godziny zajęć i punkty ECTS}

\begin{center}
\begin{tabular}{|>{\centering\arraybackslash}p{2.0cm}
                |>{\centering\arraybackslash}p{2.0cm}
                |>{\centering\arraybackslash}p{2.0cm}
                |>{\centering\arraybackslash}p{2.4cm}
                |>{\centering\arraybackslash}p{2.4cm}
                |>{\centering\arraybackslash}p{2.0cm}
                |>{\centering\arraybackslash}p{1.4cm}|}
\hline
\rowcolor{tableHeader}
\textbf{Wykłady} & \textbf{Ćwiczenia} & \textbf{Laboratorium} &
\textbf{Z prowadzącym} & \textbf{Praca własna} & \textbf{Łącznie} & \textbf{ECTS} \\
\hline
16 h & --- & 16 h & 32 h & 68 h & 100 h & \textbf{4} \\
\hline
\end{tabular}
\end{center}

\section{Forma zajęć}

\begin{tabular}{ll}
  \hline
  \textbf{Forma zajęć} & \textbf{Sposób zaliczenia} \\
  \hline
  Laboratorium & Zaliczenie z oceną \\
  Ćwiczenia & Zaliczenie z oceną \\
  Wykład & Nieoceniany \\
  \hline
\end{tabular}

\section{Cel dydaktyczny}

Celem przedmiotu jest zaznajomienie studentów z budową i podstawowymi metodami projektowania i programowania systemów wbudowanych (ang. embedded systems), głównie o opartych o mikrokontrolery, czyli specjalizowanych systemów informatycznych, odpowiadających za wykonywanie ściśle określonych zadań - zwykle związanych z monitorowaniem i sterowaniem. Omówione zostaną fizyczne podstawy funkcjonowania czujników (receptorów) i elementów wykonawczych (efektorów / aktuatorów), jako podstawowych składników tego typu systemów.

\section{Treści programowe}

\begin{enumerate}
  \item Zagadnienia ogólne
  \item • Definicja systemu wbudowanego, historia, zastosowania
  \item • Architektury mikroprocesorów i mikrokontrolerów. Rodziny mikrokontrolerów
  \item • Modelowanie systemów czasu rzeczywistego, system przerwań
  \item • Szyny komunikacyjne (SPI, I2C, 1-wire)
  \item • Systemy monitoringu domowego (Home Assistant, Domoticz itp.)
  \item • Wstęp do Internetu Rzeczy (IoT), protokoły komunikacyjne (ZigBee, Z-Wave, MQTT)
  \item • Przykładowe realizacje
  \item Omówienie wybranych platform sprzętowych
  \item • Arduino (Uno, Mega, Nano, Pro Mini itp.) – platforma oparta na 8-bitowych mikrokontrolerach z rodziny Atmel AVR
  \item • Raspberry Pi – rodzina 32/64-bitowych jednopłytkowych minikomputerów opartych na mikroprocesorach z rodziny ARM (11, Cortex)
  \item • ESP8266 / ESP32 – 32-bitowy mikrokontroler RISC z wbudowaną transmisją WiFi / Wifi+Bluetooth, przez co przydatny do połączenia systemu z siecią Internet
  \item • Raspberry Pi Pico – płytka mikrokontrolera, zbudowana na bazie chipu RP2040
  \item • STM32 – rodzina 32-bitowych mikrokontrolerów z rodziny ARM Cortex
  \item Omówienie wybranych języków programowania – w zakresie niezbędnym do programowania systemów wbudowanych
  \item • Pochodna języka C/C++ – programowanie Arduino / ESP8266 / ESP32
  \item • C/C++ – programowanie STM32
  \item • Python – skryptowy język ogólnego zastosowania, szczególnie wygodny w programowaniu dla Raspberry Pi, omówiony w kontekście wielowątkowości, sterowania zdalnymi urządzeniami i podstaw projektowania graficznego interfejsu użytkownika
  \item • MicroPython – wersja języka Python dla mikrokontrolerów – Raspberry Pi Pico
  \item Omówienie zasad funkcjonowania, podłączenia itp. osprzętu pomocniczego (5.5h)
  \item • Czujniki / sensory: termometry, barometry, czujniki odległości, deszczu, ruchu, koloru, dźwięku, urządzenia GPS, kamery itp.
  \item • Aktuatory / efektory / elementy wykonawcze: silniki, wyświetlacze LCD/LED/OLED itp.
  \item • Urządzenia do komunikacji: GSM, RFID/NFC itp.
  \item • Inne: zegary czasu rzeczywistego, konwertery napięć i poziomów logicznych itp.
  \item Omówienie wybranych środowisk programistycznych i narzędzi
  \item • Arduino IDE – środowisko programistyczne dla Arduino i ESP8266, wraz z niezbędnymi bibliotekami
  \item • TinkerCAD – serwis WWW do modelowania i symulowania urządzeń wbudowanych
  \item • Fritzing – projektowanie obwodów i układów
  \item • Arduino Cloud – środowisko programistyczne w chmurze
  \item • ThingSpeak – serwisy WWW, umożliwiający zbieranie i prezentowanie danych pomiarowych
\end{enumerate}

\section{Efekty kształcenia}

\subsection*{Wiedza}
\begin{itemize}
  \item Student zna architektury systemów komputerowe (w tym mikrokontrolerowych) i rozumie zasady projektowania systemów cyfrowych, opartych o mikrokontrolery.
  \item Student zna i rozumie podstawowe metody projektowania i implementacji systemów wbudowanych
  \item Student zna i rozumie zasady funkcjonowania wybranych składników systemu wbudowanego
\end{itemize}

\subsection*{Umiejętności}
\begin{itemize}
  \item Student potrafi zaprojektować proste układy cyfrowe sekwencyjne i kombinatoryczne oraz je oprogramować.
  \item Student potrafi  wybrać odpowiednie narzędzie do rozwiązania problemu zaistniałego w przedsiębiorstwie, a wymagającego użycia systemu wbudowanego
  \item Student potrafi zaprojektować, zbudować i oprogramować proste, specjalizowane systemy wbudowane oparte na wybranych mikrokontrolerach.
\end{itemize}

\section{Kryteria oceny}

\begin{itemize}
  \item rozwiązywanie zadań
  \item studia przypadków z bazy studiów przypadków opracowanych przez firmy MŚP lub będące efektem staży
  \item warsztaty
  \item Kryteria oceny
  \item Ocena praktycznego zastosowania kwestii omawianych na wykładach. Studenci otrzymują zadania praktyczne do wykonania podczas zajęć (do zdobycia maks. 25 punktów, z gradacją 0.5 p.) oraz zadanie indywidualne (do zdobycia maks. 25 punktów) – w sumie 50 punktów. Zajęcia zalicza otrzymanie co najmniej 25 punktów.
\end{itemize}

\section{Metody dydaktyczne}

Wykład, laboratoria, praca własna studenta.

\section{Literatura}

\textbf{Podstawowa:}
\begin{itemize}
  \item A. Peck, Raspberry Pi Zero W. Kontrolery, czujniki, sterowniki i gadżety. Helion, 2019W. Wrotek, Arduino od podstaw. Helion, 2023
\end{itemize}

\textbf{Uzupełniająca:}
\begin{itemize}
  \item Brak danych.
\end{itemize}

\end{document}
