% ===========================================================
%  Sylabus: Technologie .NET (DOT)
% ===========================================================
\documentclass[12pt, a4paper]{article}

\usepackage[T1]{fontenc}
\usepackage[utf8]{inputenc}
\usepackage[polish]{babel}
\usepackage{lmodern}
\usepackage{microtype}
\usepackage[a4paper, top=2.5cm, bottom=2.5cm, left=2.5cm, right=2.5cm]{geometry}
\usepackage{xcolor}
\usepackage{graphicx}
\usepackage{booktabs}
\usepackage{tabularx}
\usepackage{longtable}
\usepackage{multirow}
\usepackage{array}
\usepackage{colortbl}
\usepackage{enumitem}
\usepackage{fancyhdr}
\usepackage{titlesec}
\usepackage{mdframed}
\usepackage[colorlinks=true, linkcolor=red!70!black, urlcolor=red!70!black]{hyperref}
\usepackage{eso-pic}
\usepackage{tikz}

\definecolor{pjatkRed}{RGB}{180,0,0}
\definecolor{pjatkGray}{RGB}{80,80,80}
\definecolor{pjatkLightGray}{RGB}{245,245,245}
\definecolor{tableHeader}{RGB}{220,220,220}

\pagestyle{fancy}\fancyhf{}
\renewcommand{\headrulewidth}{0.4pt}
\renewcommand{\footrulewidth}{0.4pt}
\fancyhead[L]{\small\textcolor{pjatkGray}{PJATK -- Filia w Gdańsku \textbar\ Informatyka}}
\fancyhead[R]{\small\textcolor{pjatkGray}{Sylabus: DOT}}
\fancyfoot[C]{\small\thepage}

\titleformat{\section}{\large\bfseries\color{pjatkRed}}{\thesection.}{0.5em}{}
  [\color{pjatkRed}\rule{\linewidth}{0.8pt}]
\setlist{noitemsep, topsep=3pt, parsep=2pt}

\newmdenv[linecolor=pjatkRed, linewidth=1.2pt, backgroundcolor=pjatkLightGray,
  innerleftmargin=10pt, innerrightmargin=10pt, innertopmargin=8pt,
  innerbottommargin=8pt, roundcorner=4pt]{infobox}

\begin{document}

\AddToShipoutPictureBG{%
  \begin{tikzpicture}[remember picture, overlay]
    \node[opacity=0.5] at (current page.center) {%
      \includegraphics[width=14cm]{C:/Users/adamu/WebstormProjects/pj-studies/latex/PJATK_pl_sygnet_transparent-eps-converted-to}%
    };
  \end{tikzpicture}%
}

\begin{center}
  \includegraphics[height=2cm]{C:/Users/adamu/WebstormProjects/pj-studies/latex/PJATK_pl_poziom_1}\\[0.8cm]
  {\LARGE\bfseries\color{pjatkRed} SYLABUS PRZEDMIOTU}\\[0.8cm]
\end{center}

\begin{infobox}
\begin{tabularx}{\textwidth}{@{}lX@{}}
  \textbf{Nazwa przedmiotu:}  & {\bfseries Technologie .NET} \\[3pt]
  \textbf{Kod przedmiotu:}    & DOT \\[3pt]
  \textbf{Kierunek / Profil:} & Informatyka / praktyczny \\[3pt]
  \textbf{Tryb studiów:}      & stacjonarny \\[3pt]
  \textbf{Rok / Semestr:}     & 3 / 6 \\[3pt]
  \textbf{Charakter:}         & obieralny \\[3pt]
  \textbf{Odpowiedzialny:}    & dr Tomasz Borzyszkowski \\[3pt]
  \textbf{Wersja z dnia:}     & 19.02.2026 \\
\end{tabularx}
\end{infobox}

\vspace{1cm}

\section{Godziny zajęć i punkty ECTS}

\begin{center}
\begin{tabular}{|>{\centering\arraybackslash}p{2.0cm}
                |>{\centering\arraybackslash}p{2.0cm}
                |>{\centering\arraybackslash}p{2.0cm}
                |>{\centering\arraybackslash}p{2.4cm}
                |>{\centering\arraybackslash}p{2.4cm}
                |>{\centering\arraybackslash}p{2.0cm}
                |>{\centering\arraybackslash}p{1.4cm}|}
\hline
\rowcolor{tableHeader}
\textbf{Wykłady} & \textbf{Ćwiczenia} & \textbf{Laboratorium} &
\textbf{Z prowadzącym} & \textbf{Praca własna} & \textbf{Łącznie} & \textbf{ECTS} \\
\hline
30 h & 30 h & --- & 60 h & 40 h & 100 h & \textbf{4} \\
\hline
\end{tabular}
\end{center}

\section{Forma zajęć}

\begin{tabular}{ll}
  \hline
  \textbf{Forma zajęć} & \textbf{Sposób zaliczenia} \\
  \hline
  Laboratorium & Zaliczenie z oceną \\
  \hline
\end{tabular}

\section{Cel dydaktyczny}

Przedmiot ma na celu zapoznanie studentów z podstawami języka C\# oraz z wybranymi bibliotekami związanymi z technologiami .Net (np. ASP.NET).

\section{Przedmioty wprowadzające}

\begin{tabularx}{\textwidth}{lX}
  \hline
  \textbf{Przedmiot} & \textbf{Wymagane zagadnienia} \\
  \hline
  Programowanie obiektowe w Javie & Znajomość zasad programowania obiektowego oraz podstaw języków C++ oraz Java \\
  \hline
\end{tabularx}

\section{Treści programowe}

\begin{enumerate}
  \item Wprowadzenie do Visual Studio i języka C\#
  \item Postawy programowania obiektowego w C\#: definiowanie klas i obiektów, ukrywanie informacji w klasach języka C\#, modyfikatory dostępu, klasy częściowe, porównanie właściwości struktur i klas, konstruktory i destruktory obiektów, inicjalizowanie struktur, dziedziczenie i przesłanianie metod
  \item Konstrukcje programistyczne: definicja właściwości i indeksatorów, składowe statyczne, przeciążanie operatorów
  \item Techniki zaawansowane: atrybuty i mechanizm refleksji, delegacje i obsługa zdarzeń, serializacja
  \item Wprowadzenie do tworzenia serwisów webowych w technologii ASP.
  \item Informacje dodatkowe
  \item Wymagana instalacja współczesnej wersji kompilatora .Net wraz z odpowiednim wsparciem edycji kodu i jego uruchamiania.
  \item Uzasadnienie dla prowadzenia przedmiotu/współpraca z rynkiem pracy
  \item Zrozumienie zagadnień realizowanych w ramach tego przedmiotu jest niezbędne do wykonywania zawodów: programisty oraz architekta oprogramowania w firmach wykorzystujących bibliotekę .Net.
\end{enumerate}

\section{Efekty kształcenia}

\subsection*{Wiedza}
\begin{itemize}
  \item Student zna i rozumie konstrukcje programistyczne w języku C\#.
  \item Student zna i rozumie:
  \item paradygmaty programowania obiektowego w języku C\#
  \item wykorzystanie właściwości i indeksatorów
  \item wykorzystanie i tworzenie atrybutów oraz korzystanie z mechanizmu refleksji
  \item wykorzystanie delegacji i obsługi zdarzeń
  \item wykorzystanie serializacji
  \item Student zna i rozumie poznane pojęcia w stopniu zaawansowanym.
\end{itemize}

\subsection*{Umiejętności}
\begin{itemize}
  \item Student potrafi czytać ze zrozumieniem programy w języku C\#. Potrafi je pisać, weryfikować i uruchamiać.
  \item Student potrafi ocenić przydatność paradygmatów programistycznych języka C\# do rozwiązania zadań laboratoryjnych oraz dobrać odpowiednie środowisko programistyczne.
  \item Student potrafi  zaprojektować i wytworzyć oprogramowanie, zgodnie z  paradygmatami programistycznymi języka C\# oraz przetestować i zdebuggować program. W zależności od specyfiki zadania, potrafi zaplanować etapy wytwarzania oprogramowania oraz dobrać narzędzia programistyczne wspomagające ten proces.
  \item Student potrafi zastosować poznane pojęcia celem stworzenia działającego rozwiązania napisanego w jeżyku C\#.
\end{itemize}

\section{Kryteria oceny}

\begin{itemize}
  \item z prezentacją multimedialna
  \item z prezentacją oprogramowania
  \item rozwiązywanie zadań
  \item brak
  \item Kryteria oceny
  \item ocena praktycznego zastosowania kwestii omawianych na wykładach. W ramach każdej jednostki laboratoryjnej studenci otrzymują do wykonania zadania programistyczne. Przedmiot zalicza otrzymanie minimum 50\% punktów.
  \item Skala ocen:
  \item Poniżej 50\% - ndst
  \item Od 50\% - dst
  \item Od 60\% - dst+
  \item Od 70\% - db
  \item Od 80\% - db+
  \item Od 90\% - bdb
  \item brak
\end{itemize}

\section{Metody dydaktyczne}

Wykład, laboratoria, praca własna studenta.

\section{Literatura}

\textbf{Podstawowa:}
\begin{itemize}
  \item Ian Griffiths, C\# 10. Programowanie. Tworzenie aplikacji Windows, internetowych i biurowych. Helion, 2023.
  \item Andrew Stellman, Jennifer Greene, C\#. Rusz głową! Wydanie IV. Helion, 2022.
\end{itemize}

\textbf{Uzupełniająca:}
\begin{itemize}
  \item Adam Freeman, ASP.NET Core 3. Zaawansowane programowanie. Wydanie VIII. Helion, 2021.
\end{itemize}

\end{document}
