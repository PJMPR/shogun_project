% ===========================================================
%  Sylabus: Warsztat Programisty (WPR)
% ===========================================================
\documentclass[12pt, a4paper]{article}

\usepackage[T1]{fontenc}
\usepackage[utf8]{inputenc}
\usepackage[polish]{babel}
\usepackage{lmodern}
\usepackage{microtype}
\usepackage[a4paper, top=2.5cm, bottom=2.5cm, left=2.5cm, right=2.5cm]{geometry}
\usepackage{xcolor}
\usepackage{graphicx}
\usepackage{booktabs}
\usepackage{tabularx}
\usepackage{longtable}
\usepackage{multirow}
\usepackage{array}
\usepackage{colortbl}
\usepackage{enumitem}
\usepackage{fancyhdr}
\usepackage{titlesec}
\usepackage{mdframed}
\usepackage[colorlinks=true, linkcolor=red!70!black, urlcolor=red!70!black]{hyperref}
\usepackage{eso-pic}
\usepackage{tikz}

\definecolor{pjatkRed}{RGB}{180,0,0}
\definecolor{pjatkGray}{RGB}{80,80,80}
\definecolor{pjatkLightGray}{RGB}{245,245,245}
\definecolor{tableHeader}{RGB}{220,220,220}

\pagestyle{fancy}\fancyhf{}
\renewcommand{\headrulewidth}{0.4pt}
\renewcommand{\footrulewidth}{0.4pt}
\fancyhead[L]{\small\textcolor{pjatkGray}{PJATK -- Filia w Gdańsku \textbar\ Informatyka}}
\fancyhead[R]{\small\textcolor{pjatkGray}{Sylabus: WPR}}
\fancyfoot[C]{\small\thepage}

\titleformat{\section}{\large\bfseries\color{pjatkRed}}{\thesection.}{0.5em}{}
  [\color{pjatkRed}\rule{\linewidth}{0.8pt}]
\setlist{noitemsep, topsep=3pt, parsep=2pt}

\newmdenv[linecolor=pjatkRed, linewidth=1.2pt, backgroundcolor=pjatkLightGray,
  innerleftmargin=10pt, innerrightmargin=10pt, innertopmargin=8pt,
  innerbottommargin=8pt, roundcorner=4pt]{infobox}

\begin{document}

\AddToShipoutPictureBG{%
  \begin{tikzpicture}[remember picture, overlay]
    \node[opacity=0.5] at (current page.center) {%
      \includegraphics[width=14cm]{C:/Users/adamu/WebstormProjects/pj-studies/latex/PJATK_pl_sygnet_transparent-eps-converted-to}%
    };
  \end{tikzpicture}%
}

\begin{center}
  \includegraphics[height=2cm]{C:/Users/adamu/WebstormProjects/pj-studies/latex/PJATK_pl_poziom_1}\\[0.8cm]
  {\LARGE\bfseries\color{pjatkRed} SYLABUS PRZEDMIOTU}\\[0.8cm]
\end{center}

\begin{infobox}
\begin{tabularx}{\textwidth}{@{}lX@{}}
  \textbf{Nazwa przedmiotu:}  & {\bfseries Warsztat Programisty} \\[3pt]
  \textbf{Kod przedmiotu:}    & WPR \\[3pt]
  \textbf{Kierunek / Profil:} & Informatyka / praktyczny \\[3pt]
  \textbf{Tryb studiów:}      & stacjonarny \\[3pt]
  \textbf{Rok / Semestr:}     & 1 / 1 \\[3pt]
  \textbf{Charakter:}         & obowiązkowy \\[3pt]
  \textbf{Odpowiedzialny:}    & mgr inż. Adam Urbanowicz \\[3pt]
  \textbf{Wersja z dnia:}     & 19.02.2026 \\
\end{tabularx}
\end{infobox}

\vspace{1cm}

\section{Godziny zajęć i punkty ECTS}

\begin{center}
\begin{tabular}{|>{\centering\arraybackslash}p{2.0cm}
                |>{\centering\arraybackslash}p{2.0cm}
                |>{\centering\arraybackslash}p{2.0cm}
                |>{\centering\arraybackslash}p{2.4cm}
                |>{\centering\arraybackslash}p{2.4cm}
                |>{\centering\arraybackslash}p{2.0cm}
                |>{\centering\arraybackslash}p{1.4cm}|}
\hline
\rowcolor{tableHeader}
\textbf{Wykłady} & \textbf{Ćwiczenia} & \textbf{Laboratorium} &
\textbf{Z prowadzącym} & \textbf{Praca własna} & \textbf{Łącznie} & \textbf{ECTS} \\
\hline
16 h & --- & 16 h & 32 h & 68 h & 100 h & \textbf{4} \\
\hline
\end{tabular}
\end{center}

\section{Cel dydaktyczny}

Celem przedmiotu jest wykształcenie u studentów praktycznych umiejętności pracy warsztatowej programisty, w szczególności w zakresie korzystania z narzędzi programistycznych, debugowania kodu, pracy zespołowej z wykorzystaniem systemu kontroli wersji Git oraz platformy GitHub. Zajęcia mają na celu zapoznanie studentów z podstawami automatyzacji procesów (CI/CD), debugowania aplikacji internetowych z użyciem narzędzi deweloperskich przeglądarek, podstawami technologii webowych (HTML, CSS, JavaScript), a także wprowadzenie do świadomego i odpowiedzialnego wykorzystania narzędzi opartych na sztucznej inteligencji w pracy programisty. Dodatkowym celem jest przygotowanie studentów do budowania własnego portfolio programistycznego.

\section{Treści programowe}

\begin{enumerate}
  \item Organizacja pracy programisty: repozytorium kursowe, wymagane konta (GitHub), struktura materiałów, zasady oddawania prac i checklisty jakości.
  \item Przegląd popularnych IDE i narzędzi: Visual Studio / VS Code / IntelliJ / Rider (porównanie), konfiguracja środowiska, rozszerzenia, formatowanie i linting.
  \item Debugowanie kodu: podstawy debuggera (breakpointy, step in/out/over, watch, call stack), praca na przykładach w aplikacji konsolowej.
  \item Debugowanie błędów wykonania: wyjątki, logowanie, asercje, minimalny repro-case, czytanie stack trace, kultura zgłaszania błędów (issue).
  \item Git: model pracy i podstawowe komendy (init/clone/status/add/commit/log/diff), ignorowanie plików (.gitignore), dobre komunikaty commitów.
  \item Git: praca z repozytorium zdalnym (remote, fetch, pull, push), rozwiązywanie typowych problemów, praca w parach i code review (wstęp).
  \item Branche w Git: tworzenie i przełączanie gałęzi, merge vs rebase (intuicja), konflikty i ich rozwiązywanie na prostych przykładach.
  \item Pull Requesty na GitHub: workflow PR, szablon opisu, review, komentarze, poprawki, podstawy pracy z Issues i Projects (tablica zadań).
  \item Wprowadzenie do prostych pipeline’ów: czym jest CI/CD, podstawowy workflow (np. build/test/lint), plik konfiguracyjny, uruchomienia automatyczne po push/PR.
  \item  Publikacja statycznej strony na GitHub Pages: struktura repo, branch/folder publikacji, podstawy konfiguracji, dodanie prostej strony kursowej.
  \item  DevTools w przeglądarce: inspekcja DOM/CSS, podgląd i modyfikacja stylów na żywo, box model, debugowanie responsywności.
  \item  DevTools zaawansowane: zakładka Network (żądania/odpowiedzi), Console (logowanie, błędy), Sources (breakpointy w JS), podstawy wydajności (Performance/Lighthouse).
  \item  Wstęp do HTML: struktura dokumentu, semantyka, formularze (podstawy), dostępność (a11y) i dobre praktyki.
  \item  Wstęp do CSS: selektory, kaskada i specyficzność, flexbox/grid (wprowadzenie), responsywność, organizacja stylów w projekcie.
  \item  Wstęp do JavaScript: typy i zmienne, funkcje, zdarzenia w DOM, proste operacje na elementach strony; wstęp do pracy z agentem AI w roli asystenta programisty (promptowanie, weryfikacja odpowiedzi, zasady bezpieczeństwa i etyki).
\end{enumerate}

\section{Efekty kształcenia}

\subsection*{Wiedza}
\begin{itemize}
  \item Student potrafi  czytać ze zrozumieniem kod języków skryptowych,  tworzyć skrypty  oraz zweryfikować poprawność utworzonego kodu.
  \item Student potrafi projektować aplikacje  zgodnie z paradygmatem strukturalnym i obiektowym. Student posiada podstawową wiedzę na temat protokołu HTTP oraz tworzenia bezpiecznych aplikacji internetowych i bazodanowych.
  \item Student zna odpowiednie narzędzia wspomagające tworzenie skryptów oraz wspomagające projektowanie aplikacji internetowych, bazodanowych
\end{itemize}

\subsection*{Umiejętności}
\begin{itemize}
  \item Student potrafi wybrać odpowiednie środowisko programistyczne i inne narzędzia, które wspomagają w projektowaniu aplikacji oraz potrafi dobrać model procesu wytwarzania aplikacji  do specyfiki przedsięwzięcia. Student potrafi posługiwać się wybranym środowiskiem oraz analizować poprawność działania stworzonej aplikacji.
  \item Student potrafi wykorzystać odpowiednie biblioteki do tworzenia aplikacji oraz wykorzystać narzędzia, które umożliwiają tworzenie,  debbugowanie i uruchomienie projektów programistycznych. Student potrafi przedstawić stworzoną aplikację oraz omówić jej sposób działania.
  \item Student potrafi skonfigurować środowisko potrzebne do stworzenia skryptów i aplikacji internetowej i bazodanowej, tzn. posiada wiedzę na temat konfiguracji bazy danych oraz oprogramowania, które jest potrzebne do tworzenie projektu programistycznego.
\end{itemize}

\section{Kryteria oceny}

\begin{itemize}
  \item Ćwiczenia / Laboratorium/Lektorat:
  \item rozwiązywanie zadań
  \item warsztaty
  \item Ćwiczenia/Laboratorium/Projekt/Lektorat/Seminarium
  \item Kryteria oceny
  \item Ćwiczenia/Laboratorium/Projekt/Lektorat
  \item Procentowy udział o ocenie ostatecznej:
  \item laboratorium- 20\%
  \item projektu- 50\%
  \item kolokwium- 30\%
  \item Uzyskanie pozytywnej oceny od 50\% zdobytych punktów, które są pomnożone przez odpowiednie wagi procentowe, przy czym z kolokwium,  należy uzyskać co najmniej 40\% punktów.
\end{itemize}

\section{Metody dydaktyczne}

Wykład, laboratoria, praca własna studenta.

\section{Literatura}

\textbf{Podstawowa:}
\begin{itemize}
  \item Dokumentacja języka: www.php.net
  \item w3schools.com
  \item Larry Ullman ,PHP i MySQL. Dynamiczne strony WWW. Szybki start, Helion 2018
\end{itemize}

\textbf{Uzupełniająca:}
\begin{itemize}
  \item Brak danych.
\end{itemize}

\end{document}
