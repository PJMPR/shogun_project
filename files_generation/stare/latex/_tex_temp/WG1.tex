% ===========================================================
%  Sylabus: Wytwarzanie gier 1 (WG1)
% ===========================================================
\documentclass[12pt, a4paper]{article}

\usepackage[T1]{fontenc}
\usepackage[utf8]{inputenc}
\usepackage[polish]{babel}
\usepackage{lmodern}
\usepackage{microtype}
\usepackage[a4paper, top=2.5cm, bottom=2.5cm, left=2.5cm, right=2.5cm]{geometry}
\usepackage{xcolor}
\usepackage{graphicx}
\usepackage{booktabs}
\usepackage{tabularx}
\usepackage{longtable}
\usepackage{multirow}
\usepackage{array}
\usepackage{colortbl}
\usepackage{enumitem}
\usepackage{fancyhdr}
\usepackage{titlesec}
\usepackage{mdframed}
\usepackage[colorlinks=true, linkcolor=red!70!black, urlcolor=red!70!black]{hyperref}
\usepackage{eso-pic}
\usepackage{tikz}

\definecolor{pjatkRed}{RGB}{180,0,0}
\definecolor{pjatkGray}{RGB}{80,80,80}
\definecolor{pjatkLightGray}{RGB}{245,245,245}
\definecolor{tableHeader}{RGB}{220,220,220}

\pagestyle{fancy}\fancyhf{}
\renewcommand{\headrulewidth}{0.4pt}
\renewcommand{\footrulewidth}{0.4pt}
\fancyhead[L]{\small\textcolor{pjatkGray}{PJATK -- Filia w Gdańsku \textbar\ Informatyka}}
\fancyhead[R]{\small\textcolor{pjatkGray}{Sylabus: WG1}}
\fancyfoot[C]{\small\thepage}

\titleformat{\section}{\large\bfseries\color{pjatkRed}}{\thesection.}{0.5em}{}
  [\color{pjatkRed}\rule{\linewidth}{0.8pt}]
\setlist{noitemsep, topsep=3pt, parsep=2pt}

\newmdenv[linecolor=pjatkRed, linewidth=1.2pt, backgroundcolor=pjatkLightGray,
  innerleftmargin=10pt, innerrightmargin=10pt, innertopmargin=8pt,
  innerbottommargin=8pt, roundcorner=4pt]{infobox}

\begin{document}

\AddToShipoutPictureBG{%
  \begin{tikzpicture}[remember picture, overlay]
    \node[opacity=0.5] at (current page.center) {%
      \includegraphics[width=14cm]{C:/Users/adamu/WebstormProjects/pj-studies/latex/PJATK_pl_sygnet_transparent-eps-converted-to}%
    };
  \end{tikzpicture}%
}

\begin{center}
  \includegraphics[height=2cm]{C:/Users/adamu/WebstormProjects/pj-studies/latex/PJATK_pl_poziom_1}\\[0.8cm]
  {\LARGE\bfseries\color{pjatkRed} SYLABUS PRZEDMIOTU}\\[0.8cm]
\end{center}

\begin{infobox}
\begin{tabularx}{\textwidth}{@{}lX@{}}
  \textbf{Nazwa przedmiotu:}  & {\bfseries Wytwarzanie gier 1} \\[3pt]
  \textbf{Kod przedmiotu:}    & WG1 \\[3pt]
  \textbf{Kierunek / Profil:} & Informatyka / praktyczny \\[3pt]
  \textbf{Tryb studiów:}      & stacjonarny \\[3pt]
  \textbf{Rok / Semestr:}     & 3 / 5 \\[3pt]
  \textbf{Charakter:}         & obowiązkowy \\[3pt]
  \textbf{Odpowiedzialny:}    & Lic. Aleksandr Polin (alex.polin@pjwstk.edu.pl) \\[3pt]
  \textbf{Wersja z dnia:}     & 19.02.2026 \\
\end{tabularx}
\end{infobox}

\vspace{1cm}

\section{Godziny zajęć i punkty ECTS}

\begin{center}
\begin{tabular}{|>{\centering\arraybackslash}p{2.0cm}
                |>{\centering\arraybackslash}p{2.0cm}
                |>{\centering\arraybackslash}p{2.0cm}
                |>{\centering\arraybackslash}p{2.4cm}
                |>{\centering\arraybackslash}p{2.4cm}
                |>{\centering\arraybackslash}p{2.0cm}
                |>{\centering\arraybackslash}p{1.4cm}|}
\hline
\rowcolor{tableHeader}
\textbf{Wykłady} & \textbf{Ćwiczenia} & \textbf{Laboratorium} &
\textbf{Z prowadzącym} & \textbf{Praca własna} & \textbf{Łącznie} & \textbf{ECTS} \\
\hline
30 h & 30 h & --- & 60 h & 65 h & 125 h & \textbf{} \\
\hline
\end{tabular}
\end{center}

\section{Forma zajęć}

\begin{tabular}{ll}
  \hline
  \textbf{Forma zajęć} & \textbf{Sposób zaliczenia} \\
  \hline
  Laboratorium & Zaliczenie z oceną \\
  Wykład & Egzamin \\
  \hline
\end{tabular}

\section{Cel dydaktyczny}

Celem tego kursu jest zanurzenie studentów w cyklu tworzenia gier, od projektu koncepcyjnego po kontrolę jakości, zaszczepienie dogłębnego zrozumienia praktyk branżowych, narzędzi (takich jak Unreal Engine) i kluczowych procesów niezbędnych do udanej kariery w branży gier.

\section{Treści programowe}

\begin{enumerate}
  \item Wprowadzenie do branży tworzenia gier. Prezentowanie historii gier z przykładami unikalnych i nieoczekiwanych ciekawych faktów na temat gier.
  \item Dyskusja na temat branży gier, osobistych doświadczeń uczniów z grami, wpływu gier na życie i tego, czego można się spodziewać pracując w tej branży.
  \item Wprowadzenie do projektowania gier. Dlaczego istnieje i jak to rozumieć.
  \item Projektowanie małej gry przy użyciu wyłącznie pióra i papieru. Granie i ulepszanie swoich gier.
  \item Projektowanie puzzli, jako głównego interaktywnego elementu zabawy. Jak łamigłówki angażują graczy.
  \item Projektowanie układanki przy użyciu dostarczonych ograniczeń i testowanie pojawiających się mechanik gry w małych grupach.
  \item Wprowadzenie do Unreal Engine. Przegląd układu i funkcji UE 5.
  \item Wykonanie pierwszego planu UE mechaniki gry z wykorzystaniem elementów interaktywnych.
  \item Podstawy silnika Unreal Engine. Ciąg dalszy przedstawiania silnika studentom. Wyjaśnienie rozwoju opartego na planach za pomocą interaktywnych przykładów.
  \item Studenci doskonalą i rozwijają pierwszy projekt stworzony podczas poprzedniego ćwiczenia, wykorzystując nową wiedzę zdobytą na wykładzie.
  \item Podstawowe zasady projektowania poziomu podstawowego i Asset Pipeline w UE5.
  \item Uczniowie korzystają z dostarczonych zasobów i niestandardowych materiałów, aby konfigurować i wykorzystywać własne projekty oraz projektować układ poziomów.
  \item Dokumentacja gry i dlaczego jest to ważne.
  \item Napisanie krótkiego opisu gry na podanym przykładzie.
  \item GDD. Przegląd i ocena GDD znanych gier.
  \item Pisanie GDD. Uczniowie korzystają z dostarczonego przykładu GDD i pracują w grupach, aby stworzyć własne dokumenty.
  \item Praca w zespole i zarządzanie projektem gry.
  \item Omówienie ról zespołowych i zarządzania sytuacją, kontynuacja pracy w zespołach, nad grami.
  \item Wprowadzenie do animacji w UE5. Rzuć okiem na możliwe potoki animacji i narzędzia dostępne w Unreal Engine.
  \item Implementacja podstawowych animacji w UE z wykorzystaniem dostarczonych materiałów i przykładów.
  \item Zrozumienie wpływu projektowania dźwięku w grach i sposobu implementacji dźwięków w UE5.
  \item Implementacja podstawowych dźwięków w UE z wykorzystaniem dostarczonych materiałów i przykładów.
  \item Zapewnienie jakości gier. Jak zespoły zapewniają jakość i dbałość o szczegóły.
  \item Testowanie dostarczonej przykładowej gry i przygotowanie raportu o błędzie.
  \item Praktyki HR w tworzeniu gier. Oczekiwania firm zajmujących się tworzeniem gier i czego szukają u potencjalnych kandydatów.
  \item Napisanie osobistego CV na podanym przykładzie.
  \item Zarządzanie zasobami studia gier. Ukryte koszty i dlaczego tak trudno osiągnąć równowagę pomiędzy zyskami a zabawą.
  \item Praca w grupach w celu przygotowania analizy rynkowej dostarczonej gry, przedstawienia planu ulepszeń z prośbą o dodatkowe inwestycje, analizy konkurencji i oceny innych grup uczniów.
  \item Pełne podsumowanie kursu. Powracanie do poprzednich lekcji, odświeżanie wiedzy i podkreślanie najważniejszych aspektów.
  \item Omawianie projektów gier zespołowych, ulepszanie i dopracowywanie projektów.
\end{enumerate}

\section{Efekty kształcenia}

\subsection*{Wiedza}
\begin{itemize}
  \item Jest w stanie zastosować teorie i zasady projektowania graficznego oraz interfejsów użytkownika do tworzenia intuicyjnych i angażujących elementów gry.
  \item Potrafi przeprowadzić kompleksowe testy gry, identyfikując i dokumentując błędy, co przyczynia się do wydania produktu o wysokiej jakości.
\end{itemize}

\subsection*{Umiejętności}
\begin{itemize}
  \item Wie, jak koncepcyjnie zaprojektować grę, zastosować narzędzia projektowe jak Unreal Engine i przeprowadzić analizę rynkową.
  \item Posiada kompetencje do tworzenia kompleksowej dokumentacji projektowej, która komunikuje kluczowe aspekty rozwoju gry międzynarodowej publiczności i zespołowi projektowemu.
  \item Demonstruje zdolność do ciągłego doskonalenia swoich umiejętności deweloperskich, korzystając z różnorodnych źródeł informacji i nowoczesnych metod edukacyjnych.
  \item Jest wyposażony w umiejętności niezbędne do przeprowadzania skutecznych testów gier, identyfikowania błędów i zapewniania jakości produktu końcowego.
  \item Demonstruje umiejętność tworzenia elementów gry przy użyciu Unreal Engine, w tym blueprintów i animacji.
\end{itemize}

\subsection*{Kompetencje społeczne}
\begin{itemize}
  \item Wykazuje gotowość do ciągłego rozwijania kompetencji i samokształcenia, co jest kluczowe w szybko zmieniającej się branży gier komputerowych.
  \item Jest gotów do aktywnego uczestnictwa w procesie produkcyjnym gier, pełniąc różnorodne role w zespole deweloperskim i adaptując się do dynamiki projektu gamedev.
  \item Jest przygotowany do efektywnego zarządzania czasem i zasobami, określając priorytety w celu skutecznej realizacji zadań w procesie tworzenia gier.
  \item Rozumie i angażuje się w analizę oraz rozwiązywanie kwestii etycznych i prawnych związanych z projektowaniem i tworzeniem gier.
  \item Demonstruje umiejętności komunikacyjne potrzebne do efektywnego dialogu z różnorodnymi interesariuszami projektu gamedev, w tym inwestorami, w celu tworzenia wartości dodanej dla produktu.
\end{itemize}

\section{Kryteria oceny}

\begin{itemize}
  \item wykład z elementami dyskusji z prezentacją multimedialną, wykład zaproszony
  \item burza mózgów
  \item rozwiązywanie zadań
  \item analiza przypadków
  \item prezentacje
  \item praca w Unreal Engine
  \item Kryteria oceny
  \item Studenci prezentują swoje projekty (gry) i są oceniani przez publiczność punktowo:
  \item Styl (1-10)
  \item Rozgrywka (1-10)
  \item Zabawa (1-10)
  \item Kreatywność (1-10)
  \item Ukończenie (1-10)
  \item Studenci opowiadają także o swoim wkładzie w projekt i przyznają sobie punkty za:
  \item Wysiłek (1-10).
  \item Następnie obliczana jest ocena grupowa. Każda osoba z grupy określa, czy jej ocena końcowa powinna być taka sama jak ocena grupowa, czy też powinna otrzymać ocenę wyższą lub niższą. Ocena ta staje się oceną końcową dla każdego ucznia.
\end{itemize}

\section{Metody dydaktyczne}

Wykład, laboratoria, praca własna studenta.

\section{Literatura}

\textbf{Podstawowa:}
\begin{itemize}
  \item "Rules of Play: Game Design Fundamentals" by Katie Salen and Eric Zimmerman (2003)
  \item "The Art of Game Design: A Book of Lenses" by Jesse Schell (2008)
  \item "Level Up! The Guide to Great Video Game Design" by Scott Rogers (2010)
  \item "Unreal Engine 4 for Design Visualization: Developing Stunning Interactive Visualizations, Animations, and Renderings" by Tom Shannon (2017)
  \item "Game Engine Architecture" by Jason Gregory (2018)
  \item "Introduction to Game Development" by Steve Rabin (2009)
  \item "A Theory of Fun for Game Design" by Raph Koster (2012)
  \item Unreal Engine Educator Resources
\end{itemize}

\textbf{Uzupełniająca:}
\begin{itemize}
  \item "David Perry on Game Design: A Brainstorming ToolBox" by David Perry and Rusel DeMaria (2009)
  \item "Challenges for Game Designers" by Brenda Romero and Ian Schreiber (2008)
  \item "Character Development and Storytelling for Games" by Lee Sheldon (2004)
  \item "Game Design: Theory and Practice" by Richard Rouse III (2001)
  \item "Game Feel: A Game Designer's Guide to Virtual Sensation" by Steve Swink (2008)
  \item "Game Mechanics: Advanced Game Design" by Ernest W. Adams and Joris Dormans (2012)
  \item Freese, Maria \& Lukosch, Heide. (2023). The Funnel of Game Design – An Adaptive Game Design Approach for Complex Systems. Simulation \& Gaming. 55. 10.1177/10468781231222524.
  \item Hammar, Emil \& Canossa, Alessandro \& Juul, Jesper \& Debus, Michael \& Pfau, Johannes \& El-Nasr, Magy \& Azadvar, Ahmad. (2023). From Teams to Games: Connecting Game Development to Game Characteristics. 336-340. 10.1007/978-3-031-48050-8\_23.
  \item Zehnder, S. M., \& Lipscomb, S. D. (2006). The Role of Music in Video Games. In P. Vorderer \& J. Bryant (Eds.), Playing video games: Motives, responses, and consequences (pp. 241–258). Lawrence Erlbaum Associates Publishers.
  \item Ermi, Laura \& Mäyrä, Frans. (2005). Player-Centred Game Design: Experiences in Using Scenario Study to Inform Mobile Game Design.. Game Studies. 5.
\end{itemize}

\end{document}
