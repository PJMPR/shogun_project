% ===========================================================
%  Sylabus: Analiza matematyczna (AM)
% ===========================================================
\documentclass[12pt, a4paper]{article}

\usepackage[T1]{fontenc}
\usepackage[utf8]{inputenc}
\usepackage[polish]{babel}
\usepackage{lmodern}
\usepackage{microtype}
\usepackage[a4paper, top=2.5cm, bottom=2.5cm, left=2.5cm, right=2.5cm]{geometry}
\usepackage{xcolor}
\usepackage{graphicx}
\usepackage{booktabs}
\usepackage{tabularx}
\usepackage{longtable}
\usepackage{multirow}
\usepackage{array}
\usepackage{colortbl}
\usepackage{enumitem}
\usepackage{fancyhdr}
\usepackage{titlesec}
\usepackage{mdframed}
\usepackage[colorlinks=true, linkcolor=red!70!black, urlcolor=red!70!black]{hyperref}
\usepackage{eso-pic}
\usepackage{tikz}

\definecolor{pjatkRed}{RGB}{180,0,0}
\definecolor{pjatkGray}{RGB}{80,80,80}
\definecolor{pjatkLightGray}{RGB}{245,245,245}
\definecolor{tableHeader}{RGB}{220,220,220}

\pagestyle{fancy}\fancyhf{}
\renewcommand{\headrulewidth}{0.4pt}
\renewcommand{\footrulewidth}{0.4pt}
\fancyhead[L]{\small\textcolor{pjatkGray}{PJATK -- Filia w Gdańsku \textbar\ Informatyka}}
\fancyhead[R]{\small\textcolor{pjatkGray}{Sylabus: AM}}
\fancyfoot[C]{\small\thepage}

\titleformat{\section}{\large\bfseries\color{pjatkRed}}{\thesection.}{0.5em}{}
  [\color{pjatkRed}\rule{\linewidth}{0.8pt}]
\setlist{noitemsep, topsep=3pt, parsep=2pt}

\newmdenv[linecolor=pjatkRed, linewidth=1.2pt, backgroundcolor=pjatkLightGray,
  innerleftmargin=10pt, innerrightmargin=10pt, innertopmargin=8pt,
  innerbottommargin=8pt, roundcorner=4pt]{infobox}

\begin{document}

\AddToShipoutPictureBG{%
  \begin{tikzpicture}[remember picture, overlay]
    \node[opacity=0.5] at (current page.center) {%
      \includegraphics[width=14cm]{C:/Users/adamu/WebstormProjects/pj-studies/latex/PJATK_pl_sygnet_transparent-eps-converted-to}%
    };
  \end{tikzpicture}%
}

\begin{center}
  \includegraphics[height=2cm]{C:/Users/adamu/WebstormProjects/pj-studies/latex/PJATK_pl_poziom_1}\\[0.8cm]
  {\LARGE\bfseries\color{pjatkRed} SYLABUS PRZEDMIOTU}\\[0.8cm]
\end{center}

\begin{infobox}
\begin{tabularx}{\textwidth}{@{}lX@{}}
  \textbf{Nazwa przedmiotu:}  & {\bfseries Analiza matematyczna} \\[3pt]
  \textbf{Kod przedmiotu:}    & AM \\[3pt]
  \textbf{Kierunek / Profil:} & Informatyka / praktyczny \\[3pt]
  \textbf{Tryb studiów:}      & niestacjonarny \\[3pt]
  \textbf{Rok / Semestr:}     & 1 / 1 \\[3pt]
  \textbf{Charakter:}         & obowiązkowy \\[3pt]
  \textbf{Odpowiedzialny:}    & dr Elżbieta Puźniakowska-Gałuch, ela@pjwstk.edu.pl \\[3pt]
  \textbf{Wersja z dnia:}     & 19.02.2026 \\
\end{tabularx}
\end{infobox}

\vspace{1cm}

\section{Godziny zajęć i punkty ECTS}

\begin{center}
\begin{tabular}{|>{\centering\arraybackslash}p{2.0cm}
                |>{\centering\arraybackslash}p{2.0cm}
                |>{\centering\arraybackslash}p{2.0cm}
                |>{\centering\arraybackslash}p{2.4cm}
                |>{\centering\arraybackslash}p{2.4cm}
                |>{\centering\arraybackslash}p{2.0cm}
                |>{\centering\arraybackslash}p{1.4cm}|}
\hline
\rowcolor{tableHeader}
\textbf{Wykłady} & \textbf{Ćwiczenia} & \textbf{Laboratorium} &
\textbf{Z prowadzącym} & \textbf{Praca własna} & \textbf{Łącznie} & \textbf{ECTS} \\
\hline
16 h & 16 h & --- & 32 h & 93 h & 125 h & \textbf{5} \\
\hline
\end{tabular}
\end{center}

\section{Forma zajęć}

\begin{tabular}{ll}
  \hline
  \textbf{Forma zajęć} & \textbf{Sposób zaliczenia} \\
  \hline
  Ćwiczenia & Zaliczenie z oceną \\
  Wykład & Egzamin \\
  \hline
\end{tabular}

\section{Cel dydaktyczny}

Celem przedmiotu jest zapoznanie studenta z podstawowymi pojęciami aparatu rachunku różniczkowego z zakresu funkcji rzeczywistych jednej zmiennej takich jak: granice ciągów, granice funkcji jednej zmiennej, funkcje ciągłe, pochodne, całki nieoznaczona i oznaczone, szeregi liczbowe, szeregi funkcyjne oraz równania różniczkowe. Przedstawione są podstawowe zastosowania  pochodnych i całek m.in. do wyznaczania przybliżeń, ekstremów,  czy  liczenia pola lub objętości. Wskazane są związki z ekonomią, optymalizacja technologiami  informatycznymi.

\section{Przedmioty wprowadzające}

\begin{tabularx}{\textwidth}{lX}
  \hline
  \textbf{Przedmiot} & \textbf{Wymagane zagadnienia} \\
  \hline
  Algebra liniowa z geometrią & Wiedza z zakresu szkoły średniej. \\
  \hline
\end{tabularx}

\section{Treści programowe}

\begin{enumerate}
  \item Liczby rzeczywiste. Konstrukcja. Własności.
  \item Szeregi liczbowe. Definicja. Kryteria zbieżności. Szeregi naprzemienne.
  \item Ciągi liczbowe. Monotoniczność, ograniczoność, zbieżność. Metody obliczania granic ciągów liczbowych. Zastosowania.
  \item Funkcje (w szczególności wykładnicze i logarytmiczne).  Granice funkcji w punkcie. Monotoniczność, ograniczoność, ciągłość. Zastosowania.
  \item Pochodne funkcji jednej zmiennej. Definicja, podstawowe własności, podstawowe wzory.  Twierdzenia rachunku różniczkowego. Zastosowanie rachunku różniczkowego. Badanie przebiegu zmienności funkcji. Twierdzenie de l’Hospitala.  Wzór Taylora i Maclaurina.
  \item Całka nieoznaczona. Definicja funkcji pierwotnej. Kryteria całkowalności funkcji. Podstawowe metody obliczania całek nieoznaczonych. Całkowanie funkcji wymiernych oraz niewymiernych.
  \item Całka Riemanna. Definicja za pomocą granicy ciągu sum całkowych. Metody obliczania całki Riemanna. Zastosowanie całek oznaczonych.
  \item Rozwijanie funkcji w szereg potęgowy. Zastosowanie rozwinięcia w szereg potęgowy w rachunku różniczkowym.
  \item Równania różniczkowe zwyczajne. Podstawowe metody i zastosowanie.
\end{enumerate}

\section{Efekty kształcenia}

\subsection*{Wiedza}
\begin{itemize}
  \item Student zna i rozumie pojęcie ciągu liczbowego, granicy ciągu, ciągu ograniczonego, monotonicznego. Student zna i rozumie pojęcia i własności funkcji (w szczególności funkcji wykładniczej, logarytmicznej, trygonometrycznej, cyklometrycznej), granicy funkcji, funkcji monotonicznej, ograniczonej, funkcji ciągłej. Zna i rozumie definicję, własności i twierdzenia dotyczące pochodnych funkcji jednej zmiennej. Zna i rozumie definicję oraz sposoby obliczania całek nieoznaczonych i oznaczonych. Zna i rozumie definicje szeregów liczbowych oraz kryteria zbieżności szeregów liczbowych. Student zna metody numerycznego obliczania pochodnych oraz całek oznaczonych. Student zna i rozumie pojęcie szeregu potęgowego.  Student zna i rozumie zastosowanie równań różniczkowych zwyczajnych oraz zna podstawowe metody ich rozwiązywania. Student zna i rozumie sposoby przeprowadzania prostych dowodów twierdzeń matematycznych.
\end{itemize}

\subsection*{Umiejętności}
\begin{itemize}
  \item Student potrafi wyznaczać granice ciągów liczbowych oraz funkcji jednej zmiennej. Potrafi sprawdzić czy funkcja jest ciągła.  Potrafi liczyć pochodne. Zna interpretację geometryczną pochodnej i twierdzeń z nią związaną. Potrafi stosować  regułę de L'Hospitala. Potrafi  wyznaczać wielomian Taylora.  Potrafi zastosować twierdzenie Rolle’a, Lagrange’a i Cauchy’ego. Potrafi  obliczać ekstrema lokalne i globalne funkcji jednej zmiennej.  Potrafi przeprowadzić badanie przebiegu zmienności funkcji. Potrafi liczyć całki oznaczone i nieoznaczone jednej zmiennej oraz wykorzystywać je do zagadnień geometrycznych. Student potrafi opisać związek twierdzeń rachunku różniczkowego i całkowego jednej zamiennej z wybranymi metodami numerycznymi. Student potrafi użyć twierdzenie Taylora do obliczania przybliżonej wartości wyrażenia. Student potrafi określić rząd równania różniczkowego zwyczajnego oraz rozwiązać równanie różniczkowe zwyczajne pierwszego rzędu oraz zagadnienie Cauchy,ego związane z tymi równaniami. Student potrafi przeprowadzić elementarny dowód twierdzenia.
\end{itemize}

\subsection*{Kompetencje społeczne}
\begin{itemize}
  \item Student jest gotów do samodzielnego pozyskiwania informacji z różnych źródeł i przełożenia ich na potrzebny kontekst.
\end{itemize}

\section{Kryteria oceny}

\begin{itemize}
  \item Rozwiązywanie zadań na tablicy
  \item rozwiązywanie zadań na tablicy
  \item dykusja
  \item Kryteria oceny
  \item Dwa kolokwia pisemne (każde kolokwium do 10 zadań). Możliwe dodatkowe aktywności takie jak kartkówki, aktywność.  Student jest zobowiązany uzyskać wynik powyżej 50\% możliwych do zdobycia sumarycznie z obu kolokwiów.
  \item Skala ocen:
  \item Poniżej 50\% - ndst
  \item Od 50\% - dst
  \item Od 60\% - dst+
  \item Od 70\% - db
  \item Od 80\% - db+
  \item Od 90\% - bdb
  \item Poniżej 50\% - ndst
  \item Od 50\% - dst
  \item Od 60\% - dst+
  \item Od 70\% - db
  \item Od 80\% - db+
  \item Od 90\% - bdb
  \item Przed podejściem do egzaminu student musi zaliczyć część ćwiczeniową.
\end{itemize}

\section{Metody dydaktyczne}

Wykład, laboratoria, praca własna studenta.

\section{Literatura}

\textbf{Podstawowa:}
\begin{itemize}
  \item 1. W. Krysicki, L. Włodarski:, „Analiza matematyczna w zadaniach” część 1, PWN (2022 i nowsze).
  \item 2. M. Gewert, Z. Skoczylas, “ Analiza Matematyczna 1, Przykłady i zadania ”, GiS Wroclaw (2021)
  \item 3.  M. Gewert, Z. Skoczylas, “ Analiza Matematyczna 1, Definicje, twierdzenia, wzory"”, GiS Wrocław (2008 i nowsze)
\end{itemize}

\textbf{Uzupełniająca:}
\begin{itemize}
  \item 1. B. Sozański, I. Dziedzic, " Algebra i Analiza w zagadnieniach ekonomicznych", Bila (2009)
  \item 2.  M. Sullivan, " Brief  Calculus, an Applied Approach", John Wiley \& Sons (2005)
\end{itemize}

\end{document}
