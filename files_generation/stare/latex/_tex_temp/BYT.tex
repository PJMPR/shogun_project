% ===========================================================
%  Sylabus: Budowa i integracja systemów informatycznych (BYT)
% ===========================================================
\documentclass[12pt, a4paper]{article}

\usepackage[T1]{fontenc}
\usepackage[utf8]{inputenc}
\usepackage[polish]{babel}
\usepackage{lmodern}
\usepackage{microtype}
\usepackage[a4paper, top=2.5cm, bottom=2.5cm, left=2.5cm, right=2.5cm]{geometry}
\usepackage{xcolor}
\usepackage{graphicx}
\usepackage{booktabs}
\usepackage{tabularx}
\usepackage{longtable}
\usepackage{multirow}
\usepackage{array}
\usepackage{colortbl}
\usepackage{enumitem}
\usepackage{fancyhdr}
\usepackage{titlesec}
\usepackage{mdframed}
\usepackage[colorlinks=true, linkcolor=red!70!black, urlcolor=red!70!black]{hyperref}
\usepackage{eso-pic}
\usepackage{tikz}

\definecolor{pjatkRed}{RGB}{180,0,0}
\definecolor{pjatkGray}{RGB}{80,80,80}
\definecolor{pjatkLightGray}{RGB}{245,245,245}
\definecolor{tableHeader}{RGB}{220,220,220}

\pagestyle{fancy}\fancyhf{}
\renewcommand{\headrulewidth}{0.4pt}
\renewcommand{\footrulewidth}{0.4pt}
\fancyhead[L]{\small\textcolor{pjatkGray}{PJATK -- Filia w Gdańsku \textbar\ Informatyka}}
\fancyhead[R]{\small\textcolor{pjatkGray}{Sylabus: BYT}}
\fancyfoot[C]{\small\thepage}

\titleformat{\section}{\large\bfseries\color{pjatkRed}}{\thesection.}{0.5em}{}
  [\color{pjatkRed}\rule{\linewidth}{0.8pt}]
\setlist{noitemsep, topsep=3pt, parsep=2pt}

\newmdenv[linecolor=pjatkRed, linewidth=1.2pt, backgroundcolor=pjatkLightGray,
  innerleftmargin=10pt, innerrightmargin=10pt, innertopmargin=8pt,
  innerbottommargin=8pt, roundcorner=4pt]{infobox}

\begin{document}

\AddToShipoutPictureBG{%
  \begin{tikzpicture}[remember picture, overlay]
    \node[opacity=0.5] at (current page.center) {%
      \includegraphics[width=14cm]{C:/Users/adamu/WebstormProjects/pj-studies/latex/PJATK_pl_sygnet_transparent-eps-converted-to}%
    };
  \end{tikzpicture}%
}

\begin{center}
  \includegraphics[height=2cm]{C:/Users/adamu/WebstormProjects/pj-studies/latex/PJATK_pl_poziom_1}\\[0.8cm]
  {\LARGE\bfseries\color{pjatkRed} SYLABUS PRZEDMIOTU}\\[0.8cm]
\end{center}

\begin{infobox}
\begin{tabularx}{\textwidth}{@{}lX@{}}
  \textbf{Nazwa przedmiotu:}  & {\bfseries Budowa i integracja systemów informatycznych} \\[3pt]
  \textbf{Kod przedmiotu:}    & BYT \\[3pt]
  \textbf{Kierunek / Profil:} & Informatyka / praktyczny \\[3pt]
  \textbf{Tryb studiów:}      & niestacjonarny \\[3pt]
  \textbf{Rok / Semestr:}     & 4 / 7 \\[3pt]
  \textbf{Charakter:}         & obowiązkowy \\[3pt]
  \textbf{Odpowiedzialny:}    & dr hab. inż. Marta Łabuda \\[3pt]
  \textbf{Wersja z dnia:}     & 19.02.2026 \\
\end{tabularx}
\end{infobox}

\vspace{1cm}

\section{Godziny zajęć i punkty ECTS}

\begin{center}
\begin{tabular}{|>{\centering\arraybackslash}p{2.0cm}
                |>{\centering\arraybackslash}p{2.0cm}
                |>{\centering\arraybackslash}p{2.0cm}
                |>{\centering\arraybackslash}p{2.4cm}
                |>{\centering\arraybackslash}p{2.4cm}
                |>{\centering\arraybackslash}p{2.0cm}
                |>{\centering\arraybackslash}p{1.4cm}|}
\hline
\rowcolor{tableHeader}
\textbf{Wykłady} & \textbf{Ćwiczenia} & \textbf{Laboratorium} &
\textbf{Z prowadzącym} & \textbf{Praca własna} & \textbf{Łącznie} & \textbf{ECTS} \\
\hline
24 h & --- & 32 h & 56 h & 94 h & 150 h & \textbf{6} \\
\hline
\end{tabular}
\end{center}

\section{Forma zajęć}

\begin{tabular}{ll}
  \hline
  \textbf{Forma zajęć} & \textbf{Sposób zaliczenia} \\
  \hline
  Wykład & Egzamin \\
  \hline
\end{tabular}

\section{Cel dydaktyczny}

Celem cyklu wykładów jest zapoznanie słuchaczy z podstawowymi zagadnieniami inżynierii oprogramowania, w tym z fazami rozwoju i ewolucji oprogramowania oraz metodami podwyższenia jakości oprogramowania. Wykład dotyczy podstawowych aspektów inżynierii oprogramowania i jest zorganizowany według kolejnych faz cyklu powstawania oprogramowania. Omówione są fazy: strategiczna, gromadzenia wymagań, analizy, projektowania, konstrukcji, testowania, instalacji i konserwacji a także jego wdrażanie i pielęgnacji. Wykład przedstawia też zagadnienia związane z architekturą systemów informatycznych i dobrymi praktykami projektowania za pomocą narzędzi CASE. Znaczna waga przypisana jest dyskusji aktualnych modeli cyklu życia oprogramowania i kryteriom doboru strategii jego wytworzenia, jak też wykorzystaniu wzorców projektowych i standardowych API oraz dobrym praktykom projektowania i implementacji systemów, w szczególności: systemy rozproszone, systemy krytyczne, czasu rzeczywistego oraz wielokrotnego użytku. Omawiane na zajęciach tematy obejmują też jakość oprogramowania i jej miary, podstawowe informacje o procesie testowania oprogramowania oraz o zarządzaniu przedsięwzięciem programistycznym, zarządzaniu konfiguracją oprogramowania, implementację, a także problemy outsourcingu informatycznego. Charakteryzuje informatyczne systemy wsparcia procesów biznesowych omawiając ich najważniejsze cechy. Zajęcia są skorelowane z wykładem i projektem przedmiotu Projekt informatyczny (PRO), zapewniając kompleksowość przedstawianych zagadnień inżynierii oprogramowania i dając studentom możliwość samodzielnego, metodycznego przeprowadzenia niedużych projektów informatycznych - od zdefiniowania problemu, poprzez dobór strategii i zaplanowanie jego rozwiązania - do zaprojektowania i wykonania praktycznych elementów produktu, z użyciem wzorców i komponentów projektowych i przeprowadzeniem testów systemowych i walidacyjnych. Wzorcowym modelem jest wykorzystanie technologii obiektowej w cyklu kaskadowym, komponentowym lub zwinnie, nie są wykluczone inne strategie wytwórcze np. dla specyficznego produktu jakim są gry komputerowe.

\section{Przedmioty wprowadzające}

\begin{tabularx}{\textwidth}{lX}
  \hline
  \textbf{Przedmiot} & \textbf{Wymagane zagadnienia} \\
  \hline
  • Wstęp do informatyki i architektura komputerów & • Relacyjne bazy danych \\
  • Projektowanie systemów informacyjnych & • Projekt (stowarzyszony) \\
  • Podstawowa wiedza o organizacji systemu komputerowego i jego architekturze & • Znajomość obiektowości, znajomość zagadnień analizy i projektowania SI \\
  • Umiejętność posługiwania się notacją UML & • Rozumienie potrzeb systematyzacji procesu wytwarzania oprogramowania \\
  \hline
\end{tabularx}

\section{Treści programowe}

\begin{enumerate}
  \item Wykład:
  \item Przedmiot i zagadnienia inżynierii oprogramowania. Podstawowe motywacje, społeczny kontekst przedsięwzięcia. Modelowanie pojęciowe. Pojęcie metodyki. Modele cyklu życia oprogramowania
  \item Przebieg i ocena fazy strategicznej. Planowanie projektu. Projekt informatyczny; podstawowe charakterystyki, pojęcia, udziałowcy projektu; cykl życia i zakres projektu. Planowanie zadań. Identyfikacja problemu; Wzbogacony wizerunek (Rich Picture). Decyzje strategiczne i wizja rozwiązania.
  \item Studium Wykonalności projektu informatycznego. Cele, płaszczyzny oceny –techniczna, ekonomiczna, organizacyjna i prawna; ryzyko przedsięwzięcia. Przykłady studiów wykonalności
  \item Analiza wymagań i proces inżynierii wymagań. Wymagania stawiane oprogramowaniu. Cechy dobrego wymagania. Metody pozyskiwania wymagań, user stories. Podział i klasyfikacja wymagań. Miary i formalne techniki  specyfikowania wymagań.
  \item Strategie i procesy prowadzenia projektów informatycznych; tradycyjne (model kaskadowy, model V, prototypowanie, przyrostowy, spiralny) i nowoczesne cykle wytwarzania oprogramowania (reuse i komponentowość), MDA, ponowna inżynieria, RUP.
  \item Zwinne metodyki wytwarzania oprogramowania. Programowanie ekstremalne. SCRUM: procesy, artefakty, role. Dobór strategii prowadzenia projektu.
  \item Analiza i projektowanie architektury systemów; role, procesy, techniki i produkty w podejściu obiektowym oraz agile. Wzorce analizy i projektowania; Klasyfikacja i przykłady wzorców.
  \item Optymalizacja projektu dla specyfiki produktu. Projektowanie systemów rozproszonych, krytycznych, czasu rzeczywistego i wielokrotnego użytku.
  \item Zarządzanie przedsięwzięciem informatycznym. Infrastruktura - dokumentacja procesu wytwarzania i produktu programistycznego, komunikacja, struktury organizacyjne. Raportowanie. Obszary zarządzania. zarządzanie zespołem. Harmonogramowanie.
  \item Implementacja, integracja i wersjonowanie oprogramowania.; środowiska narzędziowe oraz technologie. Zarządzanie konfiguracją oprogramowania.
  \item Pojęcie jakości oprogramowania i dróg jej osiągania. Standardy jakości i dobre praktyki jakości. Zapewnianie jakości oprogramowania. Modele i miary jakości oprogramowania. Wprowadzenie do zarządzania jakością. Inspekcja kodu i analiza bezpieczeństwa. Metody szacowania nakładów. Model COCOMO II.
  \item Testowanie i walidacja oprogramowania. Rodzaje testów testy jednostkowe, integracyjne, systemowe; walidacja i testy akceptacyjne. Typowe fazy i metody testowania. Testy statyczne, funkcjonalne i strukturalne. Czynniki sukcesu i rezultaty testowania. Dokumentowanie testów. Przeglądy oprogramowania.
  \item Instalacja i wdrożenie oprogramowania. Pielęgnacja oprogramowania. Analiza potrzeby wprowadzania modyfikacji. Koszty konserwacji oprogramowania. Kluczowe czynniki sukcesu fazy utrzymania.
  \item Outsourcing w inżynierii oprogramowania.
  \item Informatyczne wsparcie procesów biznesowych. Oprogramowanie klasy CRM, ERP. Architektura systemów ERP. Przykłady dostawców systemów CRM i ERP.
\end{enumerate}

\section{Efekty kształcenia}

\subsection*{Wiedza}
\begin{itemize}
  \item Student zna i rozumie pojęcia w zakresie sieci komputerowych, ich technologii, protokołów komunikacyjnych i zagadnień bezpieczeństwa, telekomunikacji oraz potrzebę przenoszenia dobrych praktyk na grunt informatyki
  \item Student zna i rozumie pojęcia z zakresu kluczowych zagadnień w zarządzania informacją i modelowania danych; szczegółowo zna i rozumie zagadnienia konstrukcji relacyjnych baz danych, ich programowania i przetwarzania transakcji; ma znajomość aktualnie stosowanych systemów baz danych
  \item Student zna i rozumie zaawansowane pojęcia z zakresu zagadnień inżynierii oprogramowania, standardów i kształtu cykli wytwórczych oraz ewolucji oprogramowania; zna podstawy zarządzania przedsięwzięciem programistycznym i rozumie problem jakości oprogramowania; rozumie rolę modelowania i ma szczegółową wiedzę o obiektowym wytwarzaniu oprogramowania i notacji UML, zna i rozumie zasady korzystania z wzorców programowych i standardowych API; ma wiedzę o typowych narzędziach i środowiskach wspomagających;
  \item Student zna i rozumie podstawowe pojęcia z zakresu kluczowych zagadnień inżynierii wymagań, rozumie potrzebę systematycznego budowania i pielęgnacji specyfikacji wymagań; ma szczegółową wiedzę dotyczącą ich specyfikacji, analizy i modelowania z użyciem dostępnych narzędzi
  \item Student zna i rozumie kluczowe pojęcia z zakresu walidacji i testowania oprogramowania
\end{itemize}

\subsection*{Umiejętności}
\begin{itemize}
  \item Student potrafi zaplanować i przeprowadzić procesy pozyskiwania, analizy, specyfikacji i modelowania wymagań wobec oprogramowania oraz ich pielęgnacji
  \item Student potrafi dokonać przeglądu projektu oprogramowania i poprawić jego jakość
  \item Student potrafi uwzględnić społeczny, etyczny i prawny kontekst przedsięwzięcia informatycznego oraz ocenić związane z nim zagrożenia
  \item Student potrafi zaplanować i wytworzyć podstawowe dokumenty związane z realizacją prostego przedsięwzięcia informatycznego, wstępnie ocenić efekty ekonomiczne i społeczne przedsięwzięcia oraz ich wpływ na udziałowców;
\end{itemize}

\section{Kryteria oceny}

\begin{itemize}
  \item Kryteria oceny
  \item Wykład: egzamin pisemny poprzedzony pracami bieżącej oceny, motywującej do systematycznej pracy w trakcie semestru.
\end{itemize}

\section{Metody dydaktyczne}

Wykład, laboratoria, praca własna studenta.

\section{Literatura}

\textbf{Podstawowa:}
\begin{itemize}
  \item 1. Krzysztof Sacha,  Inżynieria Oprogramowania, PWN 2022
  \item 2. Ian Sommerville,  Inżynieria Oprogramowania, wyd.10, PWN 2020
  \item 3. Max Kanat-Alexander, Zrozumieć oprogramowanie, Helion 2019
  \item 4. Robert C. Martin, Czysta architektura. Struktura i design oprogramowania. Przewodnik dla profesjonalistów, Helion 2022
  \item 5. Mike Cohn, Agile. Metodyki zwinne w planowaniu projektów, Helion 2019
  \item 6. Robert C. Martin, Zwinne wytwarzanie oprogramowania. Najlepsze zasady, wzorce i praktyki, Helion 2017
\end{itemize}

\textbf{Uzupełniająca:}
\begin{itemize}
  \item 1. Bernd Bruegge,  Allen H. Dutoit, Inżynieria oprogramowania w ujęciu obiektowym UML, wzorce projektowe i Java Helion 2011
  \item 2. Keeling Michael,Zostań architektem oprogramowania, PWN 2019
  \item 3. Piotr Gaczkowski, Adrian Ostrowski, Architektura oprogramowania bez tajemnic, Helion 20224.
  \item 4. Arnon Axelrod, Automatyzacja testów. Kompletny przewodnik dla testerów oprogramowaniaPWN 2022
  \item 5. Strony domowe do wybranych narzędzi informatycznych. Instrukcje obsługi, przykłady.
\end{itemize}

\end{document}
