% ===========================================================
%  Sylabus: Narzędzia sztucznej Inteligencji (NAI)
% ===========================================================
\documentclass[12pt, a4paper]{article}

\usepackage[T1]{fontenc}
\usepackage[utf8]{inputenc}
\usepackage[polish]{babel}
\usepackage{lmodern}
\usepackage{microtype}
\usepackage[a4paper, top=2.5cm, bottom=2.5cm, left=2.5cm, right=2.5cm]{geometry}
\usepackage{xcolor}
\usepackage{graphicx}
\usepackage{booktabs}
\usepackage{tabularx}
\usepackage{longtable}
\usepackage{multirow}
\usepackage{array}
\usepackage{colortbl}
\usepackage{enumitem}
\usepackage{fancyhdr}
\usepackage{titlesec}
\usepackage{mdframed}
\usepackage[colorlinks=true, linkcolor=red!70!black, urlcolor=red!70!black]{hyperref}
\usepackage{eso-pic}
\usepackage{tikz}

\definecolor{pjatkRed}{RGB}{180,0,0}
\definecolor{pjatkGray}{RGB}{80,80,80}
\definecolor{pjatkLightGray}{RGB}{245,245,245}
\definecolor{tableHeader}{RGB}{220,220,220}

\pagestyle{fancy}\fancyhf{}
\renewcommand{\headrulewidth}{0.4pt}
\renewcommand{\footrulewidth}{0.4pt}
\fancyhead[L]{\small\textcolor{pjatkGray}{PJATK -- Filia w Gdańsku \textbar\ Informatyka}}
\fancyhead[R]{\small\textcolor{pjatkGray}{Sylabus: NAI}}
\fancyfoot[C]{\small\thepage}

\titleformat{\section}{\large\bfseries\color{pjatkRed}}{\thesection.}{0.5em}{}
  [\color{pjatkRed}\rule{\linewidth}{0.8pt}]
\setlist{noitemsep, topsep=3pt, parsep=2pt}

\newmdenv[linecolor=pjatkRed, linewidth=1.2pt, backgroundcolor=pjatkLightGray,
  innerleftmargin=10pt, innerrightmargin=10pt, innertopmargin=8pt,
  innerbottommargin=8pt, roundcorner=4pt]{infobox}

\begin{document}

\AddToShipoutPictureBG{%
  \begin{tikzpicture}[remember picture, overlay]
    \node[opacity=0.5] at (current page.center) {%
      \includegraphics[width=14cm]{C:/Users/adamu/WebstormProjects/pj-studies/latex/PJATK_pl_sygnet_transparent-eps-converted-to}%
    };
  \end{tikzpicture}%
}

\begin{center}
  \includegraphics[height=2cm]{C:/Users/adamu/WebstormProjects/pj-studies/latex/PJATK_pl_poziom_1}\\[0.8cm]
  {\LARGE\bfseries\color{pjatkRed} SYLABUS PRZEDMIOTU}\\[0.8cm]
\end{center}

\begin{infobox}
\begin{tabularx}{\textwidth}{@{}lX@{}}
  \textbf{Nazwa przedmiotu:}  & {\bfseries Narzędzia sztucznej Inteligencji} \\[3pt]
  \textbf{Kod przedmiotu:}    & NAI \\[3pt]
  \textbf{Kierunek / Profil:} & Informatyka / praktyczny \\[3pt]
  \textbf{Tryb studiów:}      & niestacjonarny \\[3pt]
  \textbf{Rok / Semestr:}     & 3 / 5 \\[3pt]
  \textbf{Charakter:}         & obowiązkowy \\[3pt]
  \textbf{Odpowiedzialny:}    &  \\[3pt]
  \textbf{Wersja z dnia:}     & 19.02.2026 \\
\end{tabularx}
\end{infobox}

\vspace{1cm}

\section{Godziny zajęć i punkty ECTS}

\begin{center}
\begin{tabular}{|>{\centering\arraybackslash}p{2.0cm}
                |>{\centering\arraybackslash}p{2.0cm}
                |>{\centering\arraybackslash}p{2.0cm}
                |>{\centering\arraybackslash}p{2.4cm}
                |>{\centering\arraybackslash}p{2.4cm}
                |>{\centering\arraybackslash}p{2.0cm}
                |>{\centering\arraybackslash}p{1.4cm}|}
\hline
\rowcolor{tableHeader}
\textbf{Wykłady} & \textbf{Ćwiczenia} & \textbf{Laboratorium} &
\textbf{Z prowadzącym} & \textbf{Praca własna} & \textbf{Łącznie} & \textbf{ECTS} \\
\hline
16 h & --- & 16 h & 32 h & 68 h & 100 h & \textbf{5} \\
\hline
\end{tabular}
\end{center}

\section{Cel dydaktyczny}

Zadaniem kursu jest zapoznanie studentów z podstawowymi technikami używanymi w sztucznej inteligencji. Techniki te wchodzą obecnie coraz bardziej do głównego nurtu technologii IT i biorą coraz większy udział w zastosowaniach przemysłowych. Ponadto duża część omawianego materiału ma szerszy zasięg i może pomóc przyszłym inżynierom w rozwiązywaniu problemów informatycznych spoza ścisłej AI.

\section{Przedmioty wprowadzające}

\begin{tabularx}{\textwidth}{lX}
  \hline
  \textbf{Przedmiot} & \textbf{Wymagane zagadnienia} \\
  \hline
  PRG1, MAD, ASD & Umiejętność programowania w dowolnym języku programowania, chociaż preferowane języki to C++ oraz Python. Znajomość klasycznych algorytmów i struktur danych \\
  \hline
\end{tabularx}

\section{Treści programowe}

\begin{enumerate}
  \item Wstęp oraz przypomnienie zagadnień które były na innych przedmiotach, ale są konieczne w dalszym przebiegu tego przedmiotu
  \item Klasyczne rozważania na temat sztucznej inteligencji
  \item Podstawy widzenia maszynowego na przykładzie biblioteki OpenCV
  \item Przykłady algorytmów heurystycznych
  \item Podstawy sieci neuronowych z punktu widzenia konstrukcji i algorytmów
  \item Podstawy drzew decyzyjnych
  \item Algorytm A* jako przykład efektywnego rozwinięcia klasyczneo algorytmu o zastosowanie heurystyki
\end{enumerate}

\section{Efekty kształcenia}

\subsection*{Wiedza}
\begin{itemize}
  \item Zna i rozumie zastosowanie pojęć matematycznych w sztucznej inteligencji
  \item Zna i rozumie zastosowanie losowości w rozwiązywaniu problemów obliczeniowych.
\end{itemize}

\subsection*{Umiejętności}
\begin{itemize}
  \item Student potrafi zidentyfikować problemy obliczeniowe które nie nadają się do rozwiązania klasycznymi algorytmami. Wie kiedy zastosować metody przybliżone.
  \item Student potrafi rozwiązywać problemy z wykorzystaniem narzędzi sztucznej inteligencji
  \item Student potrafi wyspecyfikować, zaprojektować, zaimplementować, przetestować oraz zdebuggować program; potrafi korzystać z bibliotek, środowisk programistycznych, integrujących i uruchomieniowych.
\end{itemize}

\section{Kryteria oceny}

\begin{itemize}
  \item Ćwiczenia / Laboratorium/Lektorat:
  \item rozwiązywanie zadań
  \item Laboratorium/Projekt
  \item brak
  \item Kryteria oceny
  \item LaboratoriumPrezentacje niewielkich projektów praktycznych
  \item Ocena jest liczona jak średnia z uzyskanych ocen cząstkowych.
\end{itemize}

\section{Metody dydaktyczne}

Wykład, laboratoria, praca własna studenta.

\section{Literatura}

\textbf{Podstawowa:}
\begin{itemize}
  \item Wykłady do przedmiotu
  \item Dokumentacja biblioteki OpenCV
  \item Dokumentacja biblioteki TensorFlow
  \item Algorytmy genetyczne + struktury danych = programy ewolucyjne, Zbigniew Michalewicz, 2003, ISBN: 8320428815
  \item Algorytmy, Almanach, Gorge Heineman, Gary Pollice Stanley Selkow, Helion
\end{itemize}

\textbf{Uzupełniająca:}
\begin{itemize}
  \item Brak danych.
\end{itemize}

\end{document}
