% ===========================================================
%  Sylabus: Wytwarzanie gier 2 (WG2)
% ===========================================================
\documentclass[12pt, a4paper]{article}

\usepackage[T1]{fontenc}
\usepackage[utf8]{inputenc}
\usepackage[polish]{babel}
\usepackage{lmodern}
\usepackage{microtype}
\usepackage[a4paper, top=2.5cm, bottom=2.5cm, left=2.5cm, right=2.5cm]{geometry}
\usepackage{xcolor}
\usepackage{graphicx}
\usepackage{booktabs}
\usepackage{tabularx}
\usepackage{longtable}
\usepackage{multirow}
\usepackage{array}
\usepackage{colortbl}
\usepackage{enumitem}
\usepackage{fancyhdr}
\usepackage{titlesec}
\usepackage{mdframed}
\usepackage[colorlinks=true, linkcolor=red!70!black, urlcolor=red!70!black]{hyperref}
\usepackage{eso-pic}
\usepackage{tikz}

\definecolor{pjatkRed}{RGB}{180,0,0}
\definecolor{pjatkGray}{RGB}{80,80,80}
\definecolor{pjatkLightGray}{RGB}{245,245,245}
\definecolor{tableHeader}{RGB}{220,220,220}

\pagestyle{fancy}\fancyhf{}
\renewcommand{\headrulewidth}{0.4pt}
\renewcommand{\footrulewidth}{0.4pt}
\fancyhead[L]{\small\textcolor{pjatkGray}{PJATK -- Filia w Gdańsku \textbar\ Informatyka}}
\fancyhead[R]{\small\textcolor{pjatkGray}{Sylabus: WG2}}
\fancyfoot[C]{\small\thepage}

\titleformat{\section}{\large\bfseries\color{pjatkRed}}{\thesection.}{0.5em}{}
  [\color{pjatkRed}\rule{\linewidth}{0.8pt}]
\setlist{noitemsep, topsep=3pt, parsep=2pt}

\newmdenv[linecolor=pjatkRed, linewidth=1.2pt, backgroundcolor=pjatkLightGray,
  innerleftmargin=10pt, innerrightmargin=10pt, innertopmargin=8pt,
  innerbottommargin=8pt, roundcorner=4pt]{infobox}

\begin{document}

\AddToShipoutPictureBG{%
  \begin{tikzpicture}[remember picture, overlay]
    \node[opacity=0.5] at (current page.center) {%
      \includegraphics[width=14cm]{C:/Users/adamu/WebstormProjects/pj-studies/latex/PJATK_pl_sygnet_transparent-eps-converted-to}%
    };
  \end{tikzpicture}%
}

\begin{center}
  \includegraphics[height=2cm]{C:/Users/adamu/WebstormProjects/pj-studies/latex/PJATK_pl_poziom_1}\\[0.8cm]
  {\LARGE\bfseries\color{pjatkRed} SYLABUS PRZEDMIOTU}\\[0.8cm]
\end{center}

\begin{infobox}
\begin{tabularx}{\textwidth}{@{}lX@{}}
  \textbf{Nazwa przedmiotu:}  & {\bfseries Wytwarzanie gier 2} \\[3pt]
  \textbf{Kod przedmiotu:}    & WG2 \\[3pt]
  \textbf{Kierunek / Profil:} & Informatyka / praktyczny \\[3pt]
  \textbf{Tryb studiów:}      & stacjonarny \\[3pt]
  \textbf{Rok / Semestr:}     & 3 / 6 \\[3pt]
  \textbf{Charakter:}         & obowiązkowy \\[3pt]
  \textbf{Odpowiedzialny:}    & Lic. Aleksandr Polin (alex.polin@pjwstk.edu.pl) \\[3pt]
  \textbf{Wersja z dnia:}     & 19.02.2026 \\
\end{tabularx}
\end{infobox}

\vspace{1cm}

\section{Godziny zajęć i punkty ECTS}

\begin{center}
\begin{tabular}{|>{\centering\arraybackslash}p{2.0cm}
                |>{\centering\arraybackslash}p{2.0cm}
                |>{\centering\arraybackslash}p{2.0cm}
                |>{\centering\arraybackslash}p{2.4cm}
                |>{\centering\arraybackslash}p{2.4cm}
                |>{\centering\arraybackslash}p{2.0cm}
                |>{\centering\arraybackslash}p{1.4cm}|}
\hline
\rowcolor{tableHeader}
\textbf{Wykłady} & \textbf{Ćwiczenia} & \textbf{Laboratorium} &
\textbf{Z prowadzącym} & \textbf{Praca własna} & \textbf{Łącznie} & \textbf{ECTS} \\
\hline
30 h & 30 h & --- & 60 h & 65 h & 125 h & \textbf{} \\
\hline
\end{tabular}
\end{center}

\section{Forma zajęć}

\begin{tabular}{ll}
  \hline
  \textbf{Forma zajęć} & \textbf{Sposób zaliczenia} \\
  \hline
  Laboratorium & Zaliczenie z oceną \\
  Wykład & Egzamin \\
  \hline
\end{tabular}

\section{Cel dydaktyczny}

Celem kursu jest pogłębienie wiedzy i umiejętności studentów w zakresie tworzenia gier, skupiając się na zaawansowanych technikach projektowania gier, optymalizacji, marketingu i analizie rynku. Studenci zdobędą praktyczne doświadczenie w tworzeniu gier, które będą działać płynnie na różnych platformach, a także nauczą się, jak skutecznie promować swoje produkty na rynku gier.

\section{Przedmioty wprowadzające}

\begin{tabularx}{\textwidth}{lX}
  \hline
  \textbf{Przedmiot} & \textbf{Wymagane zagadnienia} \\
  \hline
  Wytwarzanie gier 1 (WG1) & Znajomość podstawowych zagadnień projektowania gier, tworzenia dokumentacji gier, Unreal Engine, animacji i dźwięku w grach. Umiejętność pracy w zespole i podstawy zarządzania projektami gier. \\
  \hline
\end{tabularx}

\section{Treści programowe}

\begin{enumerate}
  \item Wprowadzenie i powtórka z WG1
  \item Przypomnienie najważniejszych zagadnień z Wytwarzania gier 1.
  \item Omówienie przykładowych projektów z poprzedniego semestru, analiza mocnych i słabych stron.
  \item Trendy na rynku gier, nowości technologiczne.
  \item Analityka gier
  \item Rodzaje danych zbieranych w grach i ich znaczenie.
  \item Narzędzia analityczne i ich zastosowanie w optymalizacji gier.
  \item Interpretacja danych, wyciąganie wniosków i podejmowanie decyzji na podstawie analizy.
  \item Marketing gier
  \item Wprowadzenie do marketingu w grach.
  \item Definiowanie grupy docelowej, tworzenie persony gracza.
  \item Kanały marketingowe dla gier, strategie promocji.
  \item Rynek gier: VR, PC, Konsole
  \item Omówienie specyfiki, zalet i wad tworzenia gier na różne platformy.
  \item Trendy na poszczególnych rynkach, analiza konkurencji.
  \item Wybór platformy docelowej dla gry, dostosowanie projektu do specyfiki rynku.
  \item Zaawansowane techniki projektowania gier
  \item Projektowanie narracji w grach, budowanie świata gry.
  \item Tworzenie angażującej rozgrywki, projektowanie wyzwań i nagród.
  \item Balansowanie gry, zapewnienie odpowiedniego poziomu trudności.
  \item Optymalizacja gier
  \item Zaawansowana optymalizacja gier na platformy PC i konsolowe.
  \item Optymalizacja grafiki, dźwięku i kodu gry.
  \item Narzędzia do profilowania i optymalizacji, dobre praktyki.
  \item Zwiastuny gier i kampanie społecznościowe
  \item Tworzenie zwiastunów gier przyciągających uwagę graczy.
  \item Prowadzenie kampanii w mediach społecznościowych, budowanie zaangażowania.
  \item Współpraca z influencerami, marketing szeptany.
  \item Psychologia gier: kierowanie uwagą gracza
  \item Techniki narracji wizualnej w grach, budowanie napięcia i emocji.
  \item Prowadzenie gracza przez świat gry, projektowanie intuicyjnego interfejsu.
  \item Wykorzystanie dźwięku i muzyki do budowania atmosfery.
  \item Sztuka w grach: kadrowanie i kompozycja
  \item Podstawy kompozycji obrazu, stosowanie zasad kadrowania w grach.
  \item Język filmu: ujęcia, montaż, ruch kamery.
  \item Analiza udanych przykładów, inspiracje z innych mediów.
  \item Tworzenie gier jako biznes
  \item Modele biznesowe w grach: free-to-play, premium, subskrypcje.
  \item Monetyzacja gier, projektowanie systemu mikropłatności.
  \item Aspekty prawne tworzenia gier, prawa autorskie.
  \item Budowanie relacji z partnerami
  \item Nawiązywanie kontaktów z wydawcami, platformami dystrybucji cyfrowej.
  \item Sposoby pozyskiwania funduszy na tworzenie gier, crowdfunding.
  \item Prezentacja projektu gry, przygotowanie materiałów prasowych.
  \item Budowanie i utrzymywanie społeczności
  \item Sposoby budowania zaangażowanej społeczności wokół gry.
  \item Komunikacja z graczami, moderacja forów i grup dyskusyjnych.
  \item Organizowanie konkursów i eventów, zarządzanie feedbackiem.
  \item Prezentacja projektów
  \item Prezentacja postępów prac nad projektami gier.
  \item Omówienie wyzwań i rozwiązań, analiza i feedback od prowadzącego i grupy.
  \item Dzień gier
  \item Prezentacja finalnych wersji gier.
  \item Testowanie gier przez studentów i zaproszonych gości.
  \item Wręczenie nagród, podsumowanie semestru.
\end{enumerate}

\section{Efekty kształcenia}

\subsection*{Wiedza}
\begin{itemize}
  \item Jest w stanie zastosować teorie i zasady projektowania graficznego oraz interfejsów użytkownika do tworzenia intuicyjnych i angażujących elementów gry.
  \item Potrafi przeprowadzić kompleksowe testy gry, identyfikując i dokumentując błędy, co przyczynia się do wydania produktu o wysokiej jakości.
\end{itemize}

\subsection*{Umiejętności}
\begin{itemize}
  \item Wie, jak koncepcyjnie zaprojektować grę, zastosować narzędzia projektowe jak Unreal Engine i przeprowadzić analizę rynkową.
  \item Posiada kompetencje do tworzenia kompleksowej dokumentacji projektowej, która komunikuje kluczowe aspekty rozwoju gry międzynarodowej publiczności i zespołowi projektowemu.
  \item Demonstruje zdolność do ciągłego doskonalenia swoich umiejętności deweloperskich, korzystając z różnorodnych źródeł informacji i nowoczesnych metod edukacyjnych.
  \item Jest wyposażony w umiejętności niezbędne do przeprowadzania skutecznych testów gier, identyfikowania błędów i zapewniania jakości produktu końcowego.
  \item Demonstruje umiejętność tworzenia elementów gry przy użyciu Unreal Engine, w tym blueprintów i animacji.
\end{itemize}

\subsection*{Kompetencje społeczne}
\begin{itemize}
  \item Wykazuje gotowość do ciągłego rozwijania kompetencji i samokształcenia, co jest kluczowe w szybko zmieniającej się branży gier komputerowych.
  \item Jest gotów do aktywnego uczestnictwa w procesie produkcyjnym gier, pełniąc różnorodne role w zespole deweloperskim i adaptując się do dynamiki projektu gamedev.
  \item Jest przygotowany do efektywnego zarządzania czasem i zasobami, określając priorytety w celu skutecznej realizacji zadań w procesie tworzenia gier.
  \item Rozumie i angażuje się w analizę oraz rozwiązywanie kwestii etycznych i prawnych związanych z projektowaniem i tworzeniem gier.
  \item Demonstruje umiejętności komunikacyjne potrzebne do efektywnego dialogu z różnorodnymi interesariuszami projektu gamedev, w tym inwestorami, w celu tworzenia wartości dodanej dla produktu.
\end{itemize}

\section{Kryteria oceny}

\begin{itemize}
  \item wykład z elementami dyskusji z prezentacją multimedialną, wykład zaproszony
  \item burza mózgów
  \item rozwiązywanie zadań
  \item analiza przypadków
  \item prezentacje
  \item praca w Unreal Engine
  \item Kryteria oceny
  \item Studenci prezentują swoje projekty (gry) i są oceniani przez publiczność punktowo:
  \item Styl (1-10)
  \item Rozgrywka (1-10)
  \item Zabawa (1-10)
  \item Kreatywność (1-10)
  \item Ukończenie (1-10)
  \item Studenci opowiadają także o swoim wkładzie w projekt i przyznają sobie punkty za:
  \item Wysiłek (1-10).
  \item Następnie obliczana jest ocena grupowa. Każda osoba z grupy określa, czy jej ocena końcowa powinna być taka sama jak ocena grupowa, czy też powinna otrzymać ocenę wyższą lub niższą. Ocena ta staje się oceną końcową dla każdego ucznia.
\end{itemize}

\section{Metody dydaktyczne}

Wykład, laboratoria, praca własna studenta.

\section{Literatura}

\textbf{Podstawowa:}
\begin{itemize}
  \item "Game Analytics: Maximizing the Value of Player Data" by Magy Seif El-Nasr, Anders Drachen, Alessandro Canossa (2016)
  \item “Video Game Marketing: A student textbook” by Zackariasson Peter (2016)
  \item "Augmented Reality: Where We Will All Live" by Dr. Jon Peddie (2023)
  \item "The Game Console: A Photographic History from Atari to Xbox" by Evan Amos (2018)
  \item "Game Engine Architecture" by Jason Gregory (2017)
  \item "Unreal Engine VR Cookbook: Developing Virtual Reality with UE4" by Mitch McCaffrey (2017)
  \item “Unreal Engine 5 Game Development with C++ Scripting: Become a professional game developer and create fully functional, high-quality games” by Zhenyu George Li (2023)
\end{itemize}

\textbf{Uzupełniająca:}
\begin{itemize}
  \item "Behavioral Science in the Wild" by Mazar Nina (2022)
  \item "The Psychology of Video Games" by Jamie Madigan (Podcast)
  \item "100 Principles of Game Design" by Despain, Wendy (2012)
  \item "Game Design Workshop: A Playcentric Approach to Creating Innovative Games" by Tracy Fullerton (2018)
  \item "The Indie Game Developer Handbook" by Richard Hill-Whittall (2015)
  \item "Blood, Sweat, and Pixels: The Triumphant, Turbulent Stories Behind How Video Games Are Made" by Jason Schreier (2017)
\end{itemize}

\end{document}
