% ===========================================================
%  Sylabus: Programowanie Obiektowe w C\# (POJ)
% ===========================================================
\documentclass[12pt, a4paper]{article}

\usepackage[T1]{fontenc}
\usepackage[utf8]{inputenc}
\usepackage[polish]{babel}
\usepackage{lmodern}
\usepackage{microtype}
\usepackage[a4paper, top=2.5cm, bottom=2.5cm, left=2.5cm, right=2.5cm]{geometry}
\usepackage{xcolor}
\usepackage{graphicx}
\usepackage{booktabs}
\usepackage{tabularx}
\usepackage{longtable}
\usepackage{multirow}
\usepackage{array}
\usepackage{colortbl}
\usepackage{enumitem}
\usepackage{fancyhdr}
\usepackage{titlesec}
\usepackage{mdframed}
\usepackage[colorlinks=true, linkcolor=red!70!black, urlcolor=red!70!black]{hyperref}
\usepackage{eso-pic}
\usepackage{tikz}

\definecolor{pjatkRed}{RGB}{180,0,0}
\definecolor{pjatkGray}{RGB}{80,80,80}
\definecolor{pjatkLightGray}{RGB}{245,245,245}
\definecolor{tableHeader}{RGB}{220,220,220}

\pagestyle{fancy}\fancyhf{}
\renewcommand{\headrulewidth}{0.4pt}
\renewcommand{\footrulewidth}{0.4pt}
\fancyhead[L]{\small\textcolor{pjatkGray}{PJATK -- Filia w Gdańsku \textbar\ Informatyka}}
\fancyhead[R]{\small\textcolor{pjatkGray}{Sylabus: POJ}}
\fancyfoot[C]{\small\thepage}

\titleformat{\section}{\large\bfseries\color{pjatkRed}}{\thesection.}{0.5em}{}
  [\color{pjatkRed}\rule{\linewidth}{0.8pt}]
\setlist{noitemsep, topsep=3pt, parsep=2pt}

\newmdenv[linecolor=pjatkRed, linewidth=1.2pt, backgroundcolor=pjatkLightGray,
  innerleftmargin=10pt, innerrightmargin=10pt, innertopmargin=8pt,
  innerbottommargin=8pt, roundcorner=4pt]{infobox}

\begin{document}

\AddToShipoutPictureBG{%
  \begin{tikzpicture}[remember picture, overlay]
    \node[opacity=0.5] at (current page.center) {%
      \includegraphics[width=14cm]{C:/Users/adamu/WebstormProjects/pj-studies/latex/PJATK_pl_sygnet_transparent-eps-converted-to}%
    };
  \end{tikzpicture}%
}

\begin{center}
  \includegraphics[height=2cm]{C:/Users/adamu/WebstormProjects/pj-studies/latex/PJATK_pl_poziom_1}\\[0.8cm]
  {\LARGE\bfseries\color{pjatkRed} SYLABUS PRZEDMIOTU}\\[0.8cm]
\end{center}

\begin{infobox}
\begin{tabularx}{\textwidth}{@{}lX@{}}
  \textbf{Nazwa przedmiotu:}  & {\bfseries Programowanie Obiektowe w C\#} \\[3pt]
  \textbf{Kod przedmiotu:}    & POJ \\[3pt]
  \textbf{Kierunek / Profil:} & Informatyka / praktyczny \\[3pt]
  \textbf{Tryb studiów:}      & stacjonarny \\[3pt]
  \textbf{Rok / Semestr:}     & 1 / 2 \\[3pt]
  \textbf{Charakter:}         & obowiązkowy \\[3pt]
  \textbf{Odpowiedzialny:}    & mgr inż. Adam Urbanowicz \\[3pt]
  \textbf{Wersja z dnia:}     & 15.02.2025 \\
\end{tabularx}
\end{infobox}

\vspace{1cm}

\section{Godziny zajęć i punkty ECTS}

\begin{center}
\begin{tabular}{|>{\centering\arraybackslash}p{2.0cm}
                |>{\centering\arraybackslash}p{2.0cm}
                |>{\centering\arraybackslash}p{2.0cm}
                |>{\centering\arraybackslash}p{2.4cm}
                |>{\centering\arraybackslash}p{2.4cm}
                |>{\centering\arraybackslash}p{2.0cm}
                |>{\centering\arraybackslash}p{1.4cm}|}
\hline
\rowcolor{tableHeader}
\textbf{Wykłady} & \textbf{Ćwiczenia} & \textbf{Laboratorium} &
\textbf{Z prowadzącym} & \textbf{Praca własna} & \textbf{Łącznie} & \textbf{ECTS} \\
\hline
30 h & --- & 30 h & 60 h & 40 h & 100 h & \textbf{3} \\
\hline
\end{tabular}
\end{center}

\section{Forma zajęć}

\begin{tabular}{ll}
  \hline
  \textbf{Forma zajęć} & \textbf{Sposób zaliczenia} \\
  \hline
  Laboratorium & Zaliczenie z oceną \\
  Wykład & Nieoceniany \\
  \hline
\end{tabular}

\section{Cel dydaktyczny}

Celem zajęć jest zapoznanie studentów z podstawowymi koncepcjami programowania obiektowego na przykładzie języka C\# i platformy .NET. Omawiane są pojęcia interfejs, klasa, obiekt, dziedziczenie, polimorfizm, enkapsulacja i kompozycja. Prezentowane są koncepcje i praktyka związana z wykorzystaniem kolekcji generycznych, wyjątków, delegatów i zdarzeń, operacji wejścia/wyjścia oraz wybranych elementów nowoczesnego C\# (m.in. LINQ).

\section{Przedmioty wprowadzające}

\begin{tabularx}{\textwidth}{lX}
  \hline
  \textbf{Przedmiot} & \textbf{Wymagane zagadnienia} \\
  \hline
  PRG1 & znajomość i umiejętność stosowania podstaw programowania strukturalnego (instrukcja warunkowa, pętle, tablice, funkcje i procedury) \\
  \hline
\end{tabularx}

\section{Treści programowe}

\begin{enumerate}
  \item Wprowadzenie do programowania w C\#: typy danych, definicja zmiennych, instrukcje sterujące. Tworzenie prostych programów z wykorzystaniem metody Main oraz klasy Console.
  \item Metody statyczne, tablice i ich praktyczne zastosowanie; interakcja człowiek – komputer z użyciem Console.ReadLine/WriteLine, parsowanie danych wejściowych.
  \item Programowanie obiektowe: definicja klas, tworzenie obiektów, konstruktory, pola i właściwości (properties). Implementacja metod niestatycznych.
  \item Enkapsulacja, modyfikatory dostępu, zasady przechowywania danych w pamięci: stos i sterta w kontekście .NET; typy wartościowe i referencyjne.
  \item Dziedziczenie: override oraz overload, metody wirtualne, przesłanianie składowych, dziedziczenie po klasie object (ToString, Equals, GetHashCode).
  \item Klasy abstrakcyjne, polimorfizm, kompozycja; wprowadzenie do przestrzeni nazw i organizacji projektu.
  \item Parametry opcjonalne i nazwane, params (zmienna liczba parametrów) oraz omówienie wyjątków (try/catch/finally), własne wyjątki.
  \item Wyrażenia regularne w .NET, operacje na plikach (System.IO), serializacja w podstawowym zakresie (np. JSON jako format danych – wprowadzenie).
  \item Interfejsy w C\#, implementacja wielokrotna interfejsów; interfejs IComparable, wprowadzenie do typów generycznych.
  \item Kolekcje generyczne: List, Dictionary, HashSet; iteracja (foreach), porównywanie i sortowanie; wprowadzenie do LINQ (Where/Select/OrderBy).
  \item Delegaty i zdarzenia: model zdarzeniowy w C\#, Action/Func, event; podstawy programowania zdarzeniowego na prostych przykładach.
  \item Omówienie wybranych wzorców projektowych (np. Singleton, Factory, Strategy) oraz wstęp do programowania asynchronicznego w C\# (async/await) w kontekście prostych operacji I/O.
\end{enumerate}

\section{Efekty kształcenia}

\subsection*{Wiedza}
\begin{itemize}
  \item Student zna i rozumie podstawowe konstrukcje programistyczne oraz struktury danych, jak też ich podstawowe implementacje w języku C\# i środowisku .NET
\end{itemize}

\subsection*{Umiejętności}
\begin{itemize}
  \item Student potrafi ocenić poprawność konstrukcji obiektowych w programach w języku C\#.
  \item Potrafi skonstruować i uruchomić program obiektowy w języku C\# (aplikacja konsolowa) z wykorzystaniem platformy .NET.
  \item Student potrafi znajdować błędy w tworzonych programach obiektowych przy użyciu wybranych środowisk uruchomieniowych i debuggera.
  \item Potrafi korzystać z bibliotek .NET (m.in. System, System.Collections.Generic, System.IO, System.Text.RegularExpressions).
  \item Student potrafi korzystać ze środowiska developerskiego przeznaczonego do tworzenia programów w języku C\# (np. Visual Studio / Rider / VS Code), zaplanować prostą hierarchię klas oraz zastosować proste wzorce obiektowe zależnie od przedstawionego problemu.
\end{itemize}

\section{Kryteria oceny}

\begin{itemize}
  \item rozwiązywanie zadań
  \item Kryteria oceny
  \item Ocena pracy podczas ćwiczenia – ocena zadań wykonanych przez studentów oddanych do końca zajęć.
  \item Skala ocen:
  \item Poniżej 50\% - ndst
  \item Od 50\% - dst
  \item Od 60\% - dst+
  \item Od 70\% - db
  \item Od 80\% - db+
  \item Od 90\% - bdb
  \item Brak
\end{itemize}

\section{Metody dydaktyczne}

Wykład, laboratoria, praca własna studenta.

\section{Literatura}

\textbf{Podstawowa:}
\begin{itemize}
  \item Brak danych.
\end{itemize}

\textbf{Uzupełniająca:}
\begin{itemize}
  \item Andrew Troelsen, Phil Japikse - Pro C\# with .NET, Apress (najnowsze wydanie)
  \item Mark J. Price - C\# (najnowsze wydanie, Packt)
  \item Microsoft Learn - Dokumentacja języka C\# i platformy .NET
  \item Jon Skeet - C\# in Depth, Manning Publications
\end{itemize}

\end{document}
