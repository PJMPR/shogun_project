% ===========================================================
%  Sylabus: Fizyka (FIZ)
% ===========================================================
\documentclass[12pt, a4paper]{article}

\usepackage[T1]{fontenc}
\usepackage[utf8]{inputenc}
\usepackage[polish]{babel}
\usepackage{lmodern}
\usepackage{microtype}
\usepackage[a4paper, top=2.5cm, bottom=2.5cm, left=2.5cm, right=2.5cm]{geometry}
\usepackage{xcolor}
\usepackage{graphicx}
\usepackage{booktabs}
\usepackage{tabularx}
\usepackage{longtable}
\usepackage{multirow}
\usepackage{array}
\usepackage{colortbl}
\usepackage{enumitem}
\usepackage{fancyhdr}
\usepackage{titlesec}
\usepackage{mdframed}
\usepackage[colorlinks=true, linkcolor=red!70!black, urlcolor=red!70!black]{hyperref}
\usepackage{eso-pic}
\usepackage{tikz}

\definecolor{pjatkRed}{RGB}{180,0,0}
\definecolor{pjatkGray}{RGB}{80,80,80}
\definecolor{pjatkLightGray}{RGB}{245,245,245}
\definecolor{tableHeader}{RGB}{220,220,220}

\pagestyle{fancy}\fancyhf{}
\renewcommand{\headrulewidth}{0.4pt}
\renewcommand{\footrulewidth}{0.4pt}
\fancyhead[L]{\small\textcolor{pjatkGray}{PJATK -- Filia w Gdańsku \textbar\ Informatyka}}
\fancyhead[R]{\small\textcolor{pjatkGray}{Sylabus: FIZ}}
\fancyfoot[C]{\small\thepage}

\titleformat{\section}{\large\bfseries\color{pjatkRed}}{\thesection.}{0.5em}{}
  [\color{pjatkRed}\rule{\linewidth}{0.8pt}]
\setlist{noitemsep, topsep=3pt, parsep=2pt}

\newmdenv[linecolor=pjatkRed, linewidth=1.2pt, backgroundcolor=pjatkLightGray,
  innerleftmargin=10pt, innerrightmargin=10pt, innertopmargin=8pt,
  innerbottommargin=8pt, roundcorner=4pt]{infobox}

\begin{document}

\AddToShipoutPictureBG{%
  \begin{tikzpicture}[remember picture, overlay]
    \node[opacity=0.5] at (current page.center) {%
      \includegraphics[width=14cm]{C:/Users/adamu/WebstormProjects/pj-studies/latex/PJATK_pl_sygnet_transparent-eps-converted-to}%
    };
  \end{tikzpicture}%
}

\begin{center}
  \includegraphics[height=2cm]{C:/Users/adamu/WebstormProjects/pj-studies/latex/PJATK_pl_poziom_1}\\[0.8cm]
  {\LARGE\bfseries\color{pjatkRed} SYLABUS PRZEDMIOTU}\\[0.8cm]
\end{center}

\begin{infobox}
\begin{tabularx}{\textwidth}{@{}lX@{}}
  \textbf{Nazwa przedmiotu:}  & {\bfseries Fizyka} \\[3pt]
  \textbf{Kod przedmiotu:}    & FIZ \\[3pt]
  \textbf{Kierunek / Profil:} & Informatyka / praktyczny \\[3pt]
  \textbf{Tryb studiów:}      & niestacjonarny \\[3pt]
  \textbf{Rok / Semestr:}     & 2 / 3 \\[3pt]
  \textbf{Charakter:}         & obowiązkowy \\[3pt]
  \textbf{Odpowiedzialny:}    &  \\[3pt]
  \textbf{Wersja z dnia:}     & 19.02.2026 \\
\end{tabularx}
\end{infobox}

\vspace{1cm}

\section{Godziny zajęć i punkty ECTS}

\begin{center}
\begin{tabular}{|>{\centering\arraybackslash}p{2.0cm}
                |>{\centering\arraybackslash}p{2.0cm}
                |>{\centering\arraybackslash}p{2.0cm}
                |>{\centering\arraybackslash}p{2.4cm}
                |>{\centering\arraybackslash}p{2.4cm}
                |>{\centering\arraybackslash}p{2.0cm}
                |>{\centering\arraybackslash}p{1.4cm}|}
\hline
\rowcolor{tableHeader}
\textbf{Wykłady} & \textbf{Ćwiczenia} & \textbf{Laboratorium} &
\textbf{Z prowadzącym} & \textbf{Praca własna} & \textbf{Łącznie} & \textbf{ECTS} \\
\hline
16 h & --- & 16 h & 32 h & 43 h & 75 h & \textbf{3} \\
\hline
\end{tabular}
\end{center}

\section{Forma zajęć}

\begin{tabular}{ll}
  \hline
  \textbf{Forma zajęć} & \textbf{Sposób zaliczenia} \\
  \hline
  Wykład & Nieoceniany \\
  \hline
\end{tabular}

\section{Cel dydaktyczny}

Celem przedmiotu jest zapoznanie studentów informatyki z podstawami fizyki na ogólnym poziomie akademickim, w szczególności (1) poznanie podstawowych wielkości fizycznych w zakresie kinematyki i dynamiki punktu materialnego i bryły sztywnej, (2) poznanie praw dynamiki i prawa powszechnego ciążenia, (3) omówienie zasad zachowania w mechanice, (4) scharakteryzowanie drgań mechanicznych oraz ruchu falowego, (5) zapoznanie z podstawowymi wielkościami charakteryzującymi pole elektryczne i magnetyczne oraz (6) omówienie podstawowych praw termodynamiki

\section{Przedmioty wprowadzające}

\begin{tabularx}{\textwidth}{lX}
  \hline
  \textbf{Przedmiot} & \textbf{Wymagane zagadnienia} \\
  \hline
  ALG – algebra liniowa z geometrią & Wiedza z zakresu szkoły średniej. \\
  \hline
\end{tabularx}

\section{Treści programowe}

\begin{enumerate}
  \item Kinematyka punktu materialnego
  \item Określenie ruchu (opis położenia punktu w przestrzeni, układ odniesienia, droga, tory ruchów płaskich). Ruch prostoliniowy jednostajny i jednostajnie zmienny (prędkość chwilowa, prędkość średnia, przyspieszenie stałe, przyspieszenie zmienne). Ruch po okręgu (ruch jednostajny po okręgu, ruch niejednostajny po okręgu). Ruchy ciał w polu grawitacyjnym Ziemi (swobodny spadek ciał, rzut pionowy do góry, rzut poziomy, rzut ukośny).
  \item Kinematyka bryły sztywnej
  \item Dynamika punktu materialnego i bryły sztywnej Dynamika punktu materialnego (zasady dynamiki Newtona, klasyfikacja sił, siły bezwładności, masa i ciężar ciała, tarcie, ogólniejsze ujęcie II zasady dynamiki Newtona). Dynamika ruchu obrotowego bryły sztywnej (moment siły, moment bezwładności, energia kinetyczna w ruchu obrotowym, moment pędu, zasada zachowania momentu pędu).
  \item Zasady zachowania w mechanice
  \item Zasada zachowania energii mechanicznej(siły zachowawcze i niezachowawcze, praca wykonana przez siłę stałą i zmienną, energia kinetyczna, energia potencjalna). Zasada zachowania pędu(pęd punktu materialnego, pęd układu punktów materialnych, zderzenia sprężyste i niesprężyste). Zasada zachowania momentu pędu (moment pędu, prędkość kątowa).
  \item Drgania mechaniczne
  \item Ruch harmoniczny (oscylator harmoniczny prosty, energia w prostym ruchu harmonicznym, wahadło matematyczne, wahadło fizyczne). Drgania harmoniczne tłumione Drgania wymuszone. Rezonans
  \item Ruch falowy
  \item Ogólna charakterystyka fal mechanicznych (podział fal, mechanizm rozchodzenia się fal ośrodkach sprężystych, równanie fali płaskiej harmonicznej). Zasada Huygensa (ugięcie, odbicie i załamanie fal). Superpozycja fal (graficzne przedstawienie powstawania fali stojącej, powstawanie fali stojącej podczas interferencji fali padającej i odbitej). Fale dźwiękowe (efekt Dopplera).
  \item Termodynamika
  \item Kalorymetria Gazy (równanie stanu gazu doskonałego, prawa charakteryzujące przemiany gazowe). I zasada termodynamiki (energia wewnętrzna, zastosowanie I zasady termodynamiki do izoprzemian gazu doskonałego, graficzne przedstawienie pracy) II zasada termodynamiki(entropia, sprawność silnika termodynamicznego, cykl Carnota).
  \item Podstawy elektrostatyki
  \item Wielkości charakteryzujące pole elektrostatyczne(ładunek elektryczny, pole elektrostatyczne, natężenie pola, prawo Coulomba, potencjał pola el.). Prawo Gaussa (zastosowanie prawa Gaussa). Kondensatory
  \item Pole magnetyczne – podstawy
  \item Wektor indukcji magnetycznej Siła Lorentza, Pole magnetyczne w otoczeniu przewodnika z prądem Ruch cząstki naładowanej w polu magnetycznym Prawo Biota-Savarta
\end{enumerate}

\section{Efekty kształcenia}

\subsection*{Wiedza}
\begin{itemize}
  \item Student ma wiedzę z zakresu fizyki, obejmującą dziedziny przydatne dla studiów na kierunku informatyka, w tym elementy mechaniki klasycznej, podstawy elektryczności i magnetyzmu oraz optyki i akustyki.
  \item Dodatkowo:
  \item rozumie rolę eksperymentu fizycznego, matematycznych modeli teoretycznych przybliżających rzeczywistość oraz symulacji komputerowych w metodologii badań naukowych; ma świadomość ograniczeń technologicznych, aparaturowych i metodologicznych w badaniach naukowych;
  \item wie, jak zaplanować i wykonać prosty eksperyment fizyczny oraz przeanalizować otrzymane wyniki; zna elementy teorii niepewności pomiarowych w zastosowaniu do eksperymentów fizycznych;
\end{itemize}

\subsection*{Umiejętności}
\begin{itemize}
  \item Student ma umiejętność analizowania i wyjaśniania obserwowanych zjawisk; tworzenia i weryfikacji modeli świata rzeczywistego oraz posługiwania się nimi w celu predykcji zdarzeń i stanów; potrafi posłużyć się właściwe dobranymi środowiskami programistycznymi, symulatorami oraz narzędziami wspomagania komputerowego do symulacji, projektowania i analizy prostych systemów
  \item Dodatkowo:
  \item użytkuje komputer w zakresie koniecznym do wyszukiwania informacji, komunikowania się, organizowania i wstępnej analizy danych, sporządzania raportów i prezentacji wyników;
  \item posługuje się terminologią z zakresu fizyki oraz nomenklaturą poszczególnych dyscyplin z nią związanych;
  \item stosuje podstawowe metody matematyczne, statystyczne i techniki informatyczne do opisu zjawisk i analizy danych;
  \item Student potrafi zaplanować i dobrać właściwe metody i urządzenia do przeprowadzenia eksperymentu w postaci pomiaru lub symulacji komputerowej, w celu weryfikacji działania oraz identyfikacji parametrów i właściwości systemu, z zachowaniem zasad BHP
\end{itemize}

\subsection*{Kompetencje społeczne}
\begin{itemize}
  \item Student rozumie potrzebę uczenia się przez całe życie; potrafi inspirować i organizować proces uczenia się innych osób
\end{itemize}

\section{Kryteria oceny}

\begin{itemize}
  \item Ćwiczenia / Laboratorium:
  \item rozwiązywanie zadań
  \item burza mózgów
  \item Ćwiczenia/Laboratorium
  \item Kryteria oceny
  \item Ćwiczenia/Laboratorium/Projekt/Lektorat
  \item Cztery krótkie kolokwia sprawdzające wiedzę praktyczną w formie krótkich zadań do rozwiązania oraz 4 sprawozdania pisemne z przeprowadzonych doświadczeń fizycznych.
  \item Skala ocen:
  \item Poniżej 50\% - ndst
  \item Od 50\% - dst
  \item Od 60\% - dst+
  \item Od 70\% - db
  \item Od 80\% - db+
  \item Od 90\% - bdb
  \item Brak egzaminu
\end{itemize}

\section{Metody dydaktyczne}

Wykład, laboratoria, praca własna studenta.

\section{Literatura}

\textbf{Podstawowa:}
\begin{itemize}
  \item D. Halliday, R. Resnick, J. Walter: Podstawy fizyki, PWN 2015.
  \item Cz. Bobrowski: Fizyka – krótki kurs, WNT 2003
  \item G. Białkowski: Mechanika klasyczna, PWN 1975.
  \item K. Wróblewski, J. A. Zakrzewski: Wstęp do fizyki, t. 1-2, PWN 1991.
  \item W. Rubinowicz W. Królikowski: Mechanika teoretyczna, PWN 1998.
\end{itemize}

\textbf{Uzupełniająca:}
\begin{itemize}
  \item Halliday, R. Resnick, J. Walker: Fundamentals of Physics, Wiley 2002.
  \item Berkeley Physics Course, volumes 1-5, McGraw-Hill Company.
  \item W. Christian, M. Belloni: Physlet Physics, Pearson Education, 2004.
\end{itemize}

\end{document}
