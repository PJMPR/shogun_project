% ===========================================================
%  Sylabus: Statystyczna analiza danych (SAD)
% ===========================================================
\documentclass[12pt, a4paper]{article}

\usepackage[T1]{fontenc}
\usepackage[utf8]{inputenc}
\usepackage[polish]{babel}
\usepackage{lmodern}
\usepackage{microtype}
\usepackage[a4paper, top=2.5cm, bottom=2.5cm, left=2.5cm, right=2.5cm]{geometry}
\usepackage{xcolor}
\usepackage{graphicx}
\usepackage{booktabs}
\usepackage{tabularx}
\usepackage{longtable}
\usepackage{multirow}
\usepackage{array}
\usepackage{colortbl}
\usepackage{enumitem}
\usepackage{fancyhdr}
\usepackage{titlesec}
\usepackage{mdframed}
\usepackage[colorlinks=true, linkcolor=red!70!black, urlcolor=red!70!black]{hyperref}
\usepackage{eso-pic}
\usepackage{tikz}

\definecolor{pjatkRed}{RGB}{180,0,0}
\definecolor{pjatkGray}{RGB}{80,80,80}
\definecolor{pjatkLightGray}{RGB}{245,245,245}
\definecolor{tableHeader}{RGB}{220,220,220}

\pagestyle{fancy}\fancyhf{}
\renewcommand{\headrulewidth}{0.4pt}
\renewcommand{\footrulewidth}{0.4pt}
\fancyhead[L]{\small\textcolor{pjatkGray}{PJATK -- Filia w Gdańsku \textbar\ Informatyka}}
\fancyhead[R]{\small\textcolor{pjatkGray}{Sylabus: SAD}}
\fancyfoot[C]{\small\thepage}

\titleformat{\section}{\large\bfseries\color{pjatkRed}}{\thesection.}{0.5em}{}
  [\color{pjatkRed}\rule{\linewidth}{0.8pt}]
\setlist{noitemsep, topsep=3pt, parsep=2pt}

\newmdenv[linecolor=pjatkRed, linewidth=1.2pt, backgroundcolor=pjatkLightGray,
  innerleftmargin=10pt, innerrightmargin=10pt, innertopmargin=8pt,
  innerbottommargin=8pt, roundcorner=4pt]{infobox}

\begin{document}

\AddToShipoutPictureBG{%
  \begin{tikzpicture}[remember picture, overlay]
    \node[opacity=0.5] at (current page.center) {%
      \includegraphics[width=14cm]{C:/Users/adamu/WebstormProjects/pj-studies/latex/PJATK_pl_sygnet_transparent-eps-converted-to}%
    };
  \end{tikzpicture}%
}

\begin{center}
  \includegraphics[height=2cm]{C:/Users/adamu/WebstormProjects/pj-studies/latex/PJATK_pl_poziom_1}\\[0.8cm]
  {\LARGE\bfseries\color{pjatkRed} SYLABUS PRZEDMIOTU}\\[0.8cm]
\end{center}

\begin{infobox}
\begin{tabularx}{\textwidth}{@{}lX@{}}
  \textbf{Nazwa przedmiotu:}  & {\bfseries Statystyczna analiza danych} \\[3pt]
  \textbf{Kod przedmiotu:}    & SAD \\[3pt]
  \textbf{Kierunek / Profil:} & Informatyka / praktyczny \\[3pt]
  \textbf{Tryb studiów:}      & niestacjonarny \\[3pt]
  \textbf{Rok / Semestr:}     & 2 / 3 \\[3pt]
  \textbf{Charakter:}         & obieralny \\[3pt]
  \textbf{Odpowiedzialny:}    & Dr Monika Wrzosek \\[3pt]
  \textbf{Wersja z dnia:}     & 19.02.2026 \\
\end{tabularx}
\end{infobox}

\vspace{1cm}

\section{Godziny zajęć i punkty ECTS}

\begin{center}
\begin{tabular}{|>{\centering\arraybackslash}p{2.0cm}
                |>{\centering\arraybackslash}p{2.0cm}
                |>{\centering\arraybackslash}p{2.0cm}
                |>{\centering\arraybackslash}p{2.4cm}
                |>{\centering\arraybackslash}p{2.4cm}
                |>{\centering\arraybackslash}p{2.0cm}
                |>{\centering\arraybackslash}p{1.4cm}|}
\hline
\rowcolor{tableHeader}
\textbf{Wykłady} & \textbf{Ćwiczenia} & \textbf{Laboratorium} &
\textbf{Z prowadzącym} & \textbf{Praca własna} & \textbf{Łącznie} & \textbf{ECTS} \\
\hline
16 h & --- & 16 h & 32 h & 93 h & 125 h & \textbf{5} \\
\hline
\end{tabular}
\end{center}

\section{Forma zajęć}

\begin{tabular}{ll}
  \hline
  \textbf{Forma zajęć} & \textbf{Sposób zaliczenia} \\
  \hline
  Laboratorium & Zaliczenie z oceną \\
  Wykład & Egzamin \\
  \hline
\end{tabular}

\section{Cel dydaktyczny}

Celem przedmiotu jest zapoznanie studentów ze sposobami prezentacji i analizy danych oraz narzędziami modelowania probabilistycznego; przedstawienie metod statystyki matematycznej i wnioskowania statystycznego oraz nabycieprzez studentów praktycznych umiejętności wykorzystania oprogramowania do statystycznej analizy danych.

\section{Przedmioty wprowadzające}

\begin{tabularx}{\textwidth}{lX}
  \hline
  \textbf{Przedmiot} & \textbf{Wymagane zagadnienia} \\
  \hline
  Analiza matematyczna, Programowanie & Elementy rachunku różniczkowego i całkowego, podstawy programowania. \\
  \hline
\end{tabularx}

\section{Treści programowe}

\begin{enumerate}
  \item Prawdopodobieństwo klasyczne i warunkowe. Niezależność zdarzeń. Prawdopodobieństwo całkowite. Wzór Bayesa. Niezawodność systemu.
  \item Zmienne losowe dyskretne, dystrybuanta, wartość oczekiwana i wariancja. Rozkłady dwumianowy, ujemny dwumianowy i Poissona.
  \item Zmienne losowe ciągłe, gęstość, dystrybuanta, wartość oczekiwana i wariancja. Rozkłady jednostajny, geometryczny, wykładniczy, gamma i normalny. Centralne twierdzenie graniczne.
  \item Zmienne losowe dwuwymiarowa. Rozkłady brzegowe. Niezależność zmiennych losowych.
  \item Zastosowania metody Monte Carlo.
  \item Elementy procesów stochastycznych. Łańcuchy Markowa.
  \item Elementy teorii kolejek.
  \item Statystyka opisowa.Miary położenia, rozproszenia, asymetrii, koncentracji.
  \item Estymacja punktowa i przedziałowa parametrów rozkładów.
  \item Testowanie hipotez statystycznych.
  \item Model regresji liniowej.
\end{enumerate}

\section{Efekty kształcenia}

\subsection*{Wiedza}
\begin{itemize}
  \item Student ma wiedzę z zakresu rachunku prawdopodobieństwa, statystyki matematycznej i analizy danych wykorzystywaną w rozwiązywaniu prostych zadań inżynierskich w informatyce.
  \item Student zna podstawowe pojęcia i metody wnioskowania statystycznego oraz ich zastosowania w praktyce informatycznej.
\end{itemize}

\subsection*{Umiejętności}
\begin{itemize}
  \item Student potrafi zastosować poznane metody statystyki opisowej, modelowania probabilistycznego i wnioskowania statystycznego do rozwiązywania zadań informatycznych.
  \item Student potrafi przeprowadzić analizę prostego problemu analizy danych, dokonać wyboru narzędzi oraz przedstawić i zinterpretować uzyskane wyniki przy użyciu odpowiedniego oprogramowania.
\end{itemize}

\subsection*{Kompetencje społeczne}
\begin{itemize}
  \item Student jest gotów do poszerzania kompetencji w zakresie statystycznej analizy danych.
\end{itemize}

\section{Kryteria oceny}

\begin{itemize}
  \item rozwiązywanie zadań
  \item Kryteria oceny
  \item Laboratorium–dwa kolokwia
  \item Skala ocen:
  \item Poniżej 50\% - ndst
  \item Od 50\% - dst
  \item Od 60\% - dst+
  \item Od 70\% - db
  \item Od 80\% - db+
  \item Od 90\% - bdb
  \item Wykład– egzamin pisemny
  \item Skala ocen:
  \item Poniżej 50\% - ndst
  \item Od 50\% - dst
  \item Od 60\% - dst+
  \item Od 70\% - db
  \item Od 80\% - db+
  \item Od 90\% - bdb
  \item Przed podejściem do egzaminu student musi zaliczyć część ćwiczeniową.
\end{itemize}

\section{Metody dydaktyczne}

Wykład, laboratoria, praca własna studenta.

\section{Literatura}

\textbf{Podstawowa:}
\begin{itemize}
  \item Michael Baron, Probability and Statistics for Computer Scientists, 3rd edition. American University Chapman \& Hall/CRC, 2019.
\end{itemize}

\textbf{Uzupełniająca:}
\begin{itemize}
  \item Koronacki J., Mielniczuk J., Statystyka dla studentów kierunków technicznych i przyrodniczych, WNT, 2018.
  \item Plucińska A., Pluciński E., Probabilistyka, PWN, 2017.
\end{itemize}

\end{document}
