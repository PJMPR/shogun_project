% ===========================================================
%  Sylabus: Wstęp do Zarządzania (WDZ)
% ===========================================================
\documentclass[12pt, a4paper]{article}

\usepackage[T1]{fontenc}
\usepackage[utf8]{inputenc}
\usepackage[polish]{babel}
\usepackage{lmodern}
\usepackage{microtype}
\usepackage[a4paper, top=2.5cm, bottom=2.5cm, left=2.5cm, right=2.5cm]{geometry}
\usepackage{xcolor}
\usepackage{graphicx}
\usepackage{booktabs}
\usepackage{tabularx}
\usepackage{longtable}
\usepackage{multirow}
\usepackage{array}
\usepackage{colortbl}
\usepackage{enumitem}
\usepackage{fancyhdr}
\usepackage{titlesec}
\usepackage{mdframed}
\usepackage[colorlinks=true, linkcolor=red!70!black, urlcolor=red!70!black]{hyperref}
\usepackage{eso-pic}
\usepackage{tikz}

\definecolor{pjatkRed}{RGB}{180,0,0}
\definecolor{pjatkGray}{RGB}{80,80,80}
\definecolor{pjatkLightGray}{RGB}{245,245,245}
\definecolor{tableHeader}{RGB}{220,220,220}

\pagestyle{fancy}\fancyhf{}
\renewcommand{\headrulewidth}{0.4pt}
\renewcommand{\footrulewidth}{0.4pt}
\fancyhead[L]{\small\textcolor{pjatkGray}{PJATK -- Filia w Gdańsku \textbar\ Informatyka}}
\fancyhead[R]{\small\textcolor{pjatkGray}{Sylabus: WDZ}}
\fancyfoot[C]{\small\thepage}

\titleformat{\section}{\large\bfseries\color{pjatkRed}}{\thesection.}{0.5em}{}
  [\color{pjatkRed}\rule{\linewidth}{0.8pt}]
\setlist{noitemsep, topsep=3pt, parsep=2pt}

\newmdenv[linecolor=pjatkRed, linewidth=1.2pt, backgroundcolor=pjatkLightGray,
  innerleftmargin=10pt, innerrightmargin=10pt, innertopmargin=8pt,
  innerbottommargin=8pt, roundcorner=4pt]{infobox}

\begin{document}

\AddToShipoutPictureBG{%
  \begin{tikzpicture}[remember picture, overlay]
    \node[opacity=0.5] at (current page.center) {%
      \includegraphics[width=14cm]{C:/Users/adamu/WebstormProjects/pj-studies/latex/PJATK_pl_sygnet_transparent-eps-converted-to}%
    };
  \end{tikzpicture}%
}

\begin{center}
  \includegraphics[height=2cm]{C:/Users/adamu/WebstormProjects/pj-studies/latex/PJATK_pl_poziom_1}\\[0.8cm]
  {\LARGE\bfseries\color{pjatkRed} SYLABUS PRZEDMIOTU}\\[0.8cm]
\end{center}

\begin{infobox}
\begin{tabularx}{\textwidth}{@{}lX@{}}
  \textbf{Nazwa przedmiotu:}  & {\bfseries Wstęp do Zarządzania} \\[3pt]
  \textbf{Kod przedmiotu:}    & WDZ \\[3pt]
  \textbf{Kierunek / Profil:} & Informatyka / praktyczny \\[3pt]
  \textbf{Tryb studiów:}      & stacjonarny \\[3pt]
  \textbf{Rok / Semestr:}     & 1 / 1 \\[3pt]
  \textbf{Charakter:}         & obowiązkowy \\[3pt]
  \textbf{Odpowiedzialny:}    & dr. Radosław Stojek \\[3pt]
  \textbf{Wersja z dnia:}     & 20.02.2026 \\
\end{tabularx}
\end{infobox}

\vspace{1cm}

\section{Godziny zajęć i punkty ECTS}

\begin{center}
\begin{tabular}{|>{\centering\arraybackslash}p{2.0cm}
                |>{\centering\arraybackslash}p{2.0cm}
                |>{\centering\arraybackslash}p{2.0cm}
                |>{\centering\arraybackslash}p{2.4cm}
                |>{\centering\arraybackslash}p{2.4cm}
                |>{\centering\arraybackslash}p{2.0cm}
                |>{\centering\arraybackslash}p{1.4cm}|}
\hline
\rowcolor{tableHeader}
\textbf{Wykłady} & \textbf{Ćwiczenia} & \textbf{Laboratorium} &
\textbf{Z prowadzącym} & \textbf{Praca własna} & \textbf{Łącznie} & \textbf{ECTS} \\
\hline
30 h & 30 h & --- & 60 h & 30 h & 90 h & \textbf{3} \\
\hline
\end{tabular}
\end{center}

\section{Forma zajęć}

\begin{tabular}{ll}
  \hline
  \textbf{Forma zajęć} & \textbf{Sposób zaliczenia} \\
  \hline
  Ćwiczenia & Zaliczenie z oceną \\
  \hline
\end{tabular}

\section{Cel dydaktyczny}

Celem przedmiotu jest wprowadzenie studentów kierunku Informatyka w podstawowe pojęcia i praktyki zarządzania organizacją oraz pracą zespołów projektowych, ze szczególnym uwzględnieniem realiów branży IT. Student poznaje rolę menedżera, podstawowe funkcje zarządzania (planowanie, organizowanie, przewodzenie, kontrola), zasady pracy zespołowej, komunikacji i podejmowania decyzji, a także wybrane metody zarządzania projektami i produktami w środowisku wytwarzania oprogramowania. Zajęcia rozwijają umiejętność analizy problemów organizacyjnych, przygotowania prostych planów działań oraz prezentowania wyników pracy w formie projektu.

\section{Treści programowe}

\begin{enumerate}
  \item Wprowadzenie: pojęcie zarządzania, organizacja i jej otoczenie; specyfika branży IT.
  \item Funkcje zarządzania: planowanie, organizowanie, przewodzenie, kontrola – przykłady z projektów informatycznych.
  \item Cele i mierniki: SMART/OKR (wprowadzenie), priorytetyzacja i podejmowanie decyzji.
  \item Struktury organizacyjne i role w IT (zespół wytwórczy, product, project, ops); odpowiedzialność i zakres kompetencji.
  \item Praca zespołowa: role zespołowe, normy współpracy, kontrakt zespołowy, odpowiedzialność zbiorowa.
  \item Komunikacja w zespole: kanały komunikacji, feedback, dokumentowanie ustaleń, spotkania efektywne.
  \item Konflikt i negocjacje: źródła konfliktów, techniki rozwiązywania, eskalacja i deeskalacja.
  \item Motywacja i zaangażowanie: potrzeby, motywatory, środowisko pracy w IT, kultura organizacyjna.
  \item Zarządzanie czasem i zadaniami: plan tygodnia, praca w sprintach (intuicja), ograniczenia WIP (wprowadzenie).
  \item  Wprowadzenie do zarządzania projektami: zakres, czas, koszt, jakość, ryzyko; interesariusze.
  \item  Metodyki zwinne w IT: Agile (wartości i zasady), Scrum/Kanban – role i artefakty (wprowadzenie).
  \item  Ryzyko i jakość: identyfikacja ryzyk, proste rejestry ryzyk, przeglądy jakości i kryteria akceptacji.
  \item  Etyka i odpowiedzialność w zarządzaniu: prywatność, bezpieczeństwo, AI w organizacji (wprowadzenie).
  \item  Warsztat projektowy: analiza przypadku, plan działań, podział ról, przygotowanie prezentacji projektu.
  \item  Prezentacje projektów i podsumowanie: omówienie wniosków, retrospektywa, zalecenia do dalszego rozwoju.
\end{enumerate}

\section{Efekty kształcenia}

\subsection*{Wiedza}
\begin{itemize}
  \item Student definiuje podstawowe pojęcia zarządzania (organizacja, proces, cel, zasoby, interesariusze) oraz opisuje funkcje zarządzania i podstawowe role w zespołach IT. Zna podstawowe podejścia do zarządzania projektami i produktami (m.in. klasyczne i zwinne) oraz rozumie znaczenie komunikacji, motywacji i kontroli jakości w pracy zespołowej.
\end{itemize}

\subsection*{Umiejętności}
\begin{itemize}
  \item Student potrafi przygotować prosty plan działań (cele, zakres, ryzyka, zasoby, harmonogram w zarysie) dla przedsięwzięcia informatycznego. Dobiera podstawowe narzędzia pracy zespołowej (podział zadań, priorytety, przepływ informacji), analizuje problemy organizacyjne i proponuje usprawnienia. Potrafi opracować i zaprezentować wyniki projektu w zespole, uzasadniając podjęte decyzje.
\end{itemize}

\subsection*{Kompetencje społeczne}
\begin{itemize}
  \item Student potrafi współpracować w zespole, przyjmować role i odpowiedzialności, komunikować się w sposób rzeczowy i respektujący różnorodność opinii. Rozumie potrzebę etycznego działania, terminowości i dbałości o jakość w realizacji zadań.
\end{itemize}

\section{Kryteria oceny}

\begin{itemize}
  \item Ćwiczenia:
  \item zadania wykonywane na zajęciach (indywidualne i zespołowe) - 40\%
  \item projekt zespołowy (opracowanie i prezentacja) - 60\%
  \item Warunek zaliczenia: oddanie projektu oraz uzyskanie co najmniej 50\% punktów łącznie.
  \item Wymagana obecność i aktywność: dopuszczalne 2 nieobecności nieusprawiedliwione; pozostałe do odrobienia.
\end{itemize}

\section{Metody dydaktyczne}

Wykład, laboratoria, praca własna studenta.

\section{Literatura}

\textbf{Podstawowa:}
\begin{itemize}
  \item R. Griffin, Podstawy zarządzania organizacjami
  \item H. Kerzner, Project Management: A Systems Approach (wybrane rozdziały)
  \item P. Drucker, Praktyka zarządzania (wybrane rozdziały)
\end{itemize}

\textbf{Uzupełniająca:}
\begin{itemize}
  \item M. C. Lencioni, Pięć dysfunkcji pracy zespołowej
  \item D. Pink, Drive. Kompletnie nowe spojrzenie na motywację
  \item Scrum Guide (aktualna wersja) – materiały referencyjne
\end{itemize}

\end{document}
