% ===========================================================
%  Sylabus: Testowanie bezpieczeństwa systemów IT (TBS)
% ===========================================================
\documentclass[12pt, a4paper]{article}

\usepackage[T1]{fontenc}
\usepackage[utf8]{inputenc}
\usepackage[polish]{babel}
\usepackage{lmodern}
\usepackage{microtype}
\usepackage[a4paper, top=2.5cm, bottom=2.5cm, left=2.5cm, right=2.5cm]{geometry}
\usepackage{xcolor}
\usepackage{graphicx}
\usepackage{booktabs}
\usepackage{tabularx}
\usepackage{longtable}
\usepackage{multirow}
\usepackage{array}
\usepackage{colortbl}
\usepackage{enumitem}
\usepackage{fancyhdr}
\usepackage{titlesec}
\usepackage{mdframed}
\usepackage[colorlinks=true, linkcolor=red!70!black, urlcolor=red!70!black]{hyperref}
\usepackage{eso-pic}
\usepackage{tikz}

\definecolor{pjatkRed}{RGB}{180,0,0}
\definecolor{pjatkGray}{RGB}{80,80,80}
\definecolor{pjatkLightGray}{RGB}{245,245,245}
\definecolor{tableHeader}{RGB}{220,220,220}

\pagestyle{fancy}\fancyhf{}
\renewcommand{\headrulewidth}{0.4pt}
\renewcommand{\footrulewidth}{0.4pt}
\fancyhead[L]{\small\textcolor{pjatkGray}{PJATK -- Filia w Gdańsku \textbar\ Informatyka}}
\fancyhead[R]{\small\textcolor{pjatkGray}{Sylabus: TBS}}
\fancyfoot[C]{\small\thepage}

\titleformat{\section}{\large\bfseries\color{pjatkRed}}{\thesection.}{0.5em}{}
  [\color{pjatkRed}\rule{\linewidth}{0.8pt}]
\setlist{noitemsep, topsep=3pt, parsep=2pt}

\newmdenv[linecolor=pjatkRed, linewidth=1.2pt, backgroundcolor=pjatkLightGray,
  innerleftmargin=10pt, innerrightmargin=10pt, innertopmargin=8pt,
  innerbottommargin=8pt, roundcorner=4pt]{infobox}

\begin{document}

\AddToShipoutPictureBG{%
  \begin{tikzpicture}[remember picture, overlay]
    \node[opacity=0.5] at (current page.center) {%
      \includegraphics[width=14cm]{C:/Users/adamu/WebstormProjects/pj-studies/latex/PJATK_pl_sygnet_transparent-eps-converted-to}%
    };
  \end{tikzpicture}%
}

\begin{center}
  \includegraphics[height=2cm]{C:/Users/adamu/WebstormProjects/pj-studies/latex/PJATK_pl_poziom_1}\\[0.8cm]
  {\LARGE\bfseries\color{pjatkRed} SYLABUS PRZEDMIOTU}\\[0.8cm]
\end{center}

\begin{infobox}
\begin{tabularx}{\textwidth}{@{}lX@{}}
  \textbf{Nazwa przedmiotu:}  & {\bfseries Testowanie bezpieczeństwa systemów IT} \\[3pt]
  \textbf{Kod przedmiotu:}    & TBS \\[3pt]
  \textbf{Kierunek / Profil:} & Informatyka / praktyczny \\[3pt]
  \textbf{Tryb studiów:}      & niestacjonarny \\[3pt]
  \textbf{Rok / Semestr:}     & 4 / 7 \\[3pt]
  \textbf{Charakter:}         & obowiązkowy \\[3pt]
  \textbf{Odpowiedzialny:}    & mgr Adam Kassenberg \\[3pt]
  \textbf{Wersja z dnia:}     & 19.02.2026 \\
\end{tabularx}
\end{infobox}

\vspace{1cm}

\section{Godziny zajęć i punkty ECTS}

\begin{center}
\begin{tabular}{|>{\centering\arraybackslash}p{2.0cm}
                |>{\centering\arraybackslash}p{2.0cm}
                |>{\centering\arraybackslash}p{2.0cm}
                |>{\centering\arraybackslash}p{2.4cm}
                |>{\centering\arraybackslash}p{2.4cm}
                |>{\centering\arraybackslash}p{2.0cm}
                |>{\centering\arraybackslash}p{1.4cm}|}
\hline
\rowcolor{tableHeader}
\textbf{Wykłady} & \textbf{Ćwiczenia} & \textbf{Laboratorium} &
\textbf{Z prowadzącym} & \textbf{Praca własna} & \textbf{Łącznie} & \textbf{ECTS} \\
\hline
30 h & --- & 30 h & 60 h & 65 h & 125 h & \textbf{5} \\
\hline
\end{tabular}
\end{center}

\section{Forma zajęć}

\begin{tabular}{ll}
  \hline
  \textbf{Forma zajęć} & \textbf{Sposób zaliczenia} \\
  \hline
  Laboratorium & Zaliczenie z oceną \\
  Wykład & Egzamin \\
  \hline
\end{tabular}

\section{Cel dydaktyczny}

Celem kursu jest nie tylko nauka teoretycznych podstaw zabezpieczania systemów IT, ale także zdobycie praktycznych umiejętności w identyfikacji i naprawie luk w zabezpieczeniach. Dzięki temu studenci będą lepiej przygotowani do zabezpieczania systemów informatycznych w rzeczywistych scenariuszach.

\section{Treści programowe}

\begin{enumerate}
  \item 1. Wprowadzenie do testowania bezpieczeństwa
  \item Cel i znaczenie testowania bezpieczeństwa.
  \item 2. Metodologia testów penetracyjnych
  \item Definicja testów penetracyjnych.
  \item Różne podejścia i metodologie.
  \item Etapy testów penetracyjnych: rozpoznanie, skanowanie, uzyskiwanie dostępu, utrzymanie dostępu, analiza i raportowanie.
  \item Ćwiczenia praktyczne: Rozpoznanie i skanowanie w rzeczywistych scenariuszach.
  \item 3. Metasploit
  \item Wprowadzenie do narzędzia Metasploit.
  \item Konfiguracja i podstawowe funkcje.
  \item Przykłady wykorzystania Metasploit do testów penetracyjnych.
  \item Ćwiczenia praktyczne: Wykorzystanie Metasploit do eksploatacji luk w zabezpieczeniach.
  \item 4. Bezpieczeństwo aplikacji webowych
  \item Najczęstsze zagrożenia i luki w zabezpieczeniach aplikacji webowych.
  \item Przykłady ataków i ich skutki.
  \item Techniki zabezpieczania aplikacji webowych.
  \item Ćwiczenia praktyczne: Symulacja ataków na aplikacje webowe i implementacja zabezpieczeń.
  \item 5. Kontrola dostępu do funkcji i danych
  \item Definicja i znaczenie kontroli dostępu.
  \item Najlepsze praktyki implementacji.
  \item Przykłady problemów i ich rozwiązania.
  \item Ćwiczenia praktyczne: Implementacja mechanizmów kontroli dostępu w aplikacjach.
  \item 6. SQL Injection
  \item Mechanizmy ataku SQL Injection.
  \item Metody identyfikacji i zabezpieczania przed SQL Injection.
  \item Przykłady ataków i ich skutki.
  \item Ćwiczenia praktyczne: Wykrywanie i naprawa luk typu SQL Injection w aplikacjach.
  \item 7. Cross Site Scripting (XSS)
  \item Typy XSS: reflected, stored, DOM-based.
  \item Techniki wykrywania i zabezpieczania przed XSS.
  \item Przykłady ataków i ich skutki.
  \item Ćwiczenia praktyczne: Symulacja ataków XSS i implementacja zabezpieczeń.
  \item 8. Obsługa danych z niezaufanego źródła Znaczenie walidacji danych.
  \item Techniki bezpiecznego przetwarzania danych.
  \item Przykłady zagrożeń związanych z danymi niezaufanymi.
  \item Ćwiczenia praktyczne: Implementacja walidacji danych w aplikacjach.
  \item 9. Błędy konfiguracji Najczęstsze błędy konfiguracji systemów IT.
  \item Metody identyfikacji i korekty.
  \item Przykłady konsekwencji błędnej konfiguracji.
  \item Ćwiczenia praktyczne: Audyt konfiguracji systemów i implementacja poprawnych ustawień.
  \item 10. Testowanie typu Black Box
  \item Definicja i cel testowania typu Black Box.
  \item Techniki i narzędzia używane w testach Black Box.
  \item Przykłady scenariuszy testowych.
  \item Ćwiczenia praktyczne: Przeprowadzanie testów Black Box na rzeczywistych systemach.
\end{enumerate}

\section{Efekty kształcenia}

\subsection*{Wiedza}
\begin{itemize}
  \item Student zna i rozumie metodologię i narzędzia używane do identyfikacji i eksploatacji luk w zabezpieczeniach (testy penetracyjne). Student rozumie działanie  ataków SQL Injection oraz zna metody ich wykrywania i zapobiegania. Student zna i rozumie sposoby wykrywania i zabezpieczania przed atakami XSS.
\end{itemize}

\subsection*{Umiejętności}
\begin{itemize}
  \item Student potrafi korzystać z narzędzia Metasploit do testowania systemów. Student potrafi implementować i testować mechanizmy kontroli dostępu.  Student potrafi przeprowadzić testy BlackBox na rzeczywistych systemach.
\end{itemize}

\section{Kryteria oceny}

\begin{itemize}
  \item Studium przypadków
  \item Zadania problemowe
  \item Kryteria oceny
  \item Kolokwium pisemne.
  \item Skala ocen:
  \item Poniżej 50\% - ndst
  \item Od 50\% - dst
  \item Od 60\% - dst+
  \item Od 70\% - db
  \item Od 80\% - db+
  \item Od 90\% - bdb
  \item Skala ocen:
  \item Poniżej 50\% - ndst
  \item Od 50\% - dst
  \item Od 60\% - dst+
  \item Od 70\% - db
  \item Od 80\% - db+
  \item Od 90\% - bdb
\end{itemize}

\section{Metody dydaktyczne}

Wykład, laboratoria, praca własna studenta.

\section{Literatura}

\textbf{Podstawowa:}
\begin{itemize}
  \item "Metasploit: The Penetration Tester's Guide, David Kennedy, Jim O'Gorman, Justin Brown i Peter Kim
  \item "The Web Application Hacker's Handbook: Finding and Exploiting Security Flaws", Dafydd Stuttard, Marcus Pinto
  \item "Hacking: The Art of Exploitation", Jon Erickson
  \item "Black Hat Python: Python Programming for Hackers and Pentesters", Justin Seitz
\end{itemize}

\textbf{Uzupełniająca:}
\begin{itemize}
  \item "SQL Injection Attacks and Defense", Justin Clarke
\end{itemize}

\end{document}
