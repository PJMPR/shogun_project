% ===========================================================
%  Sylabus: Algebra liniowa i geometria (ALG)
% ===========================================================
\documentclass[12pt, a4paper]{article}

\usepackage[T1]{fontenc}
\usepackage[utf8]{inputenc}
\usepackage[polish]{babel}
\usepackage{lmodern}
\usepackage{microtype}
\usepackage[a4paper, top=2.5cm, bottom=2.5cm, left=2.5cm, right=2.5cm]{geometry}
\usepackage{xcolor}
\usepackage{graphicx}
\usepackage{booktabs}
\usepackage{tabularx}
\usepackage{longtable}
\usepackage{multirow}
\usepackage{array}
\usepackage{colortbl}
\usepackage{enumitem}
\usepackage{fancyhdr}
\usepackage{titlesec}
\usepackage{mdframed}
\usepackage[colorlinks=true, linkcolor=red!70!black, urlcolor=red!70!black]{hyperref}
\usepackage{eso-pic}
\usepackage{tikz}

\definecolor{pjatkRed}{RGB}{180,0,0}
\definecolor{pjatkGray}{RGB}{80,80,80}
\definecolor{pjatkLightGray}{RGB}{245,245,245}
\definecolor{tableHeader}{RGB}{220,220,220}

\pagestyle{fancy}\fancyhf{}
\renewcommand{\headrulewidth}{0.4pt}
\renewcommand{\footrulewidth}{0.4pt}
\fancyhead[L]{\small\textcolor{pjatkGray}{PJATK -- Filia w Gdańsku \textbar\ Informatyka}}
\fancyhead[R]{\small\textcolor{pjatkGray}{Sylabus: ALG}}
\fancyfoot[C]{\small\thepage}

\titleformat{\section}{\large\bfseries\color{pjatkRed}}{\thesection.}{0.5em}{}
  [\color{pjatkRed}\rule{\linewidth}{0.8pt}]
\setlist{noitemsep, topsep=3pt, parsep=2pt}

\newmdenv[linecolor=pjatkRed, linewidth=1.2pt, backgroundcolor=pjatkLightGray,
  innerleftmargin=10pt, innerrightmargin=10pt, innertopmargin=8pt,
  innerbottommargin=8pt, roundcorner=4pt]{infobox}

\begin{document}

\AddToShipoutPictureBG{%
  \begin{tikzpicture}[remember picture, overlay]
    \node[opacity=0.5] at (current page.center) {%
      \includegraphics[width=14cm]{C:/Users/adamu/WebstormProjects/pj-studies/latex/PJATK_pl_sygnet_transparent-eps-converted-to}%
    };
  \end{tikzpicture}%
}

\begin{center}
  \includegraphics[height=2cm]{C:/Users/adamu/WebstormProjects/pj-studies/latex/PJATK_pl_poziom_1}\\[0.8cm]
  {\LARGE\bfseries\color{pjatkRed} SYLABUS PRZEDMIOTU}\\[0.8cm]
\end{center}

\begin{infobox}
\begin{tabularx}{\textwidth}{@{}lX@{}}
  \textbf{Nazwa przedmiotu:}  & {\bfseries Algebra liniowa i geometria} \\[3pt]
  \textbf{Kod przedmiotu:}    & ALG \\[3pt]
  \textbf{Kierunek / Profil:} & Informatyka / praktyczny \\[3pt]
  \textbf{Tryb studiów:}      & niestacjonarny \\[3pt]
  \textbf{Rok / Semestr:}     & 1 / 2 \\[3pt]
  \textbf{Charakter:}         & obowiązkowy \\[3pt]
  \textbf{Odpowiedzialny:}    & dr Elżbieta Puźniakowska-Gałuch, ela@pejot.edu.pl \\[3pt]
  \textbf{Wersja z dnia:}     & 19.02.2026 \\
\end{tabularx}
\end{infobox}

\vspace{1cm}

\section{Godziny zajęć i punkty ECTS}

\begin{center}
\begin{tabular}{|>{\centering\arraybackslash}p{2.0cm}
                |>{\centering\arraybackslash}p{2.0cm}
                |>{\centering\arraybackslash}p{2.0cm}
                |>{\centering\arraybackslash}p{2.4cm}
                |>{\centering\arraybackslash}p{2.4cm}
                |>{\centering\arraybackslash}p{2.0cm}
                |>{\centering\arraybackslash}p{1.4cm}|}
\hline
\rowcolor{tableHeader}
\textbf{Wykłady} & \textbf{Ćwiczenia} & \textbf{Laboratorium} &
\textbf{Z prowadzącym} & \textbf{Praca własna} & \textbf{Łącznie} & \textbf{ECTS} \\
\hline
16 h & 16 h & --- & 32 h & 93 h & 125 h & \textbf{5} \\
\hline
\end{tabular}
\end{center}

\section{Forma zajęć}

\begin{tabular}{ll}
  \hline
  \textbf{Forma zajęć} & \textbf{Sposób zaliczenia} \\
  \hline
  Ćwiczenia & Zaliczenie z oceną \\
  Wykład & Egzamin pisemny \\
  \hline
\end{tabular}

\section{Cel dydaktyczny}

Celem kształcenia jest nabycie umiejętności posługiwania się aparatem teorii mnogości, liczb zespolonych, pierścieni wielomianów; formułowania problemów w terminach macierzy i wykonywania operacji na macierzach; rozwiązywania układu równań liniowych; znajomości podstawowych struktur algebraicznych. Omawiane są również podstawowe zagadnienia geometrii analitycznej na płaszczyźnie i w przestrzeni trójwymiarowej.

\section{Przedmioty wprowadzające}

\begin{tabularx}{\textwidth}{lX}
  \hline
  \textbf{Przedmiot} & \textbf{Wymagane zagadnienia} \\
  \hline
  matematyka z zakresu szkoły średniej & --- \\
  \hline
\end{tabularx}

\section{Treści programowe}

\begin{enumerate}
  \item Liczby zespolone. Interpretacja geometryczna. Postać kartezjańska.  Postać trygonometryczna. Potęgi i pierwiastki liczb zespolonych.
  \item Działania na wielomianach. Mnożenie , dzielenie wielomianów.  Pierwiastki wielomianów rzeczywistych i zespolonych (wzory Cardana). Funkcje wymierne, ułamki proste (rzeczywiste i zespolone).
  \item Macierze. Podstawowe operacje na macierzach.  Macierze odwrotne (metoda bezwyznacznikowa). Macierze przekształceń liniowych.
  \item Wyznaczniki (różne definicje).  Rozwinięcie Laplace’a oraz inne metody obliczania wyznaczników dowolnych stopni.  Macierze odwrotne (za pomocą macierzy dopełnień algebraicznych).
  \item Układy równań liniowych. Metoda eliminacji Gaussa. Wzory Cramera.
  \item Geometria analityczna w 2D.  Wektory w 3D. Iloczyn skalarny, wektorowy i mieszany.  Prosta i płaszczyzna w przestrzeni 3D.
\end{enumerate}

\section{Efekty kształcenia}

\subsection*{Wiedza}
\begin{itemize}
  \item Student zna i rozumie  jak zastosować pojęcia algebry i geometrii analitycznej  do problemów informatycznych.
  \item Student zna i rozumie pojęcie zbioru liczb zespolonych.Student zna i rozumie pojęcie wielomianu, funkcji wymiernej i ułamków prostych w zbiorze liczb rzeczywistych i zespolonych.Student zna i rozumie pojęcie macierzy rzeczywistej i zespolonej  oraz ich klasyfikację. Zna i rozumie sposób wykonywania działań na macierzach  i ich zastosowanie do przekształceń liniowych oraz rozwiązywania układów równań liniowych.
  \item Student zna i rozumie metody rozwiązywania układów równań liniowych rzeczywistych i zespolonych.Student zna i rozumie analityczne sposoby opisywania obiektów w przestrzeni 2D i 3D.
\end{itemize}

\subsection*{Umiejętności}
\begin{itemize}
  \item Student potrafi wykonywać operacje na liczbach zespolonych w postaci algebraicznej i trygonometrycznej.
  \item Student potrafi wykonywać działania na wielomianach oraz funkcjach wymiernych rzeczywistych i zespolonych.
  \item Student potrafi wykonywać działania na macierzach rzeczywistych i zespolonych. Potrafi obliczać macierz odwrotną różnymi metodami. Student potrafi obliczyć macierz przekształcenia liniowego oraz składać przekształcenia liniowe. Student potrafi obliczyć wyznacznik macierzy kwadratowej i go zastosować do różnych zagadnień.Student potrafi rozwiązywać układy równań liniowych kilkoma metodami. Student potrafi  posługiwać się aparatem geometrii analitycznej 2D i 3D. Potrafi obliczyć iloczyn skalarny, wektorowy i mieszany. Student potrafi wykorzystać te pojęcia do obiektów w przestrzeni 3D. Student potrafi używać różnych postaci równań prostych, płaszczyzn oraz obliczyć rzuty prostokątne i ukośne.
\end{itemize}

\subsection*{Kompetencje społeczne}
\begin{itemize}
  \item Student jest gotów do samodzielnego pozyskiwania informacji z różnych źródeł i przełożenia ich na potrzebny kontekst.
\end{itemize}

\section{Kryteria oceny}

\begin{itemize}
  \item Rozwiązywanie zadań na tablicy
  \item rozwiązywanie zadań na tablicy
  \item dykusja
  \item Kryteria oceny
  \item Zaliczenie ćwiczeń na podstawie dwóch kolokwiów (do 10 zadań). Student jest zobowiązany uzyskać wynik powyżej 50\% możliwych do zdobycia sumarycznie z obu kolokwiów.
  \item Skala ocen:
  \item Poniżej 50\% - ndst
  \item Od 50\% - dst
  \item Od 60\% - dst+
  \item Od 70\% - db
  \item Od 80\% - db+
  \item Od 90\% - bdb
  \item Poniżej 50\% - ndst
  \item Od 50\% - dst
  \item Od 60\% - dst+
  \item Od 70\% - db
  \item Od 80\% - db+
  \item Od 90\% - bdb
  \item Przed podejściem do egzaminu student musi zaliczyć część ćwiczeniową.
\end{itemize}

\section{Metody dydaktyczne}

Wykład, laboratoria, praca własna studenta.

\section{Literatura}

\textbf{Podstawowa:}
\begin{itemize}
  \item 1. T. Jurlewicz, Z. Skoczylas, " Algebra i geometria analityczna. Definicje, twierdzenia, wzory.", wydanie z 2021 roku i nowsze.2. T. Jurlewicz, Z. Skoczylas, "Algebra liniowa 2. Definicje, twierdzenia, wzory.", wydanie z 2005 roku i nowsze.
  \item 3. T. Jurlewicz, Z. Skoczylas, "Algebra i geometria analityczna. Przykłady i zadania", wydanie z 2021 roku i nowsze.4. T. Jurlewicz, Z. Skoczylas, "Algebra liniowa 2. Przykłady i zadania", wydanie z 2001 roku i nowsze.
\end{itemize}

\textbf{Uzupełniająca:}
\begin{itemize}
  \item 1. Przemysław Kajetanowicz, Jędrzej Wierzejewski, Algebra z geometrią analityczną Wydawnictwo PWN, 2008, ISBN 978-83-01-15493-6 2008.
  \item 2. Henryk Arodź, Krzysztof Rościszewski, Algebra i geometria analityczna w zadaniach Wydawnictwo Znak, 2005, ISBN 83-240-0547-1 2005.
  \item 3. Aleksiej I. Kostrikin, Zbiór zadań z algebry Wydawnictwo PWN, 2005, ISBN 83-01-14539-0
\end{itemize}

\end{document}
