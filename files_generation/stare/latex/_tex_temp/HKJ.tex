% ===========================================================
%  Sylabus: Historia i Kultura Japonii (HKJ)
% ===========================================================
\documentclass[12pt, a4paper]{article}

\usepackage[T1]{fontenc}
\usepackage[utf8]{inputenc}
\usepackage[polish]{babel}
\usepackage{lmodern}
\usepackage{microtype}
\usepackage[a4paper, top=2.5cm, bottom=2.5cm, left=2.5cm, right=2.5cm]{geometry}
\usepackage{xcolor}
\usepackage{graphicx}
\usepackage{booktabs}
\usepackage{tabularx}
\usepackage{longtable}
\usepackage{multirow}
\usepackage{array}
\usepackage{colortbl}
\usepackage{enumitem}
\usepackage{fancyhdr}
\usepackage{titlesec}
\usepackage{mdframed}
\usepackage[colorlinks=true, linkcolor=red!70!black, urlcolor=red!70!black]{hyperref}
\usepackage{eso-pic}
\usepackage{tikz}

\definecolor{pjatkRed}{RGB}{180,0,0}
\definecolor{pjatkGray}{RGB}{80,80,80}
\definecolor{pjatkLightGray}{RGB}{245,245,245}
\definecolor{tableHeader}{RGB}{220,220,220}

\pagestyle{fancy}\fancyhf{}
\renewcommand{\headrulewidth}{0.4pt}
\renewcommand{\footrulewidth}{0.4pt}
\fancyhead[L]{\small\textcolor{pjatkGray}{PJATK -- Filia w Gdańsku \textbar\ Informatyka}}
\fancyhead[R]{\small\textcolor{pjatkGray}{Sylabus: HKJ}}
\fancyfoot[C]{\small\thepage}

\titleformat{\section}{\large\bfseries\color{pjatkRed}}{\thesection.}{0.5em}{}
  [\color{pjatkRed}\rule{\linewidth}{0.8pt}]
\setlist{noitemsep, topsep=3pt, parsep=2pt}

\newmdenv[linecolor=pjatkRed, linewidth=1.2pt, backgroundcolor=pjatkLightGray,
  innerleftmargin=10pt, innerrightmargin=10pt, innertopmargin=8pt,
  innerbottommargin=8pt, roundcorner=4pt]{infobox}

\begin{document}

\AddToShipoutPictureBG{%
  \begin{tikzpicture}[remember picture, overlay]
    \node[opacity=0.5] at (current page.center) {%
      \includegraphics[width=14cm]{C:/Users/adamu/WebstormProjects/pj-studies/latex/PJATK_pl_sygnet_transparent-eps-converted-to}%
    };
  \end{tikzpicture}%
}

\begin{center}
  \includegraphics[height=2cm]{C:/Users/adamu/WebstormProjects/pj-studies/latex/PJATK_pl_poziom_1}\\[0.8cm]
  {\LARGE\bfseries\color{pjatkRed} SYLABUS PRZEDMIOTU}\\[0.8cm]
\end{center}

\begin{infobox}
\begin{tabularx}{\textwidth}{@{}lX@{}}
  \textbf{Nazwa przedmiotu:}  & {\bfseries Historia i Kultura Japonii} \\[3pt]
  \textbf{Kod przedmiotu:}    & HKJ \\[3pt]
  \textbf{Kierunek / Profil:} & Informatyka / praktyczny \\[3pt]
  \textbf{Tryb studiów:}      & niestacjonarny \\[3pt]
  \textbf{Rok / Semestr:}     & 1 / 1 \\[3pt]
  \textbf{Charakter:}         & obowiązkowy \\[3pt]
  \textbf{Odpowiedzialny:}    &  \\[3pt]
  \textbf{Wersja z dnia:}     & 19.02.2026 \\
\end{tabularx}
\end{infobox}

\vspace{1cm}

\section{Godziny zajęć i punkty ECTS}

\begin{center}
\begin{tabular}{|>{\centering\arraybackslash}p{2.0cm}
                |>{\centering\arraybackslash}p{2.0cm}
                |>{\centering\arraybackslash}p{2.0cm}
                |>{\centering\arraybackslash}p{2.4cm}
                |>{\centering\arraybackslash}p{2.4cm}
                |>{\centering\arraybackslash}p{2.0cm}
                |>{\centering\arraybackslash}p{1.4cm}|}
\hline
\rowcolor{tableHeader}
\textbf{Wykłady} & \textbf{Ćwiczenia} & \textbf{Laboratorium} &
\textbf{Z prowadzącym} & \textbf{Praca własna} & \textbf{Łącznie} & \textbf{ECTS} \\
\hline
30 h & --- & --- & 30 h & 20 h & 50 h & \textbf{2} \\
\hline
\end{tabular}
\end{center}

\section{Forma zajęć}

\begin{tabular}{ll}
  \hline
  \textbf{Forma zajęć} & \textbf{Sposób zaliczenia} \\
  \hline
  Wykład & Zaliczenie z oceną \\
  \hline
\end{tabular}

\section{Cel dydaktyczny}

Zapoznanie studentów z historią i kulturą Japonii, z najważniejszymi osiągnięciami w dziedzinie sztuki i estetyki. Zaznajomienie z tradycjami, zwyczajami, normami zachowań społecznych, problemami współczesnego społeczeństwa japońskiego oraz współczesną i dawną architekturą.

\section{Treści programowe}

\begin{enumerate}
  \item Kultura i etykieta biznesu w Japonii (1) - Jak zdobyć zaufanie Japończyka. Dobre pierwsze wrażenie. Przedstawianie się i wymiana wizytówek. Nawiązywanie i podtrzymywanie kontaktu. Podstawowa japońska etykieta biznesu. Japońska gospodarka. Rynek pracy w Japonii.
  \item Kultura i etykieta biznesu w Japonii (2) - Charakterystyka pracy w Japonii. Kobiety w japońskich firmach. Przedsiębiorstwa japońskie (kaizen). Wartości, symbole, postawy w japońskich firmach. Przebieg spotkania biznesowego. Negocjacje i podejmowanie decyzji.
  \item Japonia w relacjach zagranicznych - Firmy zagraniczne w Japonii. Efektywny biznes w Japonii. Różnice kulturowe pomiędzy Japonią, Chinami i Koreą. Lokalizacja Japonii na świecie i elementy kultury japońskiej.
  \item Geneza i historia animacji i mangi - Wczesny rozwój animacji i mangi. Ważne dzieła i ich wpływ kulturowy. Popularność i rozpowszechnianie w Japonii i za granicą.
  \item Gatunki i cechy charakterystyczne mangi i anime - Gatunki shōnen, shōjo, seinen. Projekty postaci i style ekspresji. Przemysłowy i ekonomiczny wpływ anime i mangi.
  \item Wielkość i rozwój rynku animacji i mangi - Działalność związana z licencjonowaniem i rozwojem produktów. Wpływ gospodarczy animacji i mangi.
  \item Estetyka japońska - Percepcja natury. Pojęcie piękna i kształtowanie się estetyki japońskiej. Cechy kluczowe estetyki japońskiej: sugestia, nieregularność, prostota, nietrwałość. Pojęcia estetyczne: wabi-sabi, miyabi, aware, iki, yūgen.
  \item Architektura japońska - Miasta i prowincji, współcześni architekci. Przejście od architektury tradycyjnej do nowoczesnej. Architektura współczesna: powojenny metabolism, postmodernizm.
  \item Koleje japońskie - Ewolucja i estetyka dworców kolejowych w Japonii. Japońskie pociągi.
  \item Estetyka infrastruktury w Japonii - Architekt i inżynier w Japonii. Historia estetyki infrastruktury. Zasady estetyki kolei.
  \item Przestrzeń publiczna w Japonii - Przykłady kolei na estakadach, placów dworcowych, sztuka publiczna.
  \item Kultura otaku - Powstanie i charakterystyka. Znaczenie cosplay i doujin. Rola wydarzeń i społeczności - globalny wpływ.
  \item Popularność i wpływ anime i mangi za granicą - Interakcja z międzynarodowymi fanami. Akceptacja kulturowa.
  \item Wprowadzenie do historii japońskiej.
  \item Podsumowanie wszystkich zagadnień, kolokwium końcowe.
\end{enumerate}

\section{Efekty kształcenia}

\subsection*{Wiedza}
\begin{itemize}
  \item Student zna i rozumie podstawowe fakty z historii i kultury Japonii.
  \item Student zna najważniejsze pojęcia estetyki japońskiej.
\end{itemize}

\subsection*{Umiejętności}
\begin{itemize}
  \item Student potrafi pozyskiwać specjalistyczne informacje z literatury na temat kultury i historii Japonii.
  \item Student potrafi dokonywać oceny, krytycznej analizy i syntezy tych informacji, a także wyciągać wnioski.
  \item Student potrafi formułować i uzasadniać opinie na tematy poruszane na wykładzie.
\end{itemize}

\subsection*{Kompetencje społeczne}
\begin{itemize}
  \item Student jest gotów do uczenia się przez całe życie w kontekście nowej wiedzy dotyczącej kultury Japonii.
\end{itemize}

\section{Kryteria oceny}

\begin{itemize}
  \item Kolokwium końcowe pisemne.
  \item Poniżej 50\% - ndst
  \item Od 50\% - dst
  \item Od 60\% - dst+
  \item Od 70\% - db
  \item Od 80\% - db+
  \item Od 90\% - bdb
\end{itemize}

\section{Metody dydaktyczne}

Wykład, laboratoria, praca własna studenta.

\section{Literatura}

\textbf{Podstawowa:}
\begin{itemize}
  \item Marszałek-Kawa, J., Gawłowski, R. (red.) (2013) Kulturowe i edukacyjne oblicza współczesnej Azji, Wydawnictwo Adam Marszałek, Toruń.
  \item Nishigake, H., Dillon, T. (2009) An illustrated English-Japanese Dictionary of Japanese life, Shogakukan, Tokyo.
  \item Takei, N. (2001) Getting closer to Japan. Japanese culture, ASK., Ltd., Tokyo.
\end{itemize}

\textbf{Uzupełniająca:}
\begin{itemize}
  \item Doi, T. (1973) The Anatomy of Dependence, Kodansha International, Tokyo.
  \item Keene, D. (1971) Appreciations of Japanese culture, Kodansha International, Tokyo.
  \item Kubiak Ho-Chi, B. (2009) Estetyka i sztuka japońska, Universitas, Kraków.
  \item Mukai, K. (2008) A guidebook to Japan and its customs, Nihon Bungeisha, Tokyo.
  \item Richie, D. (2007) A tractate on Japanese aesthetics, Stone Bridges Press, Berkley.
  \item Young, D., Young, M. (2019) The art of Japanese architecture, Tuttle Publishing, Tokyo.
  \item Lynch, K. (1960) The Image of the City, MIT Press, Cambridge MA.
  \item Meeks, C.L.V. (1995) The Railroad Station: An Architectural History, Dover Publications, New York.
  \item Mizuno, K. (2017) Subway stations around the world, Seigensha, Tokyo.
\end{itemize}

\end{document}
