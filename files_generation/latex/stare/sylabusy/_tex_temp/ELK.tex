% ===========================================================
%  Sylabus: Elektronika (ELK)
% ===========================================================
\documentclass[12pt, a4paper]{article}

\usepackage[T1]{fontenc}
\usepackage[utf8]{inputenc}
\usepackage[polish]{babel}
\usepackage{lmodern}
\usepackage{microtype}
\usepackage[a4paper, top=2.5cm, bottom=2.5cm, left=2.5cm, right=2.5cm]{geometry}
\usepackage{xcolor}
\usepackage{graphicx}
\usepackage{booktabs}
\usepackage{tabularx}
\usepackage{longtable}
\usepackage{multirow}
\usepackage{array}
\usepackage{colortbl}
\usepackage{enumitem}
\usepackage{fancyhdr}
\usepackage{titlesec}
\usepackage{mdframed}
\usepackage[colorlinks=true, linkcolor=red!70!black, urlcolor=red!70!black]{hyperref}
\usepackage{eso-pic}
\usepackage{tikz}

\definecolor{pjatkRed}{RGB}{180,0,0}
\definecolor{pjatkGray}{RGB}{80,80,80}
\definecolor{pjatkLightGray}{RGB}{245,245,245}
\definecolor{tableHeader}{RGB}{220,220,220}

\pagestyle{fancy}\fancyhf{}
\renewcommand{\headrulewidth}{0.4pt}
\renewcommand{\footrulewidth}{0.4pt}
\fancyhead[L]{\small\textcolor{pjatkGray}{PJATK -- Filia w Gdańsku \textbar\ Informatyka}}
\fancyhead[R]{\small\textcolor{pjatkGray}{Sylabus: ELK}}
\fancyfoot[C]{\small\thepage}

\titleformat{\section}{\large\bfseries\color{pjatkRed}}{\thesection.}{0.5em}{}
  [\color{pjatkRed}\rule{\linewidth}{0.8pt}]
\setlist{noitemsep, topsep=3pt, parsep=2pt}

\newmdenv[linecolor=pjatkRed, linewidth=1.2pt, backgroundcolor=pjatkLightGray,
  innerleftmargin=10pt, innerrightmargin=10pt, innertopmargin=8pt,
  innerbottommargin=8pt, roundcorner=4pt]{infobox}

\begin{document}

\AddToShipoutPictureBG{%
  \begin{tikzpicture}[remember picture, overlay]
    \node[opacity=0.5] at (current page.center) {%
      \includegraphics[width=14cm]{C:/Users/adamu/WebstormProjects/pj-studies/latex/PJATK_pl_sygnet_transparent-eps-converted-to}%
    };
  \end{tikzpicture}%
}

\begin{center}
  \includegraphics[height=2cm]{C:/Users/adamu/WebstormProjects/pj-studies/latex/PJATK_pl_poziom_1}\\[0.8cm]
  {\LARGE\bfseries\color{pjatkRed} SYLABUS PRZEDMIOTU}\\[0.8cm]
\end{center}

\begin{infobox}
\begin{tabularx}{\textwidth}{@{}lX@{}}
  \textbf{Nazwa przedmiotu:}  & {\bfseries Elektronika} \\[3pt]
  \textbf{Kod przedmiotu:}    & ELK \\[3pt]
  \textbf{Kierunek / Profil:} & Informatyka / praktyczny \\[3pt]
  \textbf{Tryb studiów:}      & niestacjonarny \\[3pt]
  \textbf{Rok / Semestr:}     & 2 / 4 \\[3pt]
  \textbf{Charakter:}         & obowiązkowy \\[3pt]
  \textbf{Odpowiedzialny:}    & dr Tadeusz Puźniakowski \\[3pt]
  \textbf{Wersja z dnia:}     & 19.02.2026 \\
\end{tabularx}
\end{infobox}

\vspace{1cm}

\section{Godziny zajęć i punkty ECTS}

\begin{center}
\begin{tabular}{|>{\centering\arraybackslash}p{2.0cm}
                |>{\centering\arraybackslash}p{2.0cm}
                |>{\centering\arraybackslash}p{2.0cm}
                |>{\centering\arraybackslash}p{2.4cm}
                |>{\centering\arraybackslash}p{2.4cm}
                |>{\centering\arraybackslash}p{2.0cm}
                |>{\centering\arraybackslash}p{1.4cm}|}
\hline
\rowcolor{tableHeader}
\textbf{Wykłady} & \textbf{Ćwiczenia} & \textbf{Laboratorium} &
\textbf{Z prowadzącym} & \textbf{Praca własna} & \textbf{Łącznie} & \textbf{ECTS} \\
\hline
16 h & --- & 16 h & 32 h & 68 h & 100 h & \textbf{4} \\
\hline
\end{tabular}
\end{center}

\section{Forma zajęć}

\begin{tabular}{ll}
  \hline
  \textbf{Forma zajęć} & \textbf{Sposób zaliczenia} \\
  \hline
  Wykład & Nieoceniany \\
  \hline
\end{tabular}

\section{Cel dydaktyczny}

Celem przedmiotu jest wyposażenie studentów w podstawową wiedzę z zakresu elektroniki, elektrotechniki i technik pomiarowych. Studenci poznają zasady działania oraz zastosowania elementów i układów elektronicznych, takich jak rezystory, kondensatory, cewki, diody, tranzystory i bramki logiczne. Nauczą się również podstawowych technik pomiarowych i diagnostycznych z użyciem multimetru, oscyloskopu i innych narzędzi pomiarowych.

\section{Przedmioty wprowadzające}

\begin{tabularx}{\textwidth}{lX}
  \hline
  \textbf{Przedmiot} & \textbf{Wymagane zagadnienia} \\
  \hline
  [BHP] Szkolenie BHP & Znajomość regulaminu i zasad BHP obowiązujących w laboratorium \\
  \hline
\end{tabularx}

\section{Treści programowe}

\begin{enumerate}
  \item Zrozumienie podstawowych pojęć z zakresu elektroniki – Studenci poznają podstawowe prawa i zjawiska elektryczne, takie jak prąd, napięcie, rezystancja oraz zasady działania prostych obwodów elektrycznych.
  \item Poznanie i zastosowanie elementów elektronicznych – Celem jest nauka rozpoznawania, interpretowania oraz  stosowania różnych elementów elektronicznych w praktycznych układach; Nabycie umiejętności projektowania i analizy prostych obwodów – Studenci nauczą się projektować i analizować układy elektroniczne, sekwencyjne i kombinacyjne uwzględniając prawa Kirchhoffa, dzielniki napięcia oraz  zastosowanie wzmacniaczy i innych elementów półprzewodnikowych.
  \item Rozwój umiejętności posługiwania się przyrządami pomiarowymi – Praktyczne ćwiczenia pozwolą na  opanowanie obsługi urządzeń takich jak multimetr i oscyloskop, co jest niezbędne do diagnozowania i  analizy działania układów.; Rozwijanie umiejętności montażu układów – Celem jest nauczenie studentów technik lutowania, poprawnego montażu elementów elektronicznych oraz praktycznego doboru metod  i urządzeń, co jest kluczowe dla realizacji projektów elektronicznych.
\end{enumerate}

\section{Efekty kształcenia}

\subsection*{Wiedza}
\begin{itemize}
  \item Student zna i rozumie pojęcia z zakresu fizyki, obejmującą dziedziny przydatne dla studiów na kierunku informatyka, w  tym elementy mechaniki klasycznej, podstawy elektryczności i magnetyzmu.
  \item Student zna i rozumie pojęcia w zakresie elektrotechniki, elektroniki i miernictwa; rozumie powiązania informatyki z tymi obszarami i możliwość przenoszenia dobrych praktyk wypracowanych w tych obszarach na grunt informatyki.
\end{itemize}

\subsection*{Umiejętności}
\begin{itemize}
  \item Student potrafi zaplanować i dobrać właściwe metody i  urządzenia do przeprowadzenia eksperymentu w postaci pomiaru lub symulacji komputerowej, w celu weryfikacji działania oraz identyfikacji parametrów i właściwości systemu, z zachowaniem zasad BHP.
\end{itemize}

\subsection*{Kompetencje społeczne}
\begin{itemize}
  \item ----------
\end{itemize}

\section{Kryteria oceny}

\begin{itemize}
  \item wykład z elementami dyskusji
  \item z prezentacją multimedialną
  \item Ćwiczenia / Laboratorium/Lektorat:
  \item rozwiązywanie zadań projektowych
  \item analiza przypadków
  \item badania symulacyjne
  \item Ćwiczenia/Laboratorium
  \item Kryteria oceny
  \item Sprawozdania z przeprowadzonych ćwiczeń
  \item Bez dostarczenia sprawozdań ze wszystkich zajęć laboratoryjnych, które przewidują taką formę pracy, nie ma możliwości uzyskania oceny pozytywnej z zajęć laboratoryjnych.
\end{itemize}

\section{Metody dydaktyczne}

Wykład, laboratoria, praca własna studenta.

\section{Literatura}

\textbf{Podstawowa:}
\begin{itemize}
  \item Charles Platt; Elektronika. Od praktyki do teorii. Wydanie III; Hellion, 2022
\end{itemize}

\textbf{Uzupełniająca:}
\begin{itemize}
  \item Dickon Ross, Cathleen Shamieh, Gordon McComb; Electronics For Dummies; John Wiley \& Sons, Ltd., 2010
  \item Kimmo Karvinen and Tero Karvinen; Getting Started with Sensors; Maker Media, Inc., 2014
  \item Bill Pretty; Getting Started with Electronic Projects; Packt Publishing Ltd., 2015
\end{itemize}

\end{document}
