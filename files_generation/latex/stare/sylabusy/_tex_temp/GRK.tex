% ===========================================================
%  Sylabus: Grafika Komputerowa (GRK)
% ===========================================================
\documentclass[12pt, a4paper]{article}

\usepackage[T1]{fontenc}
\usepackage[utf8]{inputenc}
\usepackage[polish]{babel}
\usepackage{lmodern}
\usepackage{microtype}
\usepackage[a4paper, top=2.5cm, bottom=2.5cm, left=2.5cm, right=2.5cm]{geometry}
\usepackage{xcolor}
\usepackage{graphicx}
\usepackage{booktabs}
\usepackage{tabularx}
\usepackage{longtable}
\usepackage{multirow}
\usepackage{array}
\usepackage{colortbl}
\usepackage{enumitem}
\usepackage{fancyhdr}
\usepackage{titlesec}
\usepackage{mdframed}
\usepackage[colorlinks=true, linkcolor=red!70!black, urlcolor=red!70!black]{hyperref}
\usepackage{eso-pic}
\usepackage{tikz}

\definecolor{pjatkRed}{RGB}{180,0,0}
\definecolor{pjatkGray}{RGB}{80,80,80}
\definecolor{pjatkLightGray}{RGB}{245,245,245}
\definecolor{tableHeader}{RGB}{220,220,220}

\pagestyle{fancy}\fancyhf{}
\renewcommand{\headrulewidth}{0.4pt}
\renewcommand{\footrulewidth}{0.4pt}
\fancyhead[L]{\small\textcolor{pjatkGray}{PJATK -- Filia w Gdańsku \textbar\ Informatyka}}
\fancyhead[R]{\small\textcolor{pjatkGray}{Sylabus: GRK}}
\fancyfoot[C]{\small\thepage}

\titleformat{\section}{\large\bfseries\color{pjatkRed}}{\thesection.}{0.5em}{}
  [\color{pjatkRed}\rule{\linewidth}{0.8pt}]
\setlist{noitemsep, topsep=3pt, parsep=2pt}

\newmdenv[linecolor=pjatkRed, linewidth=1.2pt, backgroundcolor=pjatkLightGray,
  innerleftmargin=10pt, innerrightmargin=10pt, innertopmargin=8pt,
  innerbottommargin=8pt, roundcorner=4pt]{infobox}

\begin{document}

\AddToShipoutPictureBG{%
  \begin{tikzpicture}[remember picture, overlay]
    \node[opacity=0.5] at (current page.center) {%
      \includegraphics[width=14cm]{C:/Users/adamu/WebstormProjects/pj-studies/latex/PJATK_pl_sygnet_transparent-eps-converted-to}%
    };
  \end{tikzpicture}%
}

\begin{center}
  \includegraphics[height=2cm]{C:/Users/adamu/WebstormProjects/pj-studies/latex/PJATK_pl_poziom_1}\\[0.8cm]
  {\LARGE\bfseries\color{pjatkRed} SYLABUS PRZEDMIOTU}\\[0.8cm]
\end{center}

\begin{infobox}
\begin{tabularx}{\textwidth}{@{}lX@{}}
  \textbf{Nazwa przedmiotu:}  & {\bfseries Grafika Komputerowa} \\[3pt]
  \textbf{Kod przedmiotu:}    & GRK \\[3pt]
  \textbf{Kierunek / Profil:} & Informatyka / praktyczny \\[3pt]
  \textbf{Tryb studiów:}      & niestacjonarny \\[3pt]
  \textbf{Rok / Semestr:}     & 2 / 4 \\[3pt]
  \textbf{Charakter:}         & obowiązkowy \\[3pt]
  \textbf{Odpowiedzialny:}    &  \\[3pt]
  \textbf{Wersja z dnia:}     & 19.02.2026 \\
\end{tabularx}
\end{infobox}

\vspace{1cm}

\section{Godziny zajęć i punkty ECTS}

\begin{center}
\begin{tabular}{|>{\centering\arraybackslash}p{2.0cm}
                |>{\centering\arraybackslash}p{2.0cm}
                |>{\centering\arraybackslash}p{2.0cm}
                |>{\centering\arraybackslash}p{2.4cm}
                |>{\centering\arraybackslash}p{2.4cm}
                |>{\centering\arraybackslash}p{2.0cm}
                |>{\centering\arraybackslash}p{1.4cm}|}
\hline
\rowcolor{tableHeader}
\textbf{Wykłady} & \textbf{Ćwiczenia} & \textbf{Laboratorium} &
\textbf{Z prowadzącym} & \textbf{Praca własna} & \textbf{Łącznie} & \textbf{ECTS} \\
\hline
16 h & --- & 16 h & 32 h & 93 h & 125 h & \textbf{5} \\
\hline
\end{tabular}
\end{center}

\section{Forma zajęć}

\begin{tabular}{ll}
  \hline
  \textbf{Forma zajęć} & \textbf{Sposób zaliczenia} \\
  \hline
  Laboratorium & Zaliczenie z oceną \\
  Wykład & Egzamin \\
  \hline
\end{tabular}

\section{Cel dydaktyczny}

Celem kształcenia jest nabycie umiejętności tworzenia obrazów z wykorzystaniem standardowego API graficznego (biblioteki OpenGL i renderera Blender) oraz realizacji podstawowych transformacji obrazów 2- i 3-wymiarowych.

\section{Przedmioty wprowadzające}

\begin{tabularx}{\textwidth}{lX}
  \hline
  \textbf{Przedmiot} & \textbf{Wymagane zagadnienia} \\
  \hline
  • Algebra liniowa i geometria  (ALG) & • Analiza matematyczna  (AM) \\
  • Algorytmy i struktury danych (ASD) & • programowanie w C/C++ \\
  • operacje na macierzach & • przekształcenia liniowe \\
  \hline
\end{tabularx}

\section{Treści programowe}

\begin{enumerate}
  \item Wykład:
  \item Percepcja wizualna i modele barw
  \item Algorytmy rastrowe
  \item Formaty plików graficznych i kompresja obrazów
  \item Biblioteka LibPNG. Wizualizacja danych
  \item Biblioteka Jpeglib.
  \item Geometria Liniowa 3W
  \item Geometria Afiniczna 3W.
  \item Rzutowanie
  \item Podstawy oświetlenia
  \item Podstawy teksturowania
  \item Podstawy OpenGL i GLSL
  \item Transformacje w OpenGL
  \item Podstawy teksturowania w OpenGL,
  \item Podstawy Krzywych i powierzchni Béziera
  \item Metoda śledzenia promieni
  \item Ćwiczenia:
  \item Podstawy Blendera
  \item Podstawy edycji masha.
  \item Materiały i tekstury
  \item Ustawienia środowiska
  \item Podstawy Animacji. System cząstek
  \item Postprocessing
  \item Format svg.
  \item Algorytmy rastrowe.
  \item Filtry punktowe
  \item Geometryczne przekształcenie obrazów
  \item Podstawy OpenGL
  \item Przekształcenia w OpenGL
  \item Teksturowanie w OpenGL
  \item Oświetlenie w OpenGL
  \item Modelowanie za pomocą krzywych i powierzchni Béziera
\end{enumerate}

\section{Efekty kształcenia}

\subsection*{Wiedza}
\begin{itemize}
  \item Zna i rozumie pojęcia z zakresu matematyki niezbędne do modelowania i przetwarzania barw oraz obiektów 2D, 3D. Zna i rozumie pojęcia z zakresu kluczowych zagadnień i metod w zakresie grafiki, multimediów i komunikacji człowiek-komputer
\end{itemize}

\subsection*{Umiejętności}
\begin{itemize}
  \item Student potrafi zastosować aparat matematyczny do interpretowania pojęć z zakresu informatyki oraz rozwiązywania problemów o charakterze informatycznym
  \item Student potrafi czytać ze zrozumieniem proste programy celem ich weryfikacji, a także ich pisania i uruchamiania
  \item Student potrafi wyspecyfikować,  zaprojektować, zaimplementować, przetestować oraz zdebuggować program; potrafi korzystać z bibliotek, środowisk programistycznych, integrujących i uruchomieniowych.
  \item Student potrafi operować w oknie aplikacji obrazem dwu- i trójwymiarowym (generacja i przetwarzanie) za pomocą standardowego API graficznego oraz stworzenia graficzny interfejs użytkownika, używając właściwych metod i narzędzi, a także przeprowadzić testy użyteczności aplikacji
\end{itemize}

\subsection*{Kompetencje społeczne}
\begin{itemize}
  \item Student jest jest gotów do samodzielnego uczenia się przez całe życie
\end{itemize}

\section{Kryteria oceny}

\begin{itemize}
  \item Ćwiczenia / Laboratorium/Lektorat:
  \item rozwiązywanie zadań
  \item Kryteria oceny
  \item Zaliczenie ćwiczeń polega na zbieraniu punktów:
  \item 50\% możliwych punktów daje ocenę 3
  \item 60\% punktów daje ocenę 3½
  \item 70\% — ocenę 4
  \item 80\% — 4½
  \item 90\% i więcej — 5
  \item Test wyboru, 20 pytań. Każda poprawna odpowiedź warta jest 1 punkt. Zasady zaliczenia: 19–20 punktów: 5, 17–18 punktów: 4 12 , 14–16 punktów: 4, 11–13 punktów: 3 12 , 8–10 punktów: 3. Mniej niż 8 punktów: 2
\end{itemize}

\section{Metody dydaktyczne}

Wykład, laboratoria, praca własna studenta.

\section{Literatura}

\textbf{Podstawowa:}
\begin{itemize}
  \item M. Jankowski: Elementy grafiki komputerowej. WNT 2006.
  \item S. R. Buss: 3D Computer Graphics: A Mathematical Introduction with OpenGL.Revision draft Draft A.10.b. May 28, 2019.
  \item Przemyslaw Kiciak: Podstawy modelowania krzywych i powierzchni, Wydawnictwo Naukowe PWN, 2019
  \item Przemyslaw Kiciak: OpenGL i GLSL, Część III Gandalf.com.pl, Wydawnictwo Naukowe PWN, 2019
  \item Graham Sellers, Richard S. Wright Jr., Nicholas Haemel, OpenGL. Księga eksperta. Wydanie VII Helion 2016
\end{itemize}

\textbf{Uzupełniająca:}
\begin{itemize}
  \item Jim Chronister: Blender Basics Classroom Tutorial Book. 4th edition, 2011
  \item Wojciech Mokrzycki: Wprowadzenie do przetwarzania informacji wizualnej, tom I. Percepcja, akwizycja, wizualizacja EXIT2010978-83-60434-76-5
  \item David Austin: What is JPEG? Notices of the AMS, Volume 55, Number 2, pages 226–229, 2008.
\end{itemize}

\end{document}
