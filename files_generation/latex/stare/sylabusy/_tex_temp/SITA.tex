% ===========================================================
%  Sylabus: Projektowanie bezpiecznych architektur (SITA)
% ===========================================================
\documentclass[12pt, a4paper]{article}

\usepackage[T1]{fontenc}
\usepackage[utf8]{inputenc}
\usepackage[polish]{babel}
\usepackage{lmodern}
\usepackage{microtype}
\usepackage[a4paper, top=2.5cm, bottom=2.5cm, left=2.5cm, right=2.5cm]{geometry}
\usepackage{xcolor}
\usepackage{graphicx}
\usepackage{booktabs}
\usepackage{tabularx}
\usepackage{longtable}
\usepackage{multirow}
\usepackage{array}
\usepackage{colortbl}
\usepackage{enumitem}
\usepackage{fancyhdr}
\usepackage{titlesec}
\usepackage{mdframed}
\usepackage[colorlinks=true, linkcolor=red!70!black, urlcolor=red!70!black]{hyperref}
\usepackage{eso-pic}
\usepackage{tikz}

\definecolor{pjatkRed}{RGB}{180,0,0}
\definecolor{pjatkGray}{RGB}{80,80,80}
\definecolor{pjatkLightGray}{RGB}{245,245,245}
\definecolor{tableHeader}{RGB}{220,220,220}

\pagestyle{fancy}\fancyhf{}
\renewcommand{\headrulewidth}{0.4pt}
\renewcommand{\footrulewidth}{0.4pt}
\fancyhead[L]{\small\textcolor{pjatkGray}{PJATK -- Filia w Gdańsku \textbar\ Informatyka}}
\fancyhead[R]{\small\textcolor{pjatkGray}{Sylabus: SITA}}
\fancyfoot[C]{\small\thepage}

\titleformat{\section}{\large\bfseries\color{pjatkRed}}{\thesection.}{0.5em}{}
  [\color{pjatkRed}\rule{\linewidth}{0.8pt}]
\setlist{noitemsep, topsep=3pt, parsep=2pt}

\newmdenv[linecolor=pjatkRed, linewidth=1.2pt, backgroundcolor=pjatkLightGray,
  innerleftmargin=10pt, innerrightmargin=10pt, innertopmargin=8pt,
  innerbottommargin=8pt, roundcorner=4pt]{infobox}

\begin{document}

\AddToShipoutPictureBG{%
  \begin{tikzpicture}[remember picture, overlay]
    \node[opacity=0.5] at (current page.center) {%
      \includegraphics[width=14cm]{C:/Users/adamu/WebstormProjects/pj-studies/latex/PJATK_pl_sygnet_transparent-eps-converted-to}%
    };
  \end{tikzpicture}%
}

\begin{center}
  \includegraphics[height=2cm]{C:/Users/adamu/WebstormProjects/pj-studies/latex/PJATK_pl_poziom_1}\\[0.8cm]
  {\LARGE\bfseries\color{pjatkRed} SYLABUS PRZEDMIOTU}\\[0.8cm]
\end{center}

\begin{infobox}
\begin{tabularx}{\textwidth}{@{}lX@{}}
  \textbf{Nazwa przedmiotu:}  & {\bfseries Projektowanie bezpiecznych architektur} \\[3pt]
  \textbf{Kod przedmiotu:}    & SITA \\[3pt]
  \textbf{Kierunek / Profil:} & Informatyka / praktyczny \\[3pt]
  \textbf{Tryb studiów:}      & stacjonarny \\[3pt]
  \textbf{Rok / Semestr:}     & 4 / 7 \\[3pt]
  \textbf{Charakter:}         & obowiązkowy \\[3pt]
  \textbf{Odpowiedzialny:}    & mgr Adam Kassenberg \\[3pt]
  \textbf{Wersja z dnia:}     & 19.02.2026 \\
\end{tabularx}
\end{infobox}

\vspace{1cm}

\section{Godziny zajęć i punkty ECTS}

\begin{center}
\begin{tabular}{|>{\centering\arraybackslash}p{2.0cm}
                |>{\centering\arraybackslash}p{2.0cm}
                |>{\centering\arraybackslash}p{2.0cm}
                |>{\centering\arraybackslash}p{2.4cm}
                |>{\centering\arraybackslash}p{2.4cm}
                |>{\centering\arraybackslash}p{2.0cm}
                |>{\centering\arraybackslash}p{1.4cm}|}
\hline
\rowcolor{tableHeader}
\textbf{Wykłady} & \textbf{Ćwiczenia} & \textbf{Laboratorium} &
\textbf{Z prowadzącym} & \textbf{Praca własna} & \textbf{Łącznie} & \textbf{ECTS} \\
\hline
30 h & 30 h & --- & 60 h & 65 h & 125 h & \textbf{5} \\
\hline
\end{tabular}
\end{center}

\section{Forma zajęć}

\begin{tabular}{ll}
  \hline
  \textbf{Forma zajęć} & \textbf{Sposób zaliczenia} \\
  \hline
  Laboratorium & Zaliczenie z oceną \\
  Wykład & Egzamin \\
  \hline
\end{tabular}

\section{Cel dydaktyczny}

Tenkursmanaceludostarczeniestudentomkompleksowejwiedzyiumiejętnościniezbędnychdoprojektowaniaiutrzymaniabezpiecznychsystemówinformatycznych.Dziękipołączeniuteoretycznychwykładówipraktycznychćwiczeń,studencibędąlepiejprzygotowanidostawieniaczoławyzwaniomzwiązanymzbezpieczeństwemITw rzeczywistym świecie. Studencibędąpoznawaćpodstawowepojęciaizasadybezpieczeństwainformatycznego,takiejakpoufność,integralność idostępność danych. Nauczą się,jakprojektowaćsystemyi infrastruktury informatyczne, które są odporne na różnorodne zagrożenia i ataki. Celem kursu będzie również zrozumienie mechanizmów zarządzania tożsamością i kontrolądostępu,wtymwdrażaniepolitykzabezpieczeń. Kurs obejmuje techniki i najlepsze praktyki bezpiecznego programowania, aby zapobiegać wprowadzaniu luk bezpieczeństwa w fazie rozwoju oprogramowania. Studencibędąuczyćsię,jakreagowaćnaincydentybezpieczeństwa,wtymjakprzeprowadzać analizy i raportować naruszenia.

\section{Treści programowe}

\begin{enumerate}
  \item 1. Wybór organizacji
  \item Analiza różnych typów organizacji.
  \item Przykłady typowych struktur organizacyjnych.
  \item 2. Definicja celów biznesowych
  \item Jak określić cele biznesowe organizacji.
  \item Rola celów biznesowych w projektowaniu architektur IT.
  \item Ćwiczenia praktyczne: Definiowanie celów biznesowych dla wybranej organizacji.
  \item 3. Utworzenie diagramu przepływów DFD
  \item Wprowadzenie do diagramów przepływów danych (DFD).
  \item Tworzenie DFD dla wybranej organizacji.
  \item Ćwiczenia praktyczne: Rysowanie DFD na podstawie zebranych danych.
  \item 4. Procesowy opis organizacji wg ISO 9001
  \item Wprowadzenie do standardu ISO 9001.
  \item Opis procesów organizacyjnych zgodnie z ISO 9001.
  \item Ćwiczenia praktyczne: Tworzenie procesowego opisu organizacji.
  \item 5. Utworzenie architektury IT
  \item Zasady projektowania architektur IT.
  \item Elementy składowe architektury IT.
  \item Ćwiczenia praktyczne: Projektowanie architektury IT dla wybranej organizacji.
  \item 6. Koncepcja komunikacji, sprzętu oraz oprogramowania
  \item Planowanie komunikacji w organizacji.
  \item Wybór sprzętu i oprogramowania.
  \item Ćwiczenia praktyczne: Opracowanie koncepcji komunikacji, sprzętu i oprogramowania.
  \item 7. Klasyfikacja informacji, struktura zarządzania
  \item Klasyfikacja informacji w organizacji.
  \item Tworzenie struktury zarządzania informacją.
  \item Ćwiczenia praktyczne: Klasyfikacja informacji i tworzenie struktury zarządzania.
  \item 8. Identyfikacja zasobów organizacji
  \item Identyfikacja i katalogowanie zasobów.
  \item Zarządzanie zasobami organizacji.
  \item Ćwiczenia praktyczne: Tworzenie katalogu zasobów organizacji.
  \item 9. Utworzenie deklaracji stosowania
  \item Definicja i znaczenie deklaracji stosowania.
  \item Tworzenie deklaracji dla wybranej organizacji.
  \item 10. Identyfikacja zagrożeń dla architektury
  \item Typowe zagrożenia dla architektur IT.
  \item Metody identyfikacji zagrożeń.
  \item Ćwiczenia praktyczne: Analiza przypadków i identyfikacja zagrożeń.
  \item 11. Identyfikacja podatności dla zagrożeń
  \item Definicja podatności.
  \item Metody identyfikacji podatności w systemach IT.
  \item Ćwiczenia praktyczne: Identyfikacja podatności w wybranej architekturze IT.
  \item 12. Szacowanie ryzyka dla wyodrębnionych zagrożeń
  \item Metody szacowania ryzyka.
  \item Przykłady zastosowania metod szacowania ryzyka.
  \item Ćwiczenia praktyczne: Szacowanie ryzyka dla wybranych zagrożeń.
  \item 13. Tworzenie rekomendacji po analizie ryzyka
  \item Jak tworzyć rekomendacje bezpieczeństwa.
  \item Przykłady rekomendacji po analizie ryzyka.
  \item Ćwiczenia praktyczne: Tworzenie rekomendacji dla wybranej architektury IT.
\end{enumerate}

\section{Efekty kształcenia}

\subsection*{Wiedza}
\begin{itemize}
  \item Student zna i rozumie pojęcia takie jak poufność, integralność, dostępność danych. Student zna i rozumie zasady tworzenia bezpiecznych systemów infrastruktury informatycznej. Student zna i rozumie najlepsze praktyki kodowania, które minimalizują ryzyko wprowadzenia luk w bezpieczeństwie. Student zna i rozumie techniki zabezpieczania danych takie jak szyfrowanie, backup i monitoring. Student zna i rozumie międzynarodowe standardy bezpieczeństwa IT, takie jak ISO/IEC 27001 czy GDPR.
\end{itemize}

\subsection*{Umiejętności}
\begin{itemize}
  \item Student potrafi identyfikować zagrożenia i oceniać ryzyko związane z różnymi systemami IT. Student potrafi zaimplementować mechanizmy kontroli dostępu i zarządzania tożsamością. Student potrafi zarządzać incydentami cyberbezpieczeństwa, analizować naruszenia i raportować.
\end{itemize}

\section{Kryteria oceny}

\begin{itemize}
  \item Studium przypadków
  \item Kryteria oceny
  \item Kolokwium pisemne.
  \item Skala ocen:
  \item Poniżej 50\% - ndst
  \item Od 50\% - dst
  \item Od 60\% - dst+
  \item Od 70\% - db
  \item Od 80\% - db+
  \item Od 90\% - bdb
  \item Skala ocen:
  \item Poniżej 50\% - ndst
  \item Od 50\% - dst
  \item Od 60\% - dst+
  \item Od 70\% - db
  \item Od 80\% - db+
  \item Od 90\% - bdb
\end{itemize}

\section{Metody dydaktyczne}

Wykład, laboratoria, praca własna studenta.

\section{Literatura}

\textbf{Podstawowa:}
\begin{itemize}
  \item "Security Engineering: A Guide to Building Dependable Distributed Systems”, Ross Anderson
  \item "Threat Modeling: Designing for Security", Adam Shostack
  \item "Cross Site Scripting (XSS) Attacks: Understanding and Preventing XSS Attacks", Robert Hansen
\end{itemize}

\textbf{Uzupełniająca:}
\begin{itemize}
  \item "Black Hat Python: Python Programming for Hackers and Pentesters" autorstwa Justin Seitz
  \item "SQL Injection Attacks and Defense" autorstwa Justin Clarke
\end{itemize}

\end{document}
