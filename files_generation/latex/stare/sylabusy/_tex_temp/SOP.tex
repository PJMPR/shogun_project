% ===========================================================
%  Sylabus: Systemy Operacyjne (SOP)
% ===========================================================
\documentclass[12pt, a4paper]{article}

\usepackage[T1]{fontenc}
\usepackage[utf8]{inputenc}
\usepackage[polish]{babel}
\usepackage{lmodern}
\usepackage{microtype}
\usepackage[a4paper, top=2.5cm, bottom=2.5cm, left=2.5cm, right=2.5cm]{geometry}
\usepackage{xcolor}
\usepackage{graphicx}
\usepackage{booktabs}
\usepackage{tabularx}
\usepackage{longtable}
\usepackage{multirow}
\usepackage{array}
\usepackage{colortbl}
\usepackage{enumitem}
\usepackage{fancyhdr}
\usepackage{titlesec}
\usepackage{mdframed}
\usepackage[colorlinks=true, linkcolor=red!70!black, urlcolor=red!70!black]{hyperref}
\usepackage{eso-pic}
\usepackage{tikz}

\definecolor{pjatkRed}{RGB}{180,0,0}
\definecolor{pjatkGray}{RGB}{80,80,80}
\definecolor{pjatkLightGray}{RGB}{245,245,245}
\definecolor{tableHeader}{RGB}{220,220,220}

\pagestyle{fancy}\fancyhf{}
\renewcommand{\headrulewidth}{0.4pt}
\renewcommand{\footrulewidth}{0.4pt}
\fancyhead[L]{\small\textcolor{pjatkGray}{PJATK -- Filia w Gdańsku \textbar\ Informatyka}}
\fancyhead[R]{\small\textcolor{pjatkGray}{Sylabus: SOP}}
\fancyfoot[C]{\small\thepage}

\titleformat{\section}{\large\bfseries\color{pjatkRed}}{\thesection.}{0.5em}{}
  [\color{pjatkRed}\rule{\linewidth}{0.8pt}]
\setlist{noitemsep, topsep=3pt, parsep=2pt}

\newmdenv[linecolor=pjatkRed, linewidth=1.2pt, backgroundcolor=pjatkLightGray,
  innerleftmargin=10pt, innerrightmargin=10pt, innertopmargin=8pt,
  innerbottommargin=8pt, roundcorner=4pt]{infobox}

\begin{document}

\AddToShipoutPictureBG{%
  \begin{tikzpicture}[remember picture, overlay]
    \node[opacity=0.5] at (current page.center) {%
      \includegraphics[width=14cm]{C:/Users/adamu/WebstormProjects/pj-studies/latex/PJATK_pl_sygnet_transparent-eps-converted-to}%
    };
  \end{tikzpicture}%
}

\begin{center}
  \includegraphics[height=2cm]{C:/Users/adamu/WebstormProjects/pj-studies/latex/PJATK_pl_poziom_1}\\[0.8cm]
  {\LARGE\bfseries\color{pjatkRed} SYLABUS PRZEDMIOTU}\\[0.8cm]
\end{center}

\begin{infobox}
\begin{tabularx}{\textwidth}{@{}lX@{}}
  \textbf{Nazwa przedmiotu:}  & {\bfseries Systemy Operacyjne} \\[3pt]
  \textbf{Kod przedmiotu:}    & SOP \\[3pt]
  \textbf{Kierunek / Profil:} & Informatyka / praktyczny \\[3pt]
  \textbf{Tryb studiów:}      & niestacjonarny \\[3pt]
  \textbf{Rok / Semestr:}     & 2 / 3 \\[3pt]
  \textbf{Charakter:}         & obowiązkowy \\[3pt]
  \textbf{Odpowiedzialny:}    & Mgr inż. Michał Hyla \\[3pt]
  \textbf{Wersja z dnia:}     & 19.02.2026 \\
\end{tabularx}
\end{infobox}

\vspace{1cm}

\section{Godziny zajęć i punkty ECTS}

\begin{center}
\begin{tabular}{|>{\centering\arraybackslash}p{2.0cm}
                |>{\centering\arraybackslash}p{2.0cm}
                |>{\centering\arraybackslash}p{2.0cm}
                |>{\centering\arraybackslash}p{2.4cm}
                |>{\centering\arraybackslash}p{2.4cm}
                |>{\centering\arraybackslash}p{2.0cm}
                |>{\centering\arraybackslash}p{1.4cm}|}
\hline
\rowcolor{tableHeader}
\textbf{Wykłady} & \textbf{Ćwiczenia} & \textbf{Laboratorium} &
\textbf{Z prowadzącym} & \textbf{Praca własna} & \textbf{Łącznie} & \textbf{ECTS} \\
\hline
16 h & --- & 16 h & 32 h & 93 h & 125 h & \textbf{5} \\
\hline
\end{tabular}
\end{center}

\section{Forma zajęć}

\begin{tabular}{ll}
  \hline
  \textbf{Forma zajęć} & \textbf{Sposób zaliczenia} \\
  \hline
  Laboratorium & Zaliczenie z oceną \\
  Wykład & Egzamin \\
  \hline
\end{tabular}

\section{Cel dydaktyczny}

Celem dydaktycznym przedmiotu SOP (Systemy Operacyjne i Programowanie) jest zapoznanie studentów z zaawansowanymi mechanizmami systemów operacyjnych, szczególnie w kontekście zarządzania procesami, pamięcią oraz synchronizacją w środowiskach wielowątkowych i wieloprocesorowych. Kurs kładzie nacisk na praktyczne aspekty pracy z systemem Linux, obejmujące m.in. zarządzanie zasobami, wątki, przerwania, semafory i mutexy.

\section{Przedmioty wprowadzające}

\begin{tabularx}{\textwidth}{lX}
  \hline
  \textbf{Przedmiot} & \textbf{Wymagane zagadnienia} \\
  \hline
  UKOS & Umiejętność korzystania z systemu linux, znajomość podstaw działania systemu operacyjnego \\
  \hline
\end{tabularx}

\section{Treści programowe}

\begin{enumerate}
  \item Podstawy systemów operacyjnychWprowadzenie do kluczowych zagadnień związanych z systemami operacyjnymi, takich jak procesy, zarządzanie pamięcią, systemy plików, mechanizmy wejścia-wyjścia. Studenci zdobywają wiedzę na temat organizacji i współbieżności systemów operacyjnych, ze szczególnym naciskiem na powszechnie stosowane systemy, takie jak Linux.
  \item Wątki i zarządzanie procesamiOmówienie sposobów zarządzania procesami i wątkami w systemach operacyjnych. Studenci uczą się, jak działają wątki, procesy, oraz jakie mechanizmy synchronizacji są stosowane w celu unikania problemów takich jak deadlock czy starvation. Przykłady obejmują implementację mutexów i semaforów.
  \item Synchronizacja i rozwiązywanie problemów sekcji krytycznychZajęcia poświęcone rozwiązywaniu problemów synchronizacji i współbieżności, takich jak problem producenta-konsumenta oraz problem czytelników i pisarzy. Studenci uczą się wykorzystywać mechanizmy synchronizacji (mutexy, semafory) w systemach operacyjnych.
  \item Algorytmy szeregowania zadańOmówienie algorytmów szeregowania zadań, takich jak FCFS, SJF, RR, oraz ich zastosowanie w rzeczywistych systemach operacyjnych. Studenci nauczą się dobierać odpowiedni algorytm do specyfiki aplikacji.
  \item Instalacja i konfiguracja systemu operacyjnegoPraktyczne ćwiczenia z instalacji systemu operacyjnego (np. Linux), w tym instalacja komponentów takich jak baza danych i aplikacje współpracujące. Zajęcia obejmują konfigurację systemu oraz jego administrację.
  \item Zarządzanie pamięciąPrzegląd metod zarządzania pamięcią w systemach operacyjnych, takich jak stronicowanie, segmentacja, oraz zarządzanie pamięcią wirtualną. Studenci uczą się, jak systemy operacyjne efektywnie przydzielają zasoby pamięciowe procesom.
  \item Projektowanie prostych systemów operacyjnychNa zajęciach studenci realizują projekt polegający na zaprojektowaniu i uruchomieniu prostego systemu operacyjnego z elementami, takimi jak zarządzanie procesami, szeregowanie zadań oraz podstawowa obsługa plików.
\end{enumerate}

\section{Efekty kształcenia}

\subsection*{Wiedza}
\begin{itemize}
  \item Student zna i rozumie pojęcia z zakresu kluczowych zagadnień dotyczących systemów operacyjnych – zasady ich działania, konstrukcji, organizacji współbieżności; zna i rozumie powszechnie stosowane systemy.
\end{itemize}

\subsection*{Umiejętności}
\begin{itemize}
  \item Studentpotrafi dobrać system operacyjny i wykorzystywać oferowane przezeń funkcje i możliwości do rozwiązywania klasycznych problemów synchronizacji; potrafi dobrać algorytm szeregowania zadań do specyfiki aplikacji jak też zainstalować i skonfigurować typowy system operacyjny oraz nim administrować
  \item Studentpotrafi zaplanować i przeprowadzić proces instalacji i uruchomienia całości prostego systemu (system operacyjny, baza danych, aplikacja, oprogramowanie współdziałające)
\end{itemize}

\section{Kryteria oceny}

\begin{itemize}
  \item Programowanie na żywo
  \item Warsztaty
  \item Rozwiązywanie zadań
  \item Projekt zespołowy
  \item Kryteria oceny
  \item Laboratorium: Kolokwium, Ocena projektu;
  \item Wykład: Egzamin pisemny
  \item Skala ocen dotycząca laboratoriów oraz wykładu:
  \item Poniżej 50\% - ndst
  \item Od 50\% - dst
  \item Od 60\% - dst+
  \item Od 70\% - db
  \item Od 80\% - db+
  \item Od 90\% - bdb
\end{itemize}

\section{Metody dydaktyczne}

Wykład, laboratoria, praca własna studenta.

\section{Literatura}

\textbf{Podstawowa:}
\begin{itemize}
  \item Tanenbaum AS, Bos H. Systemy operacyjne. V ed. Gliwice: Helion; 2023.
  \item Stallings W. Systemy operacyjne: architektura, funkcjonowanie i projektowanie. IX ed. Gliwice: Helion; 2023.
\end{itemize}

\textbf{Uzupełniająca:}
\begin{itemize}
  \item Negus C. Linux. Biblia. X ed. Gliwice: Helion; 2023.
\end{itemize}

\end{document}
