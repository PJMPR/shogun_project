% ===========================================================
%  Sylabus: Programowanie platform sprzętowych (PPS)
% ===========================================================
\documentclass[12pt, a4paper]{article}

\usepackage[T1]{fontenc}
\usepackage[utf8]{inputenc}
\usepackage[polish]{babel}
\usepackage{lmodern}
\usepackage{microtype}
\usepackage[a4paper, top=2.5cm, bottom=2.5cm, left=2.5cm, right=2.5cm]{geometry}
\usepackage{xcolor}
\usepackage{graphicx}
\usepackage{booktabs}
\usepackage{tabularx}
\usepackage{longtable}
\usepackage{multirow}
\usepackage{array}
\usepackage{colortbl}
\usepackage{enumitem}
\usepackage{fancyhdr}
\usepackage{titlesec}
\usepackage{mdframed}
\usepackage[colorlinks=true, linkcolor=red!70!black, urlcolor=red!70!black]{hyperref}
\usepackage{eso-pic}
\usepackage{tikz}

\definecolor{pjatkRed}{RGB}{180,0,0}
\definecolor{pjatkGray}{RGB}{80,80,80}
\definecolor{pjatkLightGray}{RGB}{245,245,245}
\definecolor{tableHeader}{RGB}{220,220,220}

\pagestyle{fancy}\fancyhf{}
\renewcommand{\headrulewidth}{0.4pt}
\renewcommand{\footrulewidth}{0.4pt}
\fancyhead[L]{\small\textcolor{pjatkGray}{PJATK -- Filia w Gdańsku \textbar\ Informatyka}}
\fancyhead[R]{\small\textcolor{pjatkGray}{Sylabus: PPS}}
\fancyfoot[C]{\small\thepage}

\titleformat{\section}{\large\bfseries\color{pjatkRed}}{\thesection.}{0.5em}{}
  [\color{pjatkRed}\rule{\linewidth}{0.8pt}]
\setlist{noitemsep, topsep=3pt, parsep=2pt}

\newmdenv[linecolor=pjatkRed, linewidth=1.2pt, backgroundcolor=pjatkLightGray,
  innerleftmargin=10pt, innerrightmargin=10pt, innertopmargin=8pt,
  innerbottommargin=8pt, roundcorner=4pt]{infobox}

\begin{document}

\AddToShipoutPictureBG{%
  \begin{tikzpicture}[remember picture, overlay]
    \node[opacity=0.5] at (current page.center) {%
      \includegraphics[width=14cm]{C:/Users/adamu/WebstormProjects/pj-studies/latex/PJATK_pl_sygnet_transparent-eps-converted-to}%
    };
  \end{tikzpicture}%
}

\begin{center}
  \includegraphics[height=2cm]{C:/Users/adamu/WebstormProjects/pj-studies/latex/PJATK_pl_poziom_1}\\[0.8cm]
  {\LARGE\bfseries\color{pjatkRed} SYLABUS PRZEDMIOTU}\\[0.8cm]
\end{center}

\begin{infobox}
\begin{tabularx}{\textwidth}{@{}lX@{}}
  \textbf{Nazwa przedmiotu:}  & {\bfseries Programowanie platform sprzętowych} \\[3pt]
  \textbf{Kod przedmiotu:}    & PPS \\[3pt]
  \textbf{Kierunek / Profil:} & Informatyka / praktyczny \\[3pt]
  \textbf{Tryb studiów:}      & niestacjonarny \\[3pt]
  \textbf{Rok / Semestr:}     & 4 / 8 \\[3pt]
  \textbf{Charakter:}         & obowiązkowy \\[3pt]
  \textbf{Odpowiedzialny:}    & dr Adam Muc \\[3pt]
  \textbf{Wersja z dnia:}     & 19.02.2026 \\
\end{tabularx}
\end{infobox}

\vspace{1cm}

\section{Godziny zajęć i punkty ECTS}

\begin{center}
\begin{tabular}{|>{\centering\arraybackslash}p{2.0cm}
                |>{\centering\arraybackslash}p{2.0cm}
                |>{\centering\arraybackslash}p{2.0cm}
                |>{\centering\arraybackslash}p{2.4cm}
                |>{\centering\arraybackslash}p{2.4cm}
                |>{\centering\arraybackslash}p{2.0cm}
                |>{\centering\arraybackslash}p{1.4cm}|}
\hline
\rowcolor{tableHeader}
\textbf{Wykłady} & \textbf{Ćwiczenia} & \textbf{Laboratorium} &
\textbf{Z prowadzącym} & \textbf{Praca własna} & \textbf{Łącznie} & \textbf{ECTS} \\
\hline
30 h & --- & 30 h & 60 h & 65 h & 125 h & \textbf{5} \\
\hline
\end{tabular}
\end{center}

\section{Forma zajęć}

\begin{tabular}{ll}
  \hline
  \textbf{Forma zajęć} & \textbf{Sposób zaliczenia} \\
  \hline
  Laboratorium & Zaliczenie z oceną \\
  Wykład & Egzamin \\
  \hline
\end{tabular}

\section{Cel dydaktyczny}

Kurs ma na celu wykształcenie umiejętności projektowania, programowania i optymalizacji różnorodnych platform sprzętowych, co jest nieocenione w dzisiejszej technologicznej rzeczywistości. Wprowadzenie do typowych platform, na których można rozpoczynać projekty z IoT. Są one także wykorzystywane w gotowych produktach.

\section{Treści programowe}

\begin{enumerate}
  \item Programowanie platform cyfrowych i peryferiów
  \item Arduino: Arduion IDE + gcc-avr
  \item Raspberry Pi: C++, Python, NodeJS (2 do wyboru)
  \item MSP 430
  \item PLC
  \item Weryfikacja wybranych algorytmów dla potrzeb IoT w środowisku Ptolemy
  \item Atiny
\end{enumerate}

\section{Efekty kształcenia}

\subsection*{Wiedza}
\begin{itemize}
  \item Student zna i rozumie funkcje, biblioteki i podstawową składnię Arduino.  Student zna i rozumie jak działa kompilator AVR dla mikrokontrolerów Arduino. Student ma znajomość języków programowania takich jak Ladder Logic, Structured Text. Student zna i rozumie niskopoziomowe programowanie oraz zarządzanie energią w MSP430. Student zna i rozumie funkcje niskopoziomowe oraz wie jak zoptymalizować kod dla mikrokontrolerów o ograniczonych zasobach (Atiny).
\end{itemize}

\subsection*{Umiejętności}
\begin{itemize}
  \item Student potrafi programować sekwencje zdarzeń w sterownikach programowalnych PLC. Student potrafi wykorzystywać C++/Python do szybkiego prototypowania oraz NodeJS do tworzenia aplikacji sieciowych. Student potrafi wykorzystać języki programowania takie jak Ladder Logic, Structured Text  oraz potrafi zaprogramować sekwencję zdarzeń.  Student potrafi zaprogramować i zoptymalizować mikrokontrolery o ograniczonych zasobach (Atiny)
\end{itemize}

\section{Kryteria oceny}

\begin{itemize}
  \item Prezentacja multimedialna z elementami dyskusji
  \item Prezentacja gotowych rozwiązań
  \item Rozwiązywanie zadań
  \item Praca nad projektem
  \item Kryteria oceny
  \item Wykonanie indywidualnego projektu.
\end{itemize}

\section{Metody dydaktyczne}

Wykład, laboratoria, praca własna studenta.

\section{Literatura}

\textbf{Podstawowa:}
\begin{itemize}
  \item "Mikrokontrolery - podstawowe architektury" ,  Mariusz Nowak
  \item "Wprowadzenie do mikrokontrolerów AVR: od elektroniki do programowania", Filip Sala i Marzena Sala-Tefelska
\end{itemize}

\textbf{Uzupełniająca:}
\begin{itemize}
  \item "Zrozumieć małe mikrokontrolery",  M. Sibigtroth
\end{itemize}

\end{document}
